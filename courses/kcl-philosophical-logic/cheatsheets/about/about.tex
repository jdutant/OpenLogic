% Front matter about page
% no need to use \olchapter and \OLEndChapterHook

\documentclass[../../../../include/open-logic-chapter]{subfiles}

\begin{document}

\chapter{About this document}

These Cheatsheets summarize the Philosophical Logic's module core
material. Use them to:
\begin{enumerate}
\item \textbf{Check your understanding}. Go through the relevant part
    after teaching. Does it feel familiar? Can you explain it to yourself, 
    and to another student?
\item Check and improve your memory. You need to know most of the
    material here for the exam. For some parts, you need exact
    knowledge (natural deduction rules); others, to be able to
    reproduce (definition of satisfiablity); yet others, you just need
    to understand enough to use them in exercises (long semantic
    definitions, e.g. truth in first-order logic).
\item Support your exercise practice. Though you eventually need
    to be able to do exercises without the cheatsheet, using them at 
    first helps you understanding and memorizing.
\item Reference. Check definitions and rules; what can be used in exam.
\end{enumerate}

Do NOT use them to:
\begin{enumerate}
\item Learn the material. If you struggle understanding something,
    going over the cheatsheet won't help. You'll mostly by
    \emph{doing exercises}. Also reading the textbook's texts, 
    discussing with others, asking in seminars or office hours. 
\item Guess the exam's scope. \textbf{Not everything on the exam
will be solvable on the basis of the cheatsheet material!} Your best
preparation for the exam are\dots you guessed it, the \emph{exercises}.
\end{enumerate}

\end{document}
