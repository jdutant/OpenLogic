% Part: non-classical-logics Chapter: free-logics Section: introduction

\documentclass[../../../include/open-logic-section]{subfiles}

\begin{document}

\olfileid{frl}{syn}{int}

\olsection{Introduction}

First-order logic has strong existence commitments. The following
are all valid in first-order logic (with identity for the latter two):

\begin{enumerate} 
	\item $\lexists x (\Atom{\Obj F}{x} \lor \lnot \Atom{\Obj F}{x})$ 
	\item $\lexists x (\eq[x][x])$ 
	\item $\lexists x (\eq[a][x])$ 
\end{enumerate}

The first two require that something exists. They are valid because
!!{structure}s of first-order logic have a non-empty domain. They are !!{derivable} from the fact that in first-order logic, we can !!{derive} $\lexists x !A(x)$ from $!A(a)$, together with further facts. For the first, the relevant fact is that substitutions of !!{derivable} formulas of propositional logic formulas are derivable in first-order logic too. Since $!A \lor \lnot !A$ is !!{derivable} in propositional logic, substituting $Atom{\Obj F}{a}$ for $!A$, we have $\Atom{\Obj F}{a} \lor \lnot \Atom{\Obj F}{a}$ which is !!{derivable} in first-order logic. For the second, the relevant fact is that $\eq[x][x]$ is !!{derivable} in first-order logic with identity. 

The third claim requires that every !!{constant} of the language denotes something. It is valid because in first-order !!{structure}s, the interpretations assigns an object of the !!{domain} to each !!{constant}. It is !!{derivable} from the fact that $\eq[a][a]$ is !!{derivable}, together with the fact that we can !!{derive} $\lexists x !A(x)$ from $!A{a}$.

These features of classical first-order logic are controversial because the following do not seem to \emph{logical} truths:

\begin{enumerate}
	\item There is something.
	\item Alice exists. 
\end{enumerate}

Logics have been devised to avoid these implications. \emph{Inclusive logic} allows the domain of quantification to be empty. \emph{Free logic} allows empty !!{constant}s, that is !!{constant}s that do not refer to any existing object.

As soon as our language includes !!{constant}s, then our logic cannot be inclusive without
 being free. So while strictly speaking an inclusive logic need not be free, we do not
 discuss inclusive logic separately.

Free logic can be inclusive or not. Most commonly systems of free logic are devised to
be inclusive as well. The systems below are both free and inclusive. 

\end{document}
