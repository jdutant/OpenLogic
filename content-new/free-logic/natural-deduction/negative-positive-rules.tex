% Part: free-logic
% Chapter: natural-deduction
% Section: negative-positive-rules

\documentclass[../../../include/open-logic-section]{subfiles}

\begin{document}

\olfileid{frl}{nd}{npr}

\olsection{Negative and Postive Free Logic Rules}

\subsection{Negative Free Logic}

Negative free logic has the common rules plus a rule specific to
\emph{atomic} !!{formula}s and identity claims. Where $R$ is a
$n$-place !!{predicate} of~$\Lang L$ and $c$ any constant: , \dots,
$t_n$ are terms of~$\Lang L$, then $\Atom{R}{t_1,\ldots, t_n}$ is an
atomic !!{formula}.

\begin{defish}
    \emph{Negative Free Logic rules}

    \begin{center}        
    \smallskip
    \begin{tabular}{cc}
    \AxiomC{$\Atom{R}{\ldots,c,\ldots}$}
    \RightLabel{\Log{NFL}}
    \UnaryInfC{$\lfrexists c$}
    \DisplayProof
    &
    \AxiomC{$\eq[c][c]$}
    \RightLabel{\Log{NFL\eq}}
    \UnaryInfC{$\lfrexists c$}
    \DisplayProof
    \end{tabular}
    \end{center}

\end{defish}

The contraposed rules are derived immediately. You may use them too:
\begin{center}        
    \smallskip
    \begin{tabular}{cc}
    \AxiomC{$\lnot\lfrexists c$}
    \RightLabel{$\lnot$\Log{NFL}}
    \UnaryInfC{$\lnot\Atom{R}{\ldots,c,\ldots}$}
    \DisplayProof
    &
    \AxiomC{$\lnot\lfrexists c$}
    \RightLabel{$\lnot$\Log{NFL\eq}}
    \UnaryInfC{$\eq/[c][c]$}
    \DisplayProof
    \end{tabular}
\end{center}
\smallskip

\subsection{Positive Free Logic}

Positive Free Logic has the common rules plus the unrestricted
$\Intro{\eq}$ rule from first order logic with identity. 

\begin{defish}
    \emph{Positive Free Logic rule}

    \begin{center}        
    \smallskip
    \AxiomC{}
    \RightLabel{\Intro{\eq} a.k.a. \Log{PFL}}
    \UnaryInfC{$\eq[c][c]$}
    \DisplayProof
    \end{center}

\end{defish}

The weaker \Intro{\eq} \Log{FL} is then redundant. 

Positive free logic does \emph{not} have the \Log{NFL} rule. Thus 
in Positive Free Logic, unlike in Negative Free Logic, 
$\Obj{F}a\land\lnot\lfrexists a$ (`$a$ is $\Obj{F}$ and nothing is $a$')
is consistent. 

\subsection{Negative Free Logic looses the Uniform Substitution Rule}


Because of the rule \Log{NFL}, Negative Free Logic violates the 
\emph{uniforn substitution rule} that other logics uphold. 

\begin{defn}[Uniform Substitution Rule]
    If $!A$ is a logical truth, and $!A[!B/\Obj{P}]$ is the 
    result of uniformly substituting $!B$ for 
    the atomic !!{formula} $\Obj{P}$ in $!A$, then 
    $!A[!B/\Obj{P}]$.
\end{defn}

\begin{ex}
The Uniform Substitution Rule holds in propositional logic. For instance
$\Obj{P}\lor\lnot\Obj{P}$ is a logical truth (where $\Obj{P}$ is 
atomic), but so is any !!{formula} of the form $!A\lor\lnot!B$.
\end{ex}

The same holds in modal logic, first order logic, positive free logic, 
and many others.

Negative Free Logic violates the substitution rule. Where $\Obj{P}a$
is \emph{atomic}, we have the logical truth of Negative Free Logic:
$$\vdash_{\Log{NFL}}Obj{P}a\lif \lfrexists a$$
However, if we substitute the non-atomic $\lnot\Obj{P}a$ for $\Obj{P}a$,
the result is \emph{not} a logical truth of Negative Free Logic:
$$\nvdash_{\Log{NFL}}\lnot Obj{P}a\lif \lfrexists a$$

Is violating uniform substitution a problem? Two contrasting views on the issue:
\begin{itemize}
    \item The Uniform Substitution Rule is essential to logic. Logical truths are 
    true in virtue of their form. The rule captures that idea, because it 
    says that if $!A$ is a logical truth, then any !!{formula} with \emph{the 
    same form} should be a logical truth too. Because the principles
    of Negative Free Logic treat atomic vs non-atomic !!{formula}s differently, 
    they are not formal hence not logical.
    \item The difference between atomic and complex !!{formula}s \emph{is}
    a matter of form. Negative Free Logic upholds the general idea that 
    if $!A$ is a logical truth, then any !!{formula} of the same `form'
    is a logical truth; it just denies that replacing atomic sub-!!{formula}s
    by complex ones preserves the form. 
\end{itemize}

\subsection{Note on another Negative Free Logic}

To simplify matters, we only consider the Negative Free Logic above.

You should be aware that there is another way of developping Negative 
Free logic that allows $c=c$ to be a logical truth even if $c$ doesn't 
exist. I'll discuss that variant briefly below, but you can ignore it.
The bottom line is: it's ugly and complicates things. 

On that alternative Negative Free Logic, we adopt the unrestricted
\Intro{\eq} rules as in Positive Free Logic. So $\eq[c][c]$ is a
logical truth; it holds even if $c$ is empty (doesn't denote
anything). The logic is still a \emph{negative} free logic because it
retains \Log{NFL}: \emph{atomic} claims like $\Obj{F}a$ cannot be true
if $a$ is empty. But we drop the analogue rule for $\eq$,
\Log{NFL\eq}. Thus in that version of Negative Free Logic:
\begin{itemize}
\item $\Obj{F}a\land\lnot\lfrexists a$ is inconsistent, as in our
original version of Negative Free Logic.
\item $\eq[a][a]\land\lfrexists a$ is not inconsistent.
\end{itemize}
So far this might seem nice. Unfortunately, ugly complications arise.
To make $\eq[a][a]$ a logic truth, we have to change the NFL semantics
for $\eq$. We face a choice: keeping the negative semantics in which
empty names don't refer to any object, or adopting the postive semantics 
on which empty names refer to 'non-existing' objects (the `outer' domain). 
Both options have serious downsides. 

If we keep the negative semantics, to make $\eq[c][c]$ a logical truth
we say that it's always true even if $c$ is empty. But when $c,d$ are 
both empty, what do we say about $\eq[c][d]$? The negative semantics 
has only two options: all of these are true, or all false. If all 
false, not only that feels artificial, but that requires a new ugly
rule: 
\begin{prooftree}
    \AxiomC{$\eq[c][d]$}
    \RightLabel{where $c$ and $d$ are \emph{distinct} constants}
    \UnaryInfC{$\lfrexists c$}    
\end{prooftree}
If all true, that might still feel artificial, and we also have to 
accept a new unappealing rule:
\begin{prooftree}
    \AxiomC{$\lnot\lfrexists c$}
        \AxiomC{$\lnot\lfrexists d$}
    \RightLabel{??}
    \BinaryInfC{$\eq[c][d]$}    
\end{prooftree}
Bottom line: with the negative semantics, saying that $\eq[c][c]$ 
is false when $c$ is empty might sound counterintuitive, but the 
other options aren't great either.

Let's consider the positive semantics instead. On that option we could
make $\eq[c][c]$ a logical truth even when $c$ is empty, \emph{and} 
saying that when $c,d$ are empty, $\eq[c][d]$ may or may not be true.
We would avoid the two arbitrary-seeming options above. This option 
is Positive Free Logic (rules and semantics) with the added
the signature law of Negative Free Logic, $\Log{NFL}$. That solution 
seems to have the worst of both worlds. Because it has the \Log{NFL} rule, 
its logic doesn't obey the nice principle of uniform substitution (unlike 
propositoinal logic, modal logic, first order logic and Positive Free Logic).
Because it uses the semantics of Positive Free Logic, it has a semantics
that assigns objects to empty names, hence a semantics that we cannot 
take at face value (unless we adopt a strange Meinongian metaphysics 
of things that are but don't exist). It's not clear what the motivation
would be for adopting that variant of Negative Free Logic over the 
simpler Positive Free Logic. 

\end{document}