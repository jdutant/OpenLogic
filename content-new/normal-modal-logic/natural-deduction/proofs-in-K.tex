% Part: normal-modal-logic
% Chapter: natural-deduction 
% Section: proofs-in-K

\documentclass[../../../include/open-logic-section]{subfiles}

\begin{document}

\olfileid{nml}{nd}{prk}

\olsection{Examples of \usetoken{P}{derivation} in \Log{K}}

\begin{ex}
!!^{derivation} of $\Proves[\Log{K}] \Box(!A\lif!A)$.
\begin{prooftree}
\AxiomC{$\Discharge{!A}{1}$}
\DischargeRule{\Intro{\lif}}{1}
\UnaryInfC{$!A\lif!A$}
\RightLabel{$\Box$\Ax{K}}
\UnaryInfC{$\Box(!A\lif!A)$}
\end{prooftree}
\end{ex}

\begin{explain}
At the second step with have a !!{derivation} of $!A\lif !A$ without
undischarged assumptions. We can therefore apply rule $\Box$\Ax{K} 
as a special case without $\Box\ldots$ assumptions. This last step
would not be legit if some assumptions had not been discharged at
that stage.
\end{explain}

\begin{ex}
    !!^{derivation} of $\Box(!A\land!B)\Proves[\Log{K}] \Box!A\land\Box!B$.
    \begin{prooftree}
    \AxiomC{$\Box!A\land !B$}
        \AxiomC{$\Discharge{!A\land!B}{1}$}
        \RightLabel{\Elim{\land}}
        \UnaryInfC{$!A$}
    \DischargeRule{$\Box$\Ax{K}}{1}
    \BinaryInfC{$\Box!A$}
            \AxiomC{$\Box!A\land !B$}
                \AxiomC{$\Discharge{!A\land!B}{2}$}
                \RightLabel{\Elim{\land}}
                \UnaryInfC{$!B$}
            \DischargeRule{$\Box$\Ax{K}}{2}
            \BinaryInfC{$\Box!B$}
        \RightLabel{\Intro{\land}}
        \BinaryInfC{$\Box!A\land\Box!B$}
    \end{prooftree}

\end{ex}

\begin{prob}
Give !!a{derivation} of $\Box!A\land\Box!B\Proves[\Log{K}]
\Box(!A\land!B)$.
\end{prob}

\begin{ex}
!!^{derivation} of $\Proves[\Log{K}] \Box(!A\lif!B)\lif(\Box!A\lif\Box!B)$.

\begin{prooftree}
\AxiomC{$\Discharge{\Box(!A\lif!B)}{3}$}
    \AxiomC{$\Discharge{\Box!A}{2}$}
        \AxiomC{$\Discharge{!A\lif!B}{1}$}
        \AxiomC{$\Discharge{!A}{1}$}
        \RightLabel{\Elim{\lif}}
        \BinaryInfC{$!B$}
\DischargeRule{$\Box$\Ax{K}}{1}
\TrinaryInfC{$\Box!B$}
\DischargeRule{\Intro{\lif}}{2}
\UnaryInfC{$\Box!A\lif\Box!B$}
\DischargeRule{\Intro{\lif}}{3}
\UnaryInfC{$\Box(!A\lif!B)\lif(\Box!A\lif\Box!B)$}
\end{prooftree}
\end{ex}

\iftag{prvDiamond}{
\begin{ex}
    We can derive further Duality rules, such as:

    \bigskip
    \begin{tabular}{cc}
        \AxiomC{}\DeduceC{$\lnot\Diamond!A$}
        \RightLabel{Dual}
        \UnaryInfC{$\Box\lnot!A$}
        \DisplayProof
    &
        \AxiomC{}\DeduceC{$\Box\lnot!A$}
        \RightLabel{Dual}
        \UnaryInfC{$\lnot\Diamond!A$}
        \DisplayProof
    \end{tabular}
    \bigskip

   For $\lnot\Diamond!A \Proves[K] \Box\lnot!A$:
   \begin{prooftree}
        \AxiomC{$\lnot\Diamond!A$}
            \AxiomC{$\Discharge{\lnot\Box\lnot!A}{1}$}
            \RightLabel{\Intro{\Diamond}}
            \UnaryInfC{$\Diamond!A$}
        \DischargeRule{\Elim{\lnot}}{1}
        \BinaryInfC{$\Box\lnot!A$}
    \end{prooftree}

    For $\Box\lnot!A \Proves[K] \lnot\Diamond!A$:
    \begin{prooftree}
        \AxiomC{$\Box\lnot!A$}
            \AxiomC{$\Discharge{\Diamond!A}{1}$}
            \RightLabel{\Elim{\Diamond}}
            \UnaryInfC{$\lnot\Box\lnot!A$}
        \DischargeRule{\Elim{\lnot}}{1}
        \BinaryInfC{$\lnot\Diamond!A$}
    \end{prooftree}

\end{ex}
}{}

\begin{ex}
\iftag{prvDiamond}{
!!^{derivation} of $\Box(!A\lif!B) \Proves[\Log{K}] \Diamond!A\lif\Diamond!B$.
\begin{prooftree}
    \AxiomC{$\Box(!A\lif!B)$}
    \AxiomC{$\Discharge{\Box\lnot!B}{3}$}
        \AxiomC{$\Discharge{!A\lif!B}{2}$}
        \AxiomC{$\Discharge{!A}{1}$}
        \RightLabel{\Elim{\lif}}
        \BinaryInfC{$!B$}
        \AxiomC{$\Discharge{\lnot!B}{2}$}
        \DischargeRule{\Intro{\lnot}}{1}
        \BinaryInfC{$\lnot!A$}
    \DischargeRule{$\Box$\Ax{K}}{2}
    \TrinaryInfC{$\Box\lnot!A$}
        \AxiomC{$\Discharge{\Diamond!A}{4}$}
        \RightLabel{\Elim{\Diamond}}
        \UnaryInfC{$\lnot\Box\lnot!A$}
        \DischargeRule{\Intro{\lnot}}{3}
    \BinaryInfC{$\lnot\Box\lnot!B$}
    \RightLabel{\Intro{\Diamond}}
    \UnaryInfC{$\Diamond!B$}
    \DischargeRule{\Intro{\lif}}{4}
    \UnaryInfC{$\Diamond!A\lif\Diamond!B$}
\end{prooftree}
\begin{explain}
We need to show that, given $\Box(!A\lif!B)$, we can derive $\Diamond!B$
from $\Diamond!A$. Given the Duality rules, this requires deriving 
$\lnot\Box\lnot!B$ from $\lnot\Box\lnot!A$. Our strategy is to
establish the equivalent contraposition first: given $\Box(!A\lif!B)$,
we can derive $\Box\lnot !A$ from $\Box\lnot !B$. 

The $\Box$\Ax{K} step is legitimate because at the time when it is applied 
$!A\lif!B$ and $\lnot!B$ are the only undischarged assumptions of 
the derivation on the right ($!A$ has been discharged already).
\end{explain}
}{
!!^{derivation} of $\Box(!A\lif!B) \Proves[\Log{K}] \lnot\Box!B\lif\lnot\Box!A$.
    \begin{prooftree}
        \AxiomC{$\Box(!A\lif!B)$}
        \AxiomC{$\Discharge{\Box\lnot!B}{3}$}
            \AxiomC{$\Discharge{!A\lif!B}{2}$}
            \AxiomC{$\Discharge{!A}{1}$}
            \RightLabel{\Elim{\lif}}
            \BinaryInfC{$!B$}
            \AxiomC{$\Discharge{\lnot!B}{2}$}
            \DischargeRule{\Intro{\lnot}}{1}
            \BinaryInfC{$\lnot!A$}
        \DischargeRule{$\Box$\Ax{K}}{2}
        \TrinaryInfC{$\Box\lnot!A$}
        \DischargeRule{\Intro{\lif}}{3}
        \UnaryInfC{$\Box\lnot!B\lif\Box\lnot!A$}
    \end{prooftree}
}
\end{ex}

\begin{prob}
    \iftag{prvDiamond}
    Give a !!{derivation} of \iftag{prvDiamond}
    {$\Diamond(!A\land!B)\Proves[\Log{K}]\Diamond!A\land\Diamond!B $}
    {$\lnot\Box(!A\land!B)\Proves[\Log{K}]\lnot\Box!A\lor\lnot\Box!B$}. 
\end{prob}

\iftag{prvDiamond}{
\begin{ex}
The following rule is derived:
\begin{prooftree}
    \AxiomC{}
    \DeduceC{$\Diamond\lfalse$}
    \RightLabel{$\Diamond\lfalse$}
    \UnaryInfC{$\lfalse$}
\end{prooftree}

!!^{derivation}:

\begin{prooftree}
    \AxiomC{$\Discharge{\lfalse}{1}$}
    \DischargeRule{\Elim{\lnot}}{1}
    \UnaryInfC{$\lnot\lfalse$}
    \RightLabel{$\Box$\Ax{K}}
    \UnaryInfC{$\Box\lnot\lfalse$}
        \AxiomC{$\Diamond\lfalse$}
        \RightLabel{\Elim{\Diamond}}
        \UnaryInfC{$\lnot\Box\lnot\lfalse$}
    \RightLabel{\Intro{\lfalse}}
    \BinaryInfC{$\lfalse$}
    \end{prooftree}

\end{ex}
}{}

\iftag{prvDiamond}{

\begin{ex}
The following $\Diamond$\Ax{K} rule, analogue to $\Box$\Ax{K}, is derived: 
\begin{prooftree}
    \AxiomC{}\DeduceC{$\Diamond!A$}
    \AxiomC{}\DeduceC{$\Box!B$}
    \AxiomC{$\Discharge{!A}{n}\Discharge{!B}{n}$\emph{at most}}\DeduceC{$!C$}
    \DischargeRule{$\Diamond$\Ax{K}}{n}
    \TrinaryInfC{$\Diamond!C$}
\end{prooftree}
As with the $\Box$\Ax{K} rule, the sub-!!{derivation} on the right must
have at most $!A,!B$ as undischarged assumptions, and the $\Box!B$ 
premise is optional. However, unlike the $\Box$\Ax{K} rule, the premise
$\Diamond !A$ is required. This can be seen from the !!{derivation}s

Let $\mathcal{D}$ be !!a{derivation} of $!C$ from at most $!A$ and $!B$.
We have:
\begin{prooftree}
\AxiomC{}
\DeduceC{$\Diamond!A$}
\RightLabel{\Elim{\Diamond}}
\UnaryInfC{$\lnot\Box\lnot!A$}
    \AxiomC{}
    \DeduceC{$\Box!B$}
    \AxiomC{$\Discharge{\Box\lnot!C}{3}$}
        \AxiomC{$\Discharge{!A}{1},\Discharge{!B}{2}$\emph{ at most}}
        \noLine
        \UnaryInfC{$\mathcal{D}$}
        \noLine
        \UnaryInfC{$!C$}
            \AxiomC{$\Discharge{\lnot!C}{2}$}
        \DischargeRule{\Intro{\lnot}}{1}
        \BinaryInfC{$\lnot!A$}
    \DischargeRule{$\Box$\Ax{K}}{2}
    \TrinaryInfC{$\Box\lnot!A$}
\DischargeRule{\Intro{\lnot}}{3}
\BinaryInfC{$\lnot\Box\lnot!C$}
\RightLabel{\Intro{\Diamond}}
\UnaryInfC{$\Diamond!C$}
\end{prooftree}
The $\Box$\Ax{K} step is correct because by the time we get to it $!A$ has
been discharged and the only undischarged assumptions in its 
rightmost sub-!!{derivation} are \emph{at most} $!B$, 
$\lnot!C$. If $\mathcal{D}$ had any undischarged assumption beyond $!A$,
$!B$ this would not be so and the $\Box$\Ax{K} step would be incorrect. 
If $\mathcal{D}$ only has one of $!A,!B$ undischarged, or neither,
the $\Box$\Ax{K} step is still correct. 

The $\Box!B$ premise is optional; the $\Diamond!A$ premise is mandatory.
Without $\Box!B$ the proof is still correct, provided that 
$\mathcal{D}$ then uses at most $!A$ as undischarged assumption:
\begin{prooftree}
    \AxiomC{}
    \DeduceC{$\Diamond!A$}
    \RightLabel{\Elim{\Diamond}}
    \UnaryInfC{$\lnot\Box\lnot!A$}
        \AxiomC{$\Discharge{\Box\lnot!C}{3}$}
            \AxiomC{$\Discharge{!A}{1}$\emph{ at most}}
            \noLine
            \UnaryInfC{$\mathcal{D}$}
            \noLine
            \UnaryInfC{$!C$}
                \AxiomC{$\Discharge{\lnot!C}{2}$}
            \DischargeRule{\Intro{\lnot}}{1}
            \BinaryInfC{$\lnot!A$}
        \DischargeRule{$\Box$\Ax{K}}{2}
        \BinaryInfC{$\Box\lnot!A$}
    \DischargeRule{\Intro{\lnot}}{3}
    \BinaryInfC{$\lnot\Box\lnot!C$}
    \RightLabel{\Intro{\Diamond}}
    \UnaryInfC{$\Diamond!C$}
\end{prooftree}
However, the premise $\Diamond!A$ is mandatory. It is needed to derive 
$\lnot\Box\lnot!A$, thereby contradicting $\Box\lnot!A$ and concluding
$\lnot\Box\lnot!C$ (and therefore $\Diamond!C$) by \emph{reductio}.
\end{ex}
}{}


\end{document}