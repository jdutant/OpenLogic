% Part: cheat-sheets
% Chapter: modal-propositional-logics
% Section: remark-on-satisfication-notation

\documentclass[../../../../include/open-logic-section]{subfiles}

\begin{document}

\olfileid{chs}{pml}{rsn}

\olsection{Remark on the notation for satisfaction and entailment}

In earlier version of the KCL Philosophical Logic PDF you might have 
noticed two slightly different symbols in the semantics: $\vDash$ and 
$\Vdash$. Following MacFarlane, we simply use the same symbol $\vDash$. 
This note explains why some textbooks use two symbols, what they mean, 
and why it's fine to just use one. 

Logical consequence is a relation \emph{between (premise) !!{formula}s
and (conclusion) !!{formula}s}. We have two types of consequence 
relation:
\begin{itemize}
\item Proof-theoretic (aka syntactic). !!^{derivability}, noted $\Gamma \Proves !A$.
\item Semantic. Entailement (aka validity), noted $\Gamma \Entails !A$.
\end{itemize}
In semantics, we also have the notion of \emph{satisfaction}. This 
is a relation between \emph{!!{structure}s} and !!{formula}s:
roughly put, a truth-making relation---the !!{formula}
is `true at' or `true in' the !!{structure}.
\begin{itemize}
\item Satisfaction. $\mSat{M}{!A}$.
\end{itemize}

Some textbooks use the same symbol $\vDash$ for 
both entailement and satisfaction. The ambiguity is acceptable because
both notions have to do with semantics, because one is defined in terms 
of the other ($\Gamma$ entails $!A$ iff no !!{structure} satisfies all 
!!{formula}s in $\Gamma$ without satisfying $!A$ too), and because we 
can always figure out from context which one is meant. 

Some textbooks use two distinct symbols, $\vDash$ and $\Vdash$: the
former for entailement and the latter for satisfaction---never the
other way round. (The symbol $\Vdash$ is sometimes called the
`forcing' symbol because it's mostly used in advanced parts of set
theory where it denotes yet another notion, forcing.)

MacFarlane uses just one symbol. The Open Logic textbook, on which 
KCL Philosophical Logic is based, uses two symbols in its modal logic
chapter but only one symbol elsewhere. Thus you might have seen $\Vdash$
in the modal logic chapter and wondered why I used $\vDash$ instead in 
class. I've now updated KCL Philosophical Logic so that it only uses
one symbol throughout, like MacFarlane. 

For exam purposes it only matters that you use distinct symbols for
\begin{itemize}
    \item the proof-theoretic notion of !!{derivability}, with \emph{single-lined} symbol
        $\vdash$
    \item the semantic notions of entailement and satisfaction, with \emph{double-lined}
        symbols $\vDash$ and/or $\Vdash$. 
\end{itemize}
Whether you use distinct symbols or not for entailement vs satisfaction
doesn't matter. Either use $\vDash$ for both,
or use $\vDash$ for entailement and $\Vdash$ for satisfaction. It's 
wrong to use $\Vdash$ for entailement but we won't penalize you for it.

\end{document}