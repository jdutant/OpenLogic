% Part: propositional-logic
% Chapter: propositional-logic
% Section: expressive-power

\documentclass[../../../include/open-logic-section]{subfiles}

\begin{document}

\olfileid{pl}{mth}{exp}

\olsection{Expressive Power}

Propositional logic can express all finary bivalent truth functions. 
This section explains that idea.

\subsection{Truth-functional !!{operator}s}

Whether a sentence `$!A$ and $!B$' is true \emph{only} turns on whether
each component $!A$, $!B$ is true. That is all we need to know about
$!A$, $!B$ to tell whether the sentence `$!A$ and $!B$' is true. The 
same is not true for a sentence `$!A$ because $!B$'. Granted, if we
are merely told that $!A$ is false or $!B$ is false, we can conclude
that `$!A$ because $!B$' is false. But if we are merely told that $!A$
and $!B$ are both true, that is not enough to tell whether `$!A$ because $!B$'
is true. There are example of true sentences of that form where both $!A$ 
and $!B$ are true (`the Sun is warm because it emits radiation') 
and examples of false sentences of that form where both $!A$ 
and $!B$ are true (`the Sun is warm because the Earth rotates around it').

We call this feature of `and' \emph{truth-functionality}. It is that 
feature that allows us to specify the meaning of `and' fully with
a truth table. By contrast, you can't give the full meaning of `because'
with a truth table. 

\subsection{Truth functions}

Mathematically speaking, a truth table describes 
a mapping or \emph{function} from truth values to truth value. 
A \emph{binary} truth function maps two truth values to truth value.
The meaning of `and' is fully specified by the binary truth function 
$f_{\text{and}}$ described thus:

\smallskip
\begin{tabular}{lll}
    If $v_1$ is: & and $v_2$ is: & then $f_{\text{and}}(v_1,v_2)$ is: \\
    \hline
    $\True$ & $\True$ & $\True$ \\
    $\True$ & $\False$ & $\False$ \\
    $\False$ & $\True$ & $\False$ \\
    $\False$ & $\False$ & $\False$ \\
\end{tabular}
\smallskip

Our !!{operator}s `$\lnot$', `$\land$', `$\lor$', `$\lif$', `$\liff$' all 
correspond to truth functions in that way. $\lnot$ is unary: it
corresponds to a truth function from \emph{one} truth value to truth
value. The others are binary: they express a truth function from
\emph{two} truth values to truth value.

We can of course imagine more binary truth functions than the ones
associated with our !!{operator}s. For instance,
the truth function $f_{NAND}$ described thus:
\smallskip
\begin{tabular}{lll}
    If $v_1$ is: & and $v_2$ is: & then $f_{NAND}(v_1,v_2)$ is: \\
    \hline
    $\True$ & $\True$ & $\False$ \\
    $\True$ & $\False$ & $\True$ \\
    $\False$ & $\True$ & $\True$ \\
    $\False$ & $\False$ & $\True$ \\
\end{tabular}
\smallskip

(This function is often called NAND, short for `not and', for reasons 
that will become clear shortly.)

We could introduce a !!{operator} for this truth function. There is one,
in fact: the \emph{Sheffer stroke}, written $|$. A formula $!A|!B$ is
then true just if one of $!A$ or $!B$ is not true. 

Do we need this !!{operator}? Is our language lacking without it? No,
because we can \emph{express} the same truth function using $\lnot$
and $\land$. For a formula $\lnot (!A\land!B)$ is true just if one of
$!A$ or $!B$ isn't true. Thus we can write $\lnot (!A\land!B)$ instead
of $!A|!B$. We don't need the Sheffer stroke. 

Mathematically put: let $f_{AND}$ is the binary truth function 
associated with conjunction and $f_{NOT}$ is the unary truth function 
associated with negation. Mapping two truth values with $f_{NAND}$ is
the same as mapping them to a truth value with $f_{AND}$ and then 
mapping the result with $f_{NOT}$: $f_{NAND}(v)=f_{NOT}(f_{AND}(v))$, 
where $v$ is $\True$ or $\False$.

\subsection{Expressive power}

We have seen that our !!{operator}s are associated with truth functions,
and that combinations of !!{operator}s allow us to express \emph{some}
further truth functions. We ask now: can they express \emph{all} truth
functions? The answer is yes. More precisely, they can express all
finite bivalent truth functions. 

Let us illustrate first a language that is \emph{not} able to express
all truth functions. 

\begin{prop}\ollabel{prop:exp-inadequate} 
    Let $\Lang{L}^*$ be language with $\land$ and $\lor$ as the only
    !!{operator}s. $\Lang{L}^*$ cannot express negation:
    that is, there is no formula $!A$ in
    this language, however complex, that is false if $\Obj{\pvar{A}}$
    is true.
\end{prop}

\begin{prob}
    Establish \olref[pl][mth][exp]{prop:exp-inadequate}. Hint: consider
    the valuation that makes all !!{propositional variable}s true.
\end{prob}
    
Consider all possible truth functions. Unary truth functions correspond
to truth tables with two lines. How many are them? (Describe them.)
Binary truth functions correspond to truth tables with four lines.
There are 16. Ternary ones with truth tables with eight lines. How 
many are them? How many $n$-ary truth functions are there, for 
arbitrary $n$?

Let $f$ be an arbitrary ternary truth function. It will correspond 
to a truth table:

\smallskip \noindent
\begin{tabular}{llll}
    If $v_1$ is: & $v_2$ is: & and $v_3$ is & then $f_{NAND}(v_1,v_2)$ is: \\
    \hline
    $\True$ & $\True$ & $\True$ & \dots \\
    $\True$ & $\True$ & $\False$ & \dots \\
    $\True$ & $\False$ & $\True$ & \dots \\
    $\True$ & $\False$ & $\False$ & \dots \\
    $\False$ & $\True$ & $\True$ & \dots \\
    $\False$ & $\True$ & $\False$ & \dots \\
    $\False$ & $\False$ & $\True$ & \dots \\
    $\False$ & $\False$ & $\False$ & \dots \\
\end{tabular}
\smallskip

Suppose, for a start, that the truth function $f$ maps exactly one line
to $\True$, all the other to false. Let it be the third line, say:

\smallskip \noindent
\begin{tabular}{llll}
    If $v_1$ is: & $v_2$ is: & and $v_3$ is & then $f_{NAND}(v_1,v_2)$ is: \\
    \hline
    $\True$ & $\True$ & $\True$ & $\False$ \\
    $\True$ & $\True$ & $\False$ & $\False$ \\
    $\True$ & $\False$ & $\True$ & $\True$ \\
    $\True$ & $\False$ & $\False$ & $\False$  \\
    $\False$ & $\True$ & $\True$ & $\False$  \\
    $\False$ & $\True$ & $\False$ & $\False$  \\
    $\False$ & $\False$ & $\True$ & $\False$  \\
    $\False$ & $\False$ & $\False$ & $\False$  \\
\end{tabular}
\smallskip

Can we express it with our !!{operator}s $\lnot$, $\land$, $\lor$ only?
To do so, we need to find a !!{formula} such that, given three
!!{propositional variable}s
$\Obj{\pvar{A}}_1,\Obj{\pvar{A}}_2,\Obj{\pvar{A}}_3$, is $\True$ just
if $\Obj{\pvar{A}}_1$ is $\True$, $\Obj{\pvar{A}}_2$ is $\False$ and
$\Obj{\pvar{A}}_3$ is $\True$. There is obviously one, namely:
$$\Obj{\pvar{A}}_1\land\lnot\Obj{\pvar{A}}_2\land\Obj{\pvar{A}}_3$$
It's easy to see that whichever single line $f$ makes true, we can
express the same thing with a conjunction of $\Obj{\pvar{A}}_i$ or 
negations $\Obj{\pvar{A}}_i$.

Now suppose the truth function $f$ makes true exactly two lines, say
the third and the fifth:

\smallskip\noindent
\begin{tabular}{llll}
    If $v_1$ is: & $v_2$ is: & and $v_3$ is & then $f_{NAND}(v_1,v_2)$ is: \\
    \hline
    $\True$ & $\True$ & $\True$ & $\False$ \\
    $\True$ & $\True$ & $\False$ & $\False$ \\
    $\True$ & $\False$ & $\True$ & $\True$ \\
    $\True$ & $\False$ & $\False$ & $\False$  \\
    $\False$ & $\True$ & $\True$ & $\True$  \\
    $\False$ & $\True$ & $\False$ & $\False$  \\
    $\False$ & $\False$ & $\True$ & $\False$  \\
    $\False$ & $\False$ & $\False$ & $\False$  \\
\end{tabular}
\smallskip

Can we express this? We know how to give !!a{formula} that is $\True$
only at the fifth line (with 
$\Obj{\pvar{A}}_1,\Obj{\pvar{A}}_2,\Obj{\pvar{A}}_3$ for $v_1,v_2,v_3$, 
respectively), namely:
$$\Obj{\pvar{A}}_1\land\lnot\Obj{\pvar{A}}_2\land\Obj{\pvar{A}}_3$$
We can give one that is $\True$ at exactly the fifth line:
$$\lnot\Obj{\pvar{A}}_1\land\Obj{\pvar{A}}_2\land\Obj{\pvar{A}}_3$$
Hence we can simply use their \emph{disjunction}:
$$(\Obj{\pvar{A}}_1\land\lnot\Obj{\pvar{A}}_2\land\Obj{\pvar{A}}_3)
\lor(\lnot\Obj{\pvar{A}}_1\land\Obj{\pvar{A}}_2\land\Obj{\pvar{A}}_3)$$
For that will be $\True$ exactly when one of the two disjuncts are $\True$,
hence exactly at the third and fifth line.

What if $f$ maps three lines to $\True$? Well, we then use a disjunction
of three formulas. And so on if $f$ maps four or more lines to $\True$.

What if $f$ maps no lines to $\True$. Well, we then use $\lfalse$ 
or an arbitrary contradiction. 

It's easy to see that we could use the same strategy if $f$ was a
\emph{quartenary} truth function. Its truth table would be 16 lines long.
For each line, we could find conjunction of !!{propositional variable}s
and their negation that is $\True$ at that line only. We can then 
express $f$ using a disjunction of one or more of:
\begin{itemize}
    \item $\lfalse$
    \item a conjunction of !!{propositional variable}s
    and their negation
\end{itemize}
Such !!a{formula} is called a \emph{disjunctive normal formula}.

The strategy generalize to any finitary truth function. 

Thus, propositional logic expresses all finitary truth functions.

Throughout, we have assumed \emph{bivalence}: there are only two truth
values, $\True$ and $\False$. According to some logics, there are more
truth values. If so, there are more truth functions. We have only 
show that our !!{operator}s are enough to express all truth functions if 
there are more than two values. Hence, to be precise, we have established that
propositional logic expresses all finite, bivalent truth functions.

\end{document}


