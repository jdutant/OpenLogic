% Part: exercise-booklet
% Chapter: propositional-modal-logic
% Section: applications

\documentclass[../../../../include/open-logic-section]{subfiles}

\begin{document}

\olfileid{exb}{pml}{app}

\olsection{Applications of Modal Logic}

\begin{prob}[From principles to modal system]
Suppose we write `$\Box !A$' for `Sam believes that $ !A$' and
we agree that the following principles are correct / incorrect:

\begin{itemize}
\item Sam believes the logical consequences of what she believes. That is,
if Sam believes all of $ !A_{1}$, $ !A_{2}$, $\ldots$, $ !A_{n}$
and $ !A_{1}$, $\ldots$, $ !A_{n}$ entail $!B$, then Sam believes
$!B$. CORRECT
\item If Sam believes $!A$, she believes that she believes $!A$. CORRECT
\item If Sam believes $!A$, then $!A$. INCORRECT 
\item If Sam believes $!A$, she does not believe not-$!A$. CORRECT
\item If Sam does not believe $!A$, she believes that she does not believe
$ !A$. CORRECT
\end{itemize}
Which logic does (Sam) belief obey, on our view?

\emph{Note}. It is not one of the standard systems \Log{K}, \Log{D},
\Log{T}, \Log{B}, \Log{S4}, \Log{S5}. Rather,
you should name the logic by listing the principles it obeys,
starting with \Ax{K}, e.g. \Log{KTB4}.
\end{prob}

\begin{prob}[From modal system to principles]
Suppose we write `$\Box !A$' for `Sam knows that $ !A$' and we
think $\Box$, under that reading, has the logic \Log{S4}. What
does our view say about (Sam's) knowledge? 
\end{prob}

\begin{prob}
Suppose we think of deontic notions in the following way. Among all
possible worlds there is a set of ideal worlds where everything happens
as it should. At any given world $w$, it is obligatory that $!A$
iff $!A$ holds at all the ideal worlds; it is permitted that $!A$
iff it is not obligatory that not-$!A$. 

\begin{enumerate}
	\item Suppose we are given $W$ the set of possible worlds, $I_{W}$ the
	(non-empty) subset of $W$ that are the ideal worlds, and $V$ a valuation
	of the sentence letters in each world. How do you define the accessibility
	relation $R$ to get a model $\tuple{W,R,V}$ that captures
	the picture above? 
	\item What are the properties of $R$ on that picture: it is serial, reflexive,
	symmetric, transitive? Briefly justify each answer. 
	\item Which modal logic do deontic notions obey on that picture? (Give the
	strongest of the standard modal logics \Log{K}, \Log{D},
	\Log{T}, \Log{B}, \Log{S4}, \Log{S5} that applies.) 
	\item For each of the principles or claims below, say whether is true in
	your model. Briefly justify your answers.
		\begin{enumerate}
			\item If it is true that $!A$, it is permitted that $!A$.
			\item If it is permitted that $!A$, it is obligatory that it is permitted
			that $!A$.
			\item It is obligatory that: it is raining or it is not raining.
			\item If it is obligatory that Boris sends the letter, then it is obligatory
			that: Boris sends the letter or Boris burns it. 
			\item If it is obligatory that Ali replaces the lamp he has broken, then
			it is obligatory that Ali has broken the lamp.
		\end{enumerate}
\end{enumerate}
\emph{Note}. Treat `if \ldots then' as a material conditional. A
principle or claim is true in your model iff it (or its traduction
in a formula) is true at every world of the model. For the last three
questions, suppose that your valuation $V$ captures the meaning of
the sentences in question; for instance, if $\Obj{\pvar{A}}$ is the proposition
that \emph{Ali replaces the lamp he has broken and $\Obj{\pvar{B}}$ is the proposition
that \emph{Ali has broken the lamp}, every world where $V$ counts
$\Obj{\pvar{A}}$ as true is a world where $\Obj{\pvar{B}}$ is true.}
\end{prob}

\begin{prob}
Suppose we're only talking about temperature. Our `possible worlds'
only differ in what the temperature is; $w_{1}$ is the world where
the temperature is $1$º C, $w_{-26}$ is the world where the temperature
is $-26$º C, and so on. For simplicity, we assume that we have one
world per whole-number temperature (no matter how low or high). Our
sentence letters are propositions about temperature: for instance,
if $P$ is true just at $w_{-1}$, $w_{-2}$, $w_{-3}$, it is the
proposition that the temperature is between $-1$º~C and $-3$º~C,
if $Q$ is true just at $w_{1}$ and $w_{25}$ is it the proposition
that the temperature is either $1$º C or $25$º C, and so on. 

Now suppose we think of `definitely' as follows. At a given world
$w$, it is definitely the case that $!A$---'definitely $!A$',
for short---iff $!A$ holds at worlds whose temperature is plus
or minus $5$º C the temperature of $w$. We write $\ldet!A$
for `definitely $!A$', but $\ldet$ will work like a $\Box$
operator in modal logic. 

\begin{enumerate}
\item Given $W$ the set of worlds described as above, $V$ a valuation
of the sentence letters, define an accessibility relation $R$ to
capture the notion of `definitely'. 
\item At given world $w$, it is \emph{indefinite} that $!A$---`indefinitely
$!A$', for short---iff among the worlds whose temperature is plus
or minus $5$º C that of $w$, $!A$ holds at some and fails at
others. Write $\lindet!A$ for `it is indefinite that $!A$'.
Define $\lindet!A$ in terms of $\ldet!A$.
\item What are the properties of $R$ on that picture: it is serial, reflexive,
symmetric, transitive? Briefly justify each answer. 
\item For each of the principles or claims below, say whether it holds in
your model. (Treat `if $\ldots$ then' as a material conditional.)
Briefly justify your answers.
	\begin{enumerate}
		\item If definitely $!A$ then $!A$.
		\item If $!A$ then definitely $!A$.
		\item If indefinitely $!A$ then not-$!A$.
		\item Definitely: $!A$ or not $!A$.
		\item Definitely $!A$ or definitely not-$!A$.
		\item If definitely $!A$, then definitely definitely $!A$. (If it's
		definitely the case that $!A$, then it's definitely the case that
		it's definitely the case that $!A$.)
		\item If indefinitely definitely $!A$ then indefinitely $!A$. (If
		it's indefinitely the case that it's definitely the case that $!A$,
		then it's indefinitely the case that $!A$.)
	\end{enumerate}
\end{enumerate}

\end{prob}

\begin{prob}
Suppose we think of logical `necessity' in the following way: it is
logically necessary that $!A$ iff $!A$ is true no matter how
we interpret its !!{propositional variable}s. We capture the idea with a relational model
where $\Box !A$ is true at a world iff every interpretation of sentence
letters makes $!A$ true. We say that a !!{formula} $!A$ is ``logically necessary''
just if $\Box !A$ is true in that model. 

\begin{enumerate}
	\item We let each world $w$ be a (propositional-logic) !!{valuation}
	of the !!{propositional variable}s. We write $w(\Obj{\pvar{A}})=\True$ when the interpretation
	that consititutes world $w$ assigns $\True$ to $\Obj{\pvar{A}}$, $w(\Obj{\pvar{A}})=\False$
	if it assigns it false. Our set of worlds $W$ is the set of all !!{valuation}s.
	How do we define an accessibility relation $R$ and a modal interpretation
	$V$ to get the desired model $\tuple{W,R,V}$?
	\item Which modal logic does logical necessity obey on that picture? (Give
	the strongest of the standard modal logics \textbf{K, D}, \textbf{T},
	\textbf{B}, \textbf{S4}, \textbf{S5} that applies.) Briefly justify
	your answer.
	\item Are the following true in your model:
	\begin{enumerate}
		\item $\Box \Obj{\pvar{A}} \lif\Box\Box \Obj{\pvar{A}}$
		\item $\Diamond\lnot \Obj{\pvar{A}}$.
		\item $\Box\Diamond\lnot \Obj{\pvar{A}}$.
	\end{enumerate}
	\item Consider the uniform substitution rule:

	If $!A$ is a logical necessity then for any sentence letter $\Obj{\pvar{A}}$
	that appears in $!A$ and any formula $!B$, the result of substituting $!B$
	for $\Obj{\pvar{A}}$ in $!A$, namely $\Subst{!A}{!B}{\Obj{\pvar{A}}}$ is a logical necessity
	too.

	Explain why the rule \emph{fails} in the picture we just developped. 

\end{enumerate}


\end{document}