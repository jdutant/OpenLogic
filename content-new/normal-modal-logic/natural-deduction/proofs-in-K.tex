% Part: normal-modal-logic
% Chapter: natural-deduction 
% Section: proofs-in-K

\documentclass[../../../include/open-logic-section]{subfiles}

\begin{document}

\olfileid{nml}{nd}{prk}

\olsection{Examples of \usetoken{P}{derivation} in \Log{K}}

\begin{ex}
!!^{derivation} of $\Proves[\Log{K}] \Box(!A\lif!A)$.
\begin{prooftree}
\AxiomC{$\Discharge{!A}{1}$}
\DischargeRule{\Intro{\lif}}{1}
\UnaryInfC{$!A\lif!A$}
\RightLabel{$\Box$\Ax{K}}
\UnaryInfC{$\Box(!A\lif!A)$}
\end{prooftree}
\end{ex}

\begin{explain}
At the second step with have a !!{derivation} of $!A\lif !A$ without
undischarged assumptions. We can therefore apply rule $\Box$\Ax{K} 
as a special case without $\Box\ldots$ assumptions. This last step
would not be legit if some assumptions had not been discharged at
that stage.
\end{explain}

\begin{ex}
    !!^{derivation} of $\Box(!A\land!B)\Proves[\Log{K}] \Box!A\land\Box!B$.
    \begin{prooftree}
    \AxiomC{$\Box!A\land !B$}
        \AxiomC{$\Discharge{!A\land!B}{1}$}
        \RightLabel{\Elim{\land}}
        \UnaryInfC{$!A$}
    \DischargeRule{$\Box$\Ax{K}}{1}
    \BinaryInfC{$\Box!A$}
            \AxiomC{$\Box!A\land !B$}
                \AxiomC{$\Discharge{!A\land!B}{2}$}
                \RightLabel{\Elim{\land}}
                \UnaryInfC{$!B$}
            \DischargeRule{$\Box$\Ax{K}}{2}
            \BinaryInfC{$\Box!B$}
        \RightLabel{\Intro{\land}}
        \BinaryInfC{$\Box!A\land\Box!B$}
    \end{prooftree}

\end{ex}

\begin{prob}
Give !!a{derivation} of $\Box!A\land\Box!B\Proves[\Log{K}]
\Box(!A\land!B)$.
\end{prob}

\begin{prob}
    Give a !!{derivation} of $\Proves[\Log{K}]
    \Diamond(!A\land!B)\lif(\Diamond!A\land\Diamond!B)$. 
\end{prob}


\begin{ex}
!!^{derivation} of $\Proves[\Log{K}] \Box(!A\lif!B)\lif(\Box!A\lif\Box!B)$.

\begin{prooftree}
\AxiomC{$\Discharge{\Box(!A\lif!B)}{3}$}
    \AxiomC{$\Discharge{\Box!A}{2}$}
        \AxiomC{$\Discharge{!A\lif!B}{1}$}
        \AxiomC{$\Discharge{!A}{1}$}
        \RightLabel{\Elim{\lif}}
        \BinaryInfC{$!B$}
\DischargeRule{$\Box$\Ax{K}}{1}
\TrinaryInfC{$\Box!B$}
\DischargeRule{\Intro{\lif}}{2}
\UnaryInfC{$\Box!A\lif\Box!B$}
\DischargeRule{\Intro{\lif}}{3}
\UnaryInfC{$\Box(!A\lif!B)\lif(\Box!A\lif\Box!B)$}
\end{prooftree}
\end{ex}


% TODO: apply \iftag{prvDiamond}
\begin{prob}
    Give a !!{derivation} of $\Proves[\Log{K}]
    \Box(!A\lif!B)\lif(\Diamond!A\lif\Diamond!B)$.

\end{prob}

\begin{ex}
    The \emph{Duality} rules below are !!{derivable}:

    \bigskip
    \begin{tabular}{cc}
        \AxiomC{}\DeduceC{$\Box\lnot!A$}
        \RightLabel{\Intro{\Diamond}\Ax{K}$^*$}
        \UnaryInfC{$\lnot\Diamond!A$}
        \DisplayProof
    &
        \AxiomC{}\DeduceC{$\Diamond\lnot!A$}
        \RightLabel{\Elim{\Diamond}\Ax{K}$^*$}
        \UnaryInfC{$\lnot\Box!A$}
        \DisplayProof
    \end{tabular}
    \bigskip

   For \Intro{\Diamond}\Ax{K}$^*$:
    \begin{prooftree}
        \AxiomC{$\Discharge{\Diamond!A}{2}$}
        \AxiomC{}\DeduceC{$\Box\lnot!A$}
            \AxiomC{$\Discharge{!A}{1}$}
            \AxiomC{$\Discharge{\lnot!A}{1}$}
            \RightLabel{\Elim{\lnot}}
            \BinaryInfC{$\lfalse$}
        \DischargeRule{\Intro{\Diamond}\Ax{K}}{1}
        \TrinaryInfC{$\Diamond\lfalse$}
        \RightLabel{\Elim{\Diamond}\Ax{K}}
        \UnaryInfC{$\lfalse$}
        \DischargeRule{\FalseCl}{2}
        \UnaryInfC{$\lnot\Diamond!A$}        
    \end{prooftree}

    For \Elim{\Diamond}\Ax{K}$^*$:
    \begin{prooftree}
        \AxiomC{}\DeduceC{$\Diamond\lnot!A$}
        \AxiomC{$\Discharge{\Box!A}{2}$}
            \AxiomC{$\Discharge{\lnot!A}{1}$}
            \AxiomC{$\Discharge{!A}{1}$}
            \RightLabel{\Elim{\lnot}}
            \BinaryInfC{$\lfalse$}
        \DischargeRule{\Intro{\Diamond}\Ax{K}}{1}
        \TrinaryInfC{$\Diamond\lfalse$}
        \RightLabel{\Elim{\Diamond}\Ax{K}}
        \UnaryInfC{$\lfalse$}
        \DischargeRule{\FalseCl}{2}
        \UnaryInfC{$\lnot\Box!A$}        
    \end{prooftree}

\end{ex}

\begin{note}
Conversely, we can derive rules \Intro{\Diamond}\Ax{K} and
\Elim{\Diamond}\Ax{K} from rules \Intro{\Diamond}\Ax{K}$^*$ and
\Elim{\Diamond}\Ax{K}$^*$ together with rule $\Box$\Ax{K}. Thus an
equivalent system results from $\Box$\Ax{K},
\Intro{\Diamond}\Ax{K}$^*$ and \Elim{\Diamond}\Ax{K}$^*$ as primitives
instead.

We start with the simpler case, deriving \Elim{\Diamond}\Ax{K} from
$\Box$\Ax{K} and \Intro{\Diamond}\Ax{K}$^*$:
\begin{prooftree}
    \AxiomC{$\Discharge{\lfalse}{1}$}
    \DischargeRule{\FalseCl}{1}
    \UnaryInfC{$\lnot\lfalse$}
    \RightLabel{$\Box$\Ax{K}}
    \UnaryInfC{$\Box\lnot\lfalse$}
    \RightLabel{\Intro{\Diamond}\Ax{K}$^*$}
    \UnaryInfC{$\lnot\Diamond\lfalse$}
        \AxiomC{$\Diamond\lfalse$}
        \RightLabel{\Elim{\lnot}}
    \BinaryInfC{$\lfalse$}
\end{prooftree}

Deriving \Intro{\Diamond}\Ax{K} from $\Box$\Ax{K},
\Elim{\Diamond}\Ax{K}$^*$ and \Intro{\Diamond}\Ax{K}$^*$:
\begin{prooftree}
    \AxiomC{}\DeduceC{$\Diamond!A$}
        \AxiomC{$\Discharge{\lnot\Diamond!C}{3}$}
        \RightLabel{\Elim{\Diamond}\Ax{K}$^*$}
        \UnaryInfC{$\Box\lnot!C$}
        \AxiomC{}\DeduceC{$\Box!B$}
            \AxiomC{$\Discharge{!A}{1},\Discharge{!B}{2}$ \emph{at
            most}}\DeduceC{$!C$} 
            \AxiomC{$\Discharge{\lnot!C}{2}$}
            \DischargeRule{\Intro{\lnot}}{1}
            \BinaryInfC{$\lnot!A$}
        \DischargeRule{$\Box$\Ax{K}}{2}
        \TrinaryInfC{$\Box\lnot!A$}
        \RightLabel{\Intro{\Diamond}\Ax{K}$^*$}
        \UnaryInfC{$\lnot\Diamond$\Ax{K}}
    \DischargeRule{\Elim{\lnot}}{3}
    \BinaryInfC{$\Diamond!C$}
\end{prooftree}

Deriving \Elim{\Diamond}\Ax{K} from $\Box$\Ax{K} and
\Intro{\Diamond}\Ax{K}$^*$:
\begin{prooftree}
    \AxiomC{$\Discharge{\lfalse}{1}$}
    \DischargeRule{\FalseCl}{1}
    \UnaryInfC{$\lnot\lfalse$}
    \RightLabel{$\Box$\Ax{K}}
    \UnaryInfC{$\Box\lnot\lfalse$}
    \RightLabel{\Intro{\Diamond}\Ax{K}$^*$}
    \UnaryInfC{$\lnot\Diamond\lfalse$}
        \AxiomC{$\Diamond\lfalse$}
        \RightLabel{\Elim{\lnot}}
    \BinaryInfC{$\lfalse$}
\end{prooftree}

\begin{explain}
    We assume that we are given a !!{derivation} of $!C$ from $!A$, $!B$
    at most, a !!{derivation} of $\Diamond!A$ and a !!{derivation} of
    $\Box!B$. We first turn the !!{derivation} of $!C$ from $!A$, $!B$ at
    most into a !!{derivation} of $\lnot!A$ from $!B$, $\lnot!C$ alone.
    Combining this with the !!{derivation} of $\Box!B$ (given) and a
    !!{derivation} of $\Box\lnot!C$ (from $\lnot\Diamond!C$ which we
    assume for \emph{reductio} later on), we get $\Box\lnot!A$. Together
    with the !!{derivation} of $\Diamond !A$ (given) this leads to a
    contradiction, which allow us to reduce to absurdity our assumption of
    $\lnot\Diamond!C$ and hence conclude $\Diamond !C$. 
    \end{explain}

\end{note}

\end{document}