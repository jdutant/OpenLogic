% Part: normal-modal-logic
% Chapter: natural-deduction 
% Section: PL-rules

\documentclass[../../../include/open-logic-section]{subfiles}

\begin{document}

\olfileid{nml}{nd}{plr}

\olsection{Simplifying propositional logic rules}

\subsection{Rule of propositional consequence \Log{PL}}

We introduce a rule to omit purely propositional logic steps. 

\begin{defish}
When $!B$ can be derived from $!A_1,\ldots,!A_n$ ($n\geq 0$) using
propositional logic rules alone, we may simply write: 
\begin{prooftree}
    \AxiomC{}\DeduceC{$!A_1$}
    \AxiomC{}\DeduceC{$\ldots$}
    \AxiomC{}\DeduceC{$!A_n$}
    \RightLabel{\Log{PL}}
    \TrinaryInfC{$!B$}
\end{prooftree}
\end{defish}

\begin{ex}
The following !!{derivation} of $!A\lor!B,\lnot!A\Proves{\Log{K}}!B$:
\begin{prooftree}
\AxiomC{$!A\lor!B$}
    \AxiomC{$\Discharge{!A}{1}$}
    \AxiomC{$\lnot!A$}
    \RightLabel{\Elim{\lnot}}
    \BinaryInfC{$!B$}
            \AxiomC{$!B$}
    \DischargeRule{\Elim{\lor}}{1}
    \TrinaryInfC{$!B$}
\end{prooftree}
may be abbreviated:
\begin{prooftree}
    \AxiomC{$!A\lor!B$}
        \AxiomC{$\lnot!A$}
        \RightLabel{\Log{PL}}
    \BinaryInfC{$!B$}
\end{prooftree}
\end{ex}

\begin{ex}
The !!{derivation} of excluded middle, $!A\lor\lnot!A$:
\begin{prooftree}
\AxiomC{$\Discharge{!A}{1}$}
\RightLabel{\Intro{\lor}}
\UnaryInfC{$!A\lor!\lnot!A$}
    \AxiomC{$\Discharge{\lnot(!A\lor!\lnot!A)}{2}$}
\DischargeRule{\Intro{\lnot}}{1}
\BinaryInfC{$\lnot!A$}
\RightLabel{\Intro{\lor}}
\UnaryInfC{$!A\lor!\lnot!A$}
    \AxiomC{$\Discharge{\lnot(!A\lor!\lnot!A)}{2}$}
\DischargeRule{\Elim{\lnot}}{2}
\BinaryInfC{$!A\lor\lnot!A$}
\end{prooftree}
May be abbreviated:
\begin{prooftree}
\AxiomC{~}
\RightLabel{\Log{PL}}
\UnaryInfC{$!A\lor\lnot!A$}
\end{prooftree}
\end{ex}

% \begin{note}
% The \Log{PL} rule does \emph{not} permit substituting propositional 
% consequences or equivalents for sub-!!{formula}s \emph{embedded within 
% modal operators}. For an illustration, compare the two !!{derivation}s:

% \bigskip
% \begin{tabular}{cc}
% \end{tabular}
% \bigskip
% \end{note}

\end{document}