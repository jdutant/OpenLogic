% Part: first-order-logic
% Chapter: syntax-and-semantics
% Section: negative-positive-neutral

\documentclass[../../../include/open-logic-section]{subfiles}

\begin{document}

\olfileid{frl}{syn}{pnn}

\olsection{Positive, Negative and Neutral Free Logic}

There are several ways of removing the existence assumptions of
first-order logic. These give rise to different systems of free logic:
negative, positive and neutral. The dividing issue is how to treat 
!!{formula}s with empty !!{constant}s.

\begin{explain}
In free logic, not every !!{constant} needs to refer to something that exists.
A !!{constant} without referent is said to be `empty'. 
\end{explain}

Let $a$ be an empty !!{constant} and $F$ some predicate:

\begin{enumerate}
  \item According to \emph{negative} free logic, $\Atom{F}{a}$ is false.
  \item According to \emph{positive} free logic, $\Atom{F}{a}$ can be either
  true or false.
  \item According to \emph{neutral} free logic, $\Atom{F}{a}$ is neither
  true nor false.
\end{enumerate}

Positive and Negative Free Logic are compatible with classical
propositional logic. Neutral Free Logic requires a non-classical
logic suitable for non-bivalent semantics.

\iftag{neuFrl}{}
{We do not study neutral free logic in this class.}

\end{document}
