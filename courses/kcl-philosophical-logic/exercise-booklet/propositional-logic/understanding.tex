% Part: exercise-booklet
% Chapter: propositional-logic
% Section: understanding

\documentclass[../../../include/open-logic-section]{subfiles}

\begin{document}

\olfileid{exb}{pl}{und}

\olsection{Check Your Understanding}

\begin{prob}
    Write down truth tables for $\bot$, $\lor$, $\land$, $\lif$.

    \begin{ans}See textbook.\end{ans}
\end{prob}

\begin{prob}
    What does it mean to say that !!a{formula} is a \emph{tautology}?
    That it is a \emph{contradiction}? That it is \emph{logically contingent}?

    Give an example of each.
    \begin{ans}See textbook for the definitions. Examples: 
        $\Obj{\pvar{A}}\lif\Obj{\pvar{A}}$ for a tautology, 
        $\lnot(\Obj{\pvar{A}}\lif\Obj{\pvar{A}})$ for a contradiction, 
        $\Obj{\pvar{A}}$ for a logically contingent !!{formula}. 
    \end{ans}
\end{prob}

\begin{prob}
    Are the following claims correct or not? Explain why or why not.
    \begin{itemize}
    \item if $!A$ is logically contingent, $\lnot !A$ is logically contingent.
    \item if $!A$ is a tautology, $\lnot !A$ is a contradiction.
    \end{itemize}

    \begin{ans}
    Both are correct. $!A$ is logically contingent iff it is true in some
    !!{valuation} and false in some other !!{valuation}. If so, $\lnot!A$ is
    false in the former and true in the latter, hence it is logically
    contingent.

    $!A$ is a tautology iff it is true in all !!{valuation}s. If so, 
    $\lnot !A$ is false in all !!{valuation}s, hence it is a contradiction.
    \end{ans}
\end{prob}

\begin{prob}
    What does it mean to say that:
    \begin{itemize}
    \item An argument is propositionally valid?
    \item A set of !!{formula}s is satisfiable?
    \end{itemize}

    Give examples of each of the following, in the language of
    propositional logic:
    \begin{itemize}
    \item valid argument
    \item invalid argument
    \item satisfiable set of !!{formula}s (containing 3 or more
    formulas)
    \item unsatisfiable set of !!{formula}s (containing 3 or more
    formulas)
    \end{itemize}

    \begin{ans}Definitions: see textbook.

    Examples (using $/$ to separate premises and conclusions in argument):
    \begin{itemize}
        \item Valid argument: 
        $$\Obj{\pvar{A}} ~/~ \Obj{\pvar{A}} \lor \Obj{\pvar{B}}$$

        \item Invalid argument:
        $$\Obj{\pvar{A}} ~/~ \Obj{\pvar{A}} \land \Obj{\pvar{B}}$$

        \item Satisfiable set of !!{formula}s:
        $$\Obj{\pvar{A}}, \Obj{\pvar{B}}, 
        \Obj{\pvar{A}}\land\Obj{\pvar{B}}$$

        \item Unsatisfiable set of !!{formula}s:
        $$\Obj{\pvar{A}}, \Obj{\pvar{B}}, 
        \lnot(\Obj{\pvar{A}}\land \Obj{\pvar{B}})$$
    
    \end{itemize}
    \end{ans}
\end{prob}

\begin{prob}
    \citep[1.1 item 5]{MacFarlane-2020-PhilosophicalLogicContemporary}
    What does it mean to say that two !!{formula}s are \emph{logically equivalent}?
    Give an (interesting) example of two logically equivalent !!{formula}s of
    propositional logic.    

    \begin{ans}Definition: see textbook.

    Example: $\lnot\Obj{\pvar{A}}$ and 
    $\Obj{\pvar{A}}\lif\lnot\Obj{\pvar{A}}$ are logically equivalent.
    \end{ans}
\end{prob}

% \begin{prob}[Find !!{valuation}s to verify or falsify !!a{formula}]
    
% \begin{ans}
% \end{ans}
% \end{prob}
    
\end{document}

