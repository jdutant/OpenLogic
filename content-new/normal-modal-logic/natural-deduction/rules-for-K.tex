% Part: normal-modal-logic
% Chapter: natural-deduction 
% Section: rules-for-K

\documentclass[../../../include/open-logic-section]{subfiles}

\begin{document}

\olfileid{nml}{nd}{rul}

\olsection{Rules for \Log{K}}

For the modal operator $\Box$, we add the \emph{Normality} rule 
$\Box$\Ax{K}:
\begin{defish}
\AxiomC{}\DeduceC{$\Box!A$}
\AxiomC{}\DeduceC{$\Box!B$}
\AxiomC{$\Discharge{!A}{n},\Discharge{!B}{n}$ \emph{at
most}}
\DeduceC{$!C$} 
\DischargeRule{$\Box$\Ax{K}}{n}
\TrinaryInfC{$\Box!C$} 
\DisplayProof
\end{defish}

The rule plays both the role of an introduction rule (for $\Box!C$)
and of an elimination rule (for $\Box!A$ and $\Box!B$). We thus call
it $\Box$\Ax{K} rather than \Intro{\Box}\Ax{K} or \Elim{\Box}\Ax{K}. 

At the step when $\Box$\Ax{K} is applied, the sub-!!{derivation} on 
the right must have at most assumptions $!A,!B$ left undischarged. 
It may have other assumptions, but they must be discharged by the 
time we reach this step.

The rule can be applied with just one premise $\Box!A$ 
or even no premise:

\bigskip
\begin{tabular}{cc}
    \AxiomC{}\DeduceC{$\Box!A$}
    \AxiomC{$\Discharge{!A}{n}$ \emph{at
    most}}
    \DeduceC{$!C$} 
    \DischargeRule{$\Box$\Ax{K}}{n}
    \BinaryInfC{$\Box!C$} 
    \DisplayProof
&
    \AxiomC{\emph{no undischarged assumption}}
    \DeduceC{$!C$}
    \RightLabel{$\Box$\Ax{K}}
    \UnaryInfC{$\Box!C$} 
    \DisplayProof
\end{tabular}
\bigskip

\iftag{prvDiamond}{ For the modal operator $\Diamond$, we add the
following \emph{Normality} rules \Intro{\Diamond}\Ax{K} and
\Elim{\Diamond}\Ax{K}:
\begin{defish}
\begin{tabular}{cc}
    \AxiomC{}\DeduceC{$\Diamond!A$}
    \AxiomC{}\DeduceC{$\Box!B$}
    \AxiomC{$\Discharge{!A}{n}$\emph{at least},$\Discharge{!A}{n}\Discharge{!B}{n}$\emph{at most}}\DeduceC{$!C$}
    \DischargeRule{\Intro{\Diamond}\Ax{K}}{n}
    \TrinaryInfC{$\Diamond!C$}
    \DisplayProof
&   
    \AxiomC{}\DeduceC{$\Diamond\lfalse$}
    \RightLabel{\Elim{\Diamond}\Ax{K}}
    \UnaryInfC{$\lfalse$}
    \DisplayProof
\end{tabular}
\end{defish}
}{}

As with $\Box$\Ax{K}, the sub-!!{derivation} on the right should at 
most have $!A$ and $!B$ as undischarged assumptions. The rule can be 
applied without $\Box!B$ premise, but \emph{it cannot be applied without
the $\Diamond!A$ premise}:

\bigskip
\begin{tabular}{cc}
    \AxiomC{}\DeduceC{$\Diamond!A$}
    \AxiomC{$\Discharge{!A}{n}$\emph{at least}}\DeduceC{$!C$}
    \DischargeRule{\Intro{\Diamond}\Ax{K}}{n}
    \BinaryInfC{$\Diamond!C$}
    \DisplayProof
&
    \AxiomC{\emph{no undischarged assumption}}\DeduceC{$!C$}
    \RightLabel{{\color{red}Incorrect}}
    \UnaryInfC{$\Diamond!C$}
    \DisplayProof
\end{tabular}
\bigskip

Note, however, that this restriction is lifted in systems \Log{D}
and stronger.

\begin{explain}
The $\Box$\Ax{K} rule is called \emph{Normality} because it
characterizes how necessity behaves in so-called \emph{normal modal
logics}. The core idea of normal modal logics is that necessity is
\emph{closed under multi-premise logical consequence}, namely, 
that if some premises logically entail a conclusion, then the 
necessity of those premises entail the necessity of that conclusion:

\begin{quote}
\emph{Normality.} If $\Gamma\Entails!A$ then
$\Box\Gamma\Entails\Box\!A$. (Where $\Box\Gamma$ is the result of
prefixing each !!{formula} in $\Gamma$ with a $\Box$ operator, i.e.
the set $\Setabs{\Box!B}{!B\in\Gamma}$.)
\end{quote} 

Our $\Box$\Ax{K} rule captures that idea. The sub-!!{derivation} on
the right establishes that $!A,!B\Entails!C$. It is crucial that it
has \emph{no undischarged assumptions other than} $!A,!B$. Otherwise
it would not establish that $!C$ is a logical consequence of $!A,!B$
\emph{alone}. Once $!A,!B\Entails!C$ is established, according to
Normality $\Box!C$ is a logical consequence of $\Box!A$, $\Box!B$.
Therefore we can discharge the assumptions $!A,!B$ on the right, and
turn the two sub-!!{derivation}s of $\Box!A$, $\Box!B$ on the left
into a !!{derivation} of $\Box!C$. Note that the sub-!!{derivation}s
of $\Box!A$, $\Box!B$ may rest on undischarged assumptions; we then
have a !!{derivation} of $\Box!C$ from these assumptions.

The $\Box$\Ax{K} rule directly captures the idea that logical
consequences of \emph{up to two} necessities are necessary. But
Normality states that logical consequences of \emph{any number} of
necessities are necessary. Is our rule enough? Yes, because repeated
applications of $\Box$\Ax{K} ensure that logical consequences of
\emph{any finite number} of necessities are necessary. Could it happen
that $!A$ is the logical consequence of an \emph{infinite} number of
necessities, without being a logical consequence of any finite subset
of those? If that was so, logical consequence would not be compact.
When logical consequence is not compact, a complete !!{derivation}
system cannot be given; our rule would still be the best we can
achieve. As it happens, logical consequence in \Ax{K} models is
compact, and the rule is sufficient for completeness.

\iftag{prvDiamond}{ The \Elim{\Diamond}\Ax{K} idea captures the idea that contradictions
are not possible.

The \Intro{\Diamond}\Ax{K} rule captures the idea
that \emph{possibility is closed under single-premise necessary
consequence}, namely that if $!C$ is a necessary consequence of $!A$,
then $\Diamond!C$ is a logical consequence of $\Diamond!A$. To clarify this idea,
start with the related idea that possibility is closed under single-premise
\emph{logical} consequence: if $!C$ is a \emph{logical} consequence of 
$!A$, then $\Diamond!C$ is a logical consequence of $\Diamond!A$. 
This is a special case of \Intro{\Diamond}\Ax{K} with $\Box!B$ missing.
Then, define \emph{necessary} consequence as follows:
\begin{quote}
\emph{Necessary consequence}. $!C$ is a necessary consequence of $!A$
iff $!A\lif!C$ is necessary. (Equivalently, iff $!C$ is a logical
consequence of $!A$ \emph{together with} what is necessary.)
\end{quote}
This idea would be captured by the following rule:
\begin{prooftree}
    \AxiomC{}\DeduceC{$\Diamond!A$}
    \AxiomC{}\DeduceC{$\Box(!A\lif!B)$}
    \BinaryInfC{$\Diamond!C$}
\end{prooftree}
Which is equivalent to \Intro{\Diamond}\Ax{K} given the $\Box$\Ax{K}
rule. We prefer \Intro{\Diamond}\Ax{K} because, unlike the rule above,
it does not involve a specific propositional operator ($\lif$). 

Why `single premise' consequence? While we accept the idea that 
logical consequences of \emph{multiple} necessities are necessary, we
reject the idea that logical consequences of \emph{multiple} possibilities
are possible. For instance, it $!A$ is contingent it is both possible
that $!A$ and possible that $\lnot!A$. While $!A$ and $\lnot!A$ \emph{jointly}
logically entail $!A\land\lnot!A$, we do not want to say that $\Diamond!A$
and $\Diamond\lnot!A$ jointly entail $\Diamond(!A\land\lnot!A)$. For 
the same reason, we do not want to say that necessary consequences of
multiple possibilities are possibilities. We only accept the principle
that necessary consequences of \emph{one} possibility are possible.

Another way to motivate the $\Diamond$ rules is \emph{Duality}. 
Duality is the idea that $\Diamond$ is
the \emph{dual} of $\Box$: the negation of a
necessity is is the possibility of its negation ($\lnot\Box$ is
$\Diamond\lnot$) and the negation of a possibility is the necessity of
its negation ($\lnot\Diamond$ is $\Box\lnot$). As shown below, our
$\Diamond$ rules are equivalent to Duality.}{}
\end{explain}

\end{document}
