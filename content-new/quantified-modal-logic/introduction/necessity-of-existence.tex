% Part: quantified-modal-logic
% Chapter: introduction
% Section: necessity-of-existence

\documentclass[../../../include/open-logic-section]{subfiles}

\begin{document}

\olfileid{qml}{int}{noe}

\olsection{Necessity of Existence}

Combining standard first-order logic with identity and modal logic 
allows us to prove that (a) necessarily, something exists, 
and (b) everything exists necessarily---that is, everything is such
that it could not have not existed. The second result is highly 
contentious, of course.

For the first:

\begin{prooftree}
    \AxiomC{}
    \RightLabel{\Intro{\eq}}
    \UnaryInfC{$\eq[a][a]$}
    \RightLabel{\Intro{\lexists}}
    \UnaryInfC{$\lexists[x][\eq[x][x]]$}
    \RightLabel{$\Box$\Ax{K}}
    \UnaryInfC{$\Box\lexists[x][\eq[x][x]]$}
\end{prooftree}

For the second:

\begin{prooftree}
    \AxiomC{}
    \RightLabel{\Intro{\eq}}
    \UnaryInfC{$\eq[a][a]$}
    \RightLabel{\Intro{\lexists}}
    \UnaryInfC{$\lexists[x][\eq[x][x]]$}
    \RightLabel{$\Box$\Ax{K}}
    \UnaryInfC{$\Box\lexists[x][\eq[x][x]]$}
\end{prooftree}

We can show the same point using possible world semantics.

In \emph{fixed domain} semantics for Quantified Modal Logic, a model
has a set of world and a unique domain of object. It's easy to see
that the necessity of existence holds on such models. See Sider for 
a detailed presentation.

To avoid the consequence, one must give up at least one of
Necessitation, the Reflexivity of Identity, or the rules of
Existential Instantiation. The dominant approach is a to change the 
quantifier rule and adopt a \emph{free logic}.

In terms of models, we must adopt *variable*
domains: the domain of individuals may change from world to world. 
This means that a constant like $\Obj a$ may denote an object that is not
in the domain of some worlds. We must then evaluate formulas like 
$\Obj{Fa}$ or $\Obj(a=b)$ at worlds \emph{where $a$ doesn't denote
anything in the domain}. Classical semantics for predicate logic 
doesn't have a way to do that. Free logics have been designed to 
accomodate denotation-less constants. Hence a Quantified Modal Logic
that avoids the necessity of existence must combine modal logic with 
\emph{free} first order logic logic.

\end{document}
