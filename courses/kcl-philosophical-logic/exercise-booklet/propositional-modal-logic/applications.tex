% Part: exercise-booklet
% Chapter: propositional-modal-logic
% Section: applications

\documentclass[../../../../include/open-logic-section]{subfiles}

\begin{document}

\olfileid{exb}{pml}{app}

\olsection{Applications of Modal Logic}

\begin{prob}[From principles to modal system]
Suppose we write `$\Box !A$' for `Sam believes that $ !A$' and
we agree that the following principles are correct / incorrect:

\begin{itemize}
\item Sam believes the logical consequences of what she believes. That is,
if Sam believes all of $ !A_{1}$, $ !A_{2}$, $\ldots$, $ !A_{n}$
and $ !A_{1}$, $\ldots$, $ !A_{n}$ entail $!B$, then Sam believes
$!B$. CORRECT
\item If Sam believes $!A$, she believes that she believes $!A$. CORRECT
\item If Sam believes $!A$, then $!A$. INCORRECT 
\item If Sam believes $!A$, she does not believe not-$!A$. CORRECT
\item If Sam does not believe $!A$, she believes that she does not believe
$ !A$. CORRECT
\end{itemize}
Which logic does (Sam) belief obey, on our view?

\emph{Note}. It is not one of the standard systems \Log{K}, \Log{D},
\Log{T}, \Log{B}, \Log{S4}, \Log{S5}. Rather,
you should name the logic by listing the principles it obeys,
starting with \Ax{K}, e.g. \Log{KTB4}.

\begin{ans}
	The logic \Log{KD45}. 
	\begin{itemize}
		\item
	It is correct that Sam believes the logical consequences 
	of what she believes. We need a normal modal logic (any logic 
	at least as strong as\Log{K}): if $!A_1,\ldots,!A_n\Entails!B$
	then $\Box!A_1,\ldots,\Box!A_n\Entails\Box!B$. 
		\item It is correct that if Sam believes $!A$, 
		she believes that she believes $!A$. We need schema \Ax{4}, 
		$\Box!A\lif\Box\Box!A$.
		\item It is not correct that if Sam believes $!A$, $!A$ is 
		true. We do not want schema \Ax{T}, $\Box!A\lif!A$.
		\item It is correct that if Sam believes $!A$, she does not 
		believe not-$!A$. We need schema \Ax{D}, $\Box!A\lif\Diamond!A$, 
		which is equivalent to $\Box!A\lif\lnot\Box\lnot!A$ (by Duality).
		\item If Sam does not believe $!A$, she believes that she does
		not believe $!A$. We need the schema
		$\lnot\Box!A\lif\Box\lnot\Box!A$. By Duality this is
		equivalent to $\Diamond\lnot!A\lif\Box\Diamond\lnot!A$. But if
		this schema holds for any $!A$, it is equivalent to schema
		\Ax{5}, $\Diamond!B\lif\Box\Diamond!B$ (just put $\lnot!B$ for
		$!A$, you get
		$\Diamond\lnot\lnot!B\lif\Box\Diamond\lnot\lnot!B$, and recall
		that if the modal logic is normal $\Diamond\lnot\lnot!B$ is
		equivalent to $\Diamond!B$).
	\end{itemize}
	Therefore we need a normal modal logic (extension of \Log{K}) with
	schemas \Ax{D}, \Ax{4}, \Ax{5}, but not \Ax{T}: \Log{KD45}. 
\end{ans}
\end{prob}

\begin{prob}[From modal system to principles]
Suppose we write `$\Box !A$' for `Sam knows that $ !A$' and we
think $\Box$, under that reading, has the logic \Log{S4}. What
does our view say about (Sam's) knowledge? 

\begin{ans}
	\Log{S4} is a normal modal logic with \Ax{D}, \Ax{T}, \Ax{4}.
	Therefore if we take it as a logic for Sam's knowledge, it says
	that:
	\begin{itemize}
	\item Sam knows the logical consequences of what she knows.
	\item If Sam knows $!A$, she doesn't know that not-$!A$. (\Ax{D};
	another way to put it: for any $!A$, it is not the case that 
	Sam knows $!A$ and Sam also knows $\lnot!A$.)
	\item What Sam knows is true. If Sam knows $!A$, then $!A$. (\Ax{T})
	\item If Sam knows something, she knows that she knows it. If 
	Sam knows $!A$, she knows that she knows $!A$. (\Ax{4}).
	\end{itemize}
	We may also note that \Ax{B} and \Ax{5} are not valid in that logic.
	Therefore saying that the logic of Sam's knowledge isn't stronger
	that \Log{S4} means that:
	\begin{itemize}
	\item It may happen, for some $!A$, that $!A$ is true but Sam 
	does not know that she does not know that $\lnot!A$. (Failure of \Ax{B},
	$!A\land\lnot\Box\lnot\Box\lnot!A$, equiv. to $!A\land\lnot\Box\Diamond!A$)
	\item It may happen, for some $!A$, that Sam does not know $!A$
	but she also does not know that she does not know it. (Failure of
	\Ax{5}, $\lnot\Box!A\land\lnot\Box\lnot!A$, equiv. to
	$\Diamond\lnot!A\land\lnot\Box\Diamond\lnot!A$).
	\end{itemize}
\end{ans}
\end{prob}

\begin{prob}
Suppose we think of deontic notions in the following way. Among all
possible worlds there is a set of ideal worlds where everything happens
as it should. At any given world $w$, it is obligatory that $!A$
iff $!A$ holds at all the ideal worlds; it is permitted that $!A$
iff it is not obligatory that not-$!A$. 

\begin{enumerate}
	\item Suppose we are given $W$ the set of possible worlds, $I_{W}$ the
	(non-empty) subset of $W$ that are the ideal worlds, and $V$ a valuation
	of the sentence letters in each world. How do you define the accessibility
	relation $R$ to get a model $\tuple{W,R,V}$ that captures
	the picture above? 
	\item What are the properties of $R$ on that picture: it is serial, reflexive,
	symmetric, transitive? Briefly justify each answer. 
	\item Which modal logic do deontic notions obey on that picture? (Give the
	strongest of the standard modal logics \Log{K}, \Log{D},
	\Log{T}, \Log{B}, \Log{S4}, \Log{S5} that applies.) 
	\item For each of the principles or claims below, say whether is correct in
	your model. Briefly justify your answers.
		\begin{enumerate}
			\item If it is true that $!A$, it is permitted that $!A$.
			\item If it is permitted that $!A$, it is obligatory that it is permitted
			that $!A$.
			\item It is obligatory that: it is raining or it is not raining.
			\item If it is obligatory that Boris sends the letter, then it is obligatory
			that: Boris sends the letter or Boris burns it. 
			\item If it is obligatory that Ali replaces the lamp he has broken, then
			it is obligatory that Ali has broken the lamp.
		\end{enumerate}
\end{enumerate}
\emph{Note}. Treat `if \ldots then' as a material conditional. A
principle or claim is true in your model iff it (or its traduction
in a formula) is true at every world of the model. For the last three
questions, suppose that your valuation $V$ captures the meaning of
the sentences in question; for instance, if $\Obj{\pvar{A}}$ is the proposition
that \emph{Ali replaces the lamp he has broken and $\Obj{\pvar{B}}$ is the proposition
that \emph{Ali has broken the lamp}, every world where $V$ counts
$\Obj{\pvar{A}}$ as true is a world where $\Obj{\pvar{B}}$ is true.}

\begin{ans}
	\begin{enumerate}
	\item Every world has access to all and only the ideal worlds: 
	$R=\Setabs{w,w'}{w\in W,w'\in I_W}$. 
	\item $R$ is serial, transitive, euclidean. If there are non-ideal worlds, 
	it is not reflexive nor symmetric.
	\begin{itemize}
		\item Serial: the set of ideal worlds $I_W$ is not empty, and every
		has access to any world in it. So every world has access to some world.
		\item Transitive. Suppose $Rww'$ and $Rw'w''$. Since $Rw'w''$ 
		it must be that $Rw''$ is an ideal world. Hence $Rww''$. Therefore
		whenever $Rww'$ and $Rw'w''$ we have $Rww''$: the model is transitive.
		\item Euclidean. Suppose $w$ has access to worlds $w',w''$: 
		$Rww'$ and $Rww''$. Then $w'$, $w''$ are both ideal worlds. 
		Then $w',w''$ have access to each other: $Rw'w''$ (and $Rw''w'$). 
		Therefore the relation $R$ is euclidean.
		\item Not reflexive (if $I_W\neq W$). If some world $w$ is not 
		in $I_W$, we do not have $Rww$ so the model is not reflexive.
		\item Not symmetric (if $I_W\neq W$). Suppose some world $w$ is 
		not ideal. Since $I_W$ is not empty, $w$ has access to some world 
		$w'$ in $I_W$; but since $w$ is not in $I_W$, $w'$ does not have 
		access to $w$. Since we have $Rww'$ but not $Rww'$ for some $w,w'$,
		the relation is not symmetric.
	\end{itemize}
	\emph{Comment}: you would not get the full points if you said that 
	$R$ was not reflexive nor symmetric without noting that this is 
	only so if not all worlds are ideal.

	\item The only properties that are guaranteed to hold are seriality,
	transitivity and euclideanity. We've seen in the previous answer 
	that some models compatible with the stated pictures have failures
	of reflexivity or symmetry, therefore the logic should \emph{not}
	include those. 

	Therefore the desired logic is \Log{KD45} (corresponding to 
	serial, transitive, euclidean frames). It is not, in fact, 
	one of the standard logics  
	\Log{K}, \Log{D}, \Log{T}, \Log{B}, \Log{S4}, \Log{S5}. Each of 
	\Log{T}, \Log{B}, \Log{S4}, \Log{S5} includes \Ax{T}, which fails
	in our picture (since the frame need not be reflexive). So the 
	strongest of the \emph{standard} modal logics compatible with 
	that picture is \Log{D}. 

	\item Principles that are correct or not in the picture (\Log{KD45}).
	\begin{enumerate}
	\item If it is true that $!A$, it is permitted that $!A$: $!A\lif \Diamond !A$.
	Incorrect. Consider a model with a non-ideal world where $\Obj{\pvar{A}}$ true 
	at the non ideal world but false at all ideal worlds. (You may also note 
	that $!A\lif \Diamond !A$ is characteristic of reflexive frames and 
	our frames are not restricted to reflexive ones).
	
	\item If it is permitted that $!A$, it is obligatory that it is permitted
	that $!A$: $\Diamond!A\lif\Box\Diamond !A$. Correct. If $!A$ is 
	permitted, then it is true in some ideal world. But every ideal world
	has access to every ideal worlds: so $!A$ is permitted in every 
	ideal world too. So it is obligatory that $!A$ is permitted. (You 
	may also note that $\Diamond!A\lif\Box\Diamond !A$ is characteristic
	of euclidean frames and the picture requires euclidean frames.)

	\item It is obligatory that: it is raining or it is not raining:
	$\Box(\Obj{\pvar{A}}\lor\lnot\Obj{\pvar{A}})$. Correct: 
	$\Obj{\pvar{A}}\lor\lnot\Obj{\pvar{A}}$
	is true at any ideal world, therefore it is obligatory. (You may 
	also note that for any logical truth $!A$, $\Box!A$ is a logical truth,
	in possible worlds semantics for $\Box$.)

	\item If it is obligatory that Boris sends the letter, then it is obligatory
	that: Boris sends the letter or Boris burns it: 
	$\Box\Obj{\pvar{A}}\lif\Box(\Obj{\pvar{A}}\lor\Obj{\pvar{B}})$.
	Correct. If $\Obj{\pvar{A}}$ is true at all ideal worlds, then 
	$Obj{\pvar{A}}\lor\Obj{\pvar{B}}$ is true at all ideal worlds. 
	(You may also notice that $\Box!A\lif\Box(!A\lor\!B)$ is valid in 
	\Log{K} and hence in our frames.)

	\item If it is obligatory that Ali replaces the lamp he has broken, then
	it is obligatory that Ali has broken the lamp. I'll assume that `the lamp'
	is short for the lamp Ali has broken. Correct. Ali replacing the 
	lamp that he has broken entails that he has broken a lamp. On the 
	picture we're examining, if it is obligatory that Ali replaces the 
	lamp he has broken, then in every ideal world Ali replaces the 
	lamp he has broken. But then in every ideal world Ali has broken a 
	lamp. On the picture we're examining, that entails that it is 
	obligatory that Ali has broken a lamp. 

	We could capture the entailment between `Ali replaces the lamp he
	has broken' and `Ali has broken a lamp` by formalizing `Ali has
	broken the lamp' as $\Obj{\pvar{A}}$ and `Ali replaces the lamp he
	has broken' as $\Obj{\pvar{A}}\land \Obj{\pvar{B}}$ (`Ali has
	broken the lamp and Ali replaces the lamp he has broken'). The
	principle, then, is 
	$\Box(\Obj{\pvar{A}}\land \Obj{\pvar{B}})\lif\Box\Obj{\pvar{A}}$.
	This is valid in \Log{K} (hence in any normal modal logic, 
	including \Log{KD45}).

	\end{enumerate} 

	\emph{Comment}. The upshot of the last item that if we find the
	principle "If it is obligatory that Ali replaces the lamp he has
	broken, then it is obligatory that Ali has broken the lamp"
	implausible, we cannot use a normal modal logic (like \Log{K} or
	the stronger standard systems) to formalize "it is obligatory
	that". Alternatively, if we do think that a normal logic like
	\Log{K} or stronger (say \Log{D} or \Log{KD45}) is right for "it
	is obligatory that", we have to accept that principle, and find a
	way to make sense of it. Both options exist in the literature. 
	For more on this you can search the `paradox of contrary to duty
	obligations'. 

	\end{enumerate}

	\end{ans}

\end{prob}

\begin{prob}
Suppose we're only talking about temperature. Our `possible worlds'
only differ in what the temperature is; $w_{1}$ is the world where
the temperature is $1$º C, $w_{-26}$ is the world where the temperature
is $-26$º C, and so on. For simplicity, we assume that we have one
world per whole-number temperature (no matter how low or high). Our
sentence letters are propositions about temperature: for instance,
if $P$ is true just at $w_{-1}$, $w_{-2}$, $w_{-3}$, it is the
proposition that the temperature is between $-1$º~C and $-3$º~C,
if $Q$ is true just at $w_{1}$ and $w_{25}$ is it the proposition
that the temperature is either $1$º C or $25$º C, and so on. 

Now suppose we think of `definitely' as follows. At a given world
$w$, it is definitely the case that $!A$---'definitely $!A$',
for short---iff $!A$ holds at worlds whose temperature is within plus
or minus $5$º C the temperature of $w$. We write $\ldet!A$
for `definitely $!A$', but $\ldet$ will work like a $\Box$
operator in modal logic. 

\begin{enumerate}
\item Given $W$ the set of worlds described as above, $V$ a valuation
of the sentence letters, define an accessibility relation $R$ to
capture the notion of `definitely'. 
\item At given world $w$, it is \emph{indefinite} that $!A$---`indefinitely
$!A$', for short---iff among the worlds whose temperature is plus
or minus $5$º C that of $w$, $!A$ holds at some and fails at
others. Write $\lindet!A$ for `it is indefinite that $!A$'.
Define $\lindet!A$ in terms of $\ldet!A$.
\item What are the properties of $R$ on that picture: it is serial, reflexive,
symmetric, transitive? Briefly justify each answer. 
\item For each of the principles or claims below, say whether it holds in
your model. (Treat `if $\ldots$ then' as a material conditional.)
Briefly justify your answers.
	\begin{enumerate}
		\item If definitely $!A$ then $!A$.
		\item If $!A$ then definitely $!A$.
		\item If indefinitely $!A$ then not-$!A$.
		\item Definitely: $!A$ or not $!A$.
		\item Definitely $!A$ or definitely not-$!A$.
		\item If definitely $!A$, then definitely definitely $!A$. (If it's
		definitely the case that $!A$, then it's definitely the case that
		it's definitely the case that $!A$.)
		\item If indefinitely definitely $!A$ then indefinitely $!A$. (If
		it's indefinitely the case that it's definitely the case that $!A$,
		then it's indefinitely the case that $!A$.)
	\end{enumerate}
\end{enumerate}


	\begin{ans}
	\begin{enumerate}
	\item $R=\Setabs{\tuple{w_i,w_j}}{i-5\geq j \i+5}$. That is, $w_i$ 
	has $R$ to $w_j$ iff $j$ is between $i-5$ and $i+5$ inclusive. 
	\item $\lindet!A$ can be defined as
	$\lnot\ldet!A\land\lnot\ldet\lnot!A$. 
	
	\emph{Comment}. $\lindet$ is a contingency operator, not a mere 
	possibility operator. That is, if we think of $\ldet!A$ has a 
	necessity operator $\Box!A$, $\lindet!A$ is not merely $\Diamond\not!A$
	but $\Diamond!A\land\Diamond\lnot!A$. 
	
	\item $R$ is reflexive (hence also serial), symmetric, but not transitive
	nor euclidean. 

	Reflexive: since $i$ is within $i-5$ and $i+5$, we have $Rw_iw_i$.

	Serial: since it is reflexive, it is serial. 

	Symmetric. Suppose $Rw_iw_j$. Then $j$ is within $i-5$ and $i+5$.
	Therefore $i$ is within $j+5$ and $j-5$. (If $j$ is at most 5 steps
	way from $i$, $i$ is at most 5 steps away from $j$.) Therefore 
	$Rw_jw_i$. Hence $R$ is symmetric.

	Not transitive. We have $Rw_1w_5$, $Rw_5w_{10}$ but not 
	$Rw_1w_{10}$.

	Not euclidean. We have $Rw_5w_1$, $Rw_5w_{10}$, but not 
	$Rw_1w_{10}$.
	
	\item Principles:
		\begin{enumerate}
			\item If definitely $!A$ then $!A$: $\ldet !A\lif!A$. Correct
			(by reflexivity).
			\item If $!A$ then definitely $!A$: $!A \lif \ldet!A$.
			Incorrect. Consider $\Obj{\pvar{A}}$ true at $w_5$ but 
			nowhere else. ($\Obj{\pvar{A}}$ could be the claim that the 
			temperature is 5ºC, for instance.)
			\item If indefinitely $!A$ then not-$!A$: $\lindet!A \lif
			\lnot!A$. Incorrect. Consider the same model as in the 
			the previous question. We have $\lindet\Obj{\pvar{A}}$
			at $w_5$ but $\lnot\Obj{\pvar{A}}$ false at $w_5$.
			\item Definitely: $!A$ or not $!A$. $\ldet(!A\lor\lnot!A)$. 
			Correct, because $!A\lor\lnot!A$ holds at every world.
			\item Definitely $!A$ or definitely not-$!A$. Incorrect. 
			Consider the same model as previously: $\Obj{\pvar{A}}$ 
			true at $w_5$ but nowhere else. Neither $\Obj{\pvar{A}}$ 
			nor its negation $\lnot\Obj{\pvar{A}}$ are true at all 
			worlds accessible to $w_5$, namely $w_0,\ldots,w_{10}$. 
			Therefore neither $\ldet\Obj{\pvar{A}}$ nor $\ldet\lnot\Obj{\pvar{A}}$
			hold at $w_5$. 
			\item If definitely $!A$, then definitely definitely $!A$: 
			$\ldet!A\lif\ldet\ldet!A$.
			Incorrect. Consider a !!{structure} with $\Obj{\pvar{A}}$ true 
			at all worlds accessible from $w_5$, namely $w_0,\ldots,w_{10}$,
			and false everywhere else. We have $\ldet\Obj{\pvar{A}}$ at
			$w_5$. But $\Obj{\pvar{A}}$ is false at $w_{11}$, within
			5 steps of $w_6$, so $\ldet\Obj{\pvar{A}}$ is not true at 
			$w_6$. Since $w_6$ is acessible from $w_5$, $\ldet\ldet!A$
			is not true at $w_5$. 
			
			You may also note that $\ldet!A\lif\ldet\ldet!A$, 
			with the $\ldet$ understood as a $\Box$-like operator, is 
			the characteristic schema \Ax{4} that fails in non-transitive frames.
			Note also how our counterexample used a transitivity failure:
			we used the fact that $w_5$ has access to $w_6$ which has 
			access to $w_{11}$, but $w_5$ does not itself have access
			to $w_{11}$.

			\item If indefinitely definitely $!A$ then indefinitely $!A$:
			$\lindet\ldet!A\lif\lindet!A$. Incorrect. Consider the 
			!!{structure} given in answer to the previous question. 
			Since $\Obj{\pvar{A}}$ is true at all worlds accessible 
			from $w_5$, $\ldet\Obj{\pvar{A}}$ is true at $w_5$, 
			and therefore $\lindet\Obj{\pvar{A}}$ is false at $w_5$.
			But we also have that $\ldet\Obj{\pvar{A}}$ is true at 
			some but not all worlds accessible to $w_5$: it is true 
			at $w_5$, which $w_5$ has access to (reflexivity), but we 
			also that it is false at $w_6$, which $w_5$ has access to.
			Since $\ldet\Obj{\pvar{A}}$ is true at 
			some but not all worlds accessible to $w_5$, 
			$\lindet\ldet\Obj{pvar{A}}$ is false at $w_5$.

			Therefore the schema $\lindet\ldet!A\lif\lindet!A$ is 
			not valid. 

		\end{enumerate}
	\end{enumerate}
\end{ans}

\end{prob}

\begin{prob}
Suppose we think of logical `necessity' in the following way: it is
logically necessary that $!A$ iff $!A$ is true no matter how we
interpret its !!{propositional variable}s. We capture the idea with a
relational model where $\Box !A$ is true at a world iff every
interpretation of sentence letters makes $!A$ true. We say that a
!!{formula} $!A$ is ``logically necessary'' just if $\Box !A$ is true
in that model. 

\begin{enumerate}
	\item We let each world $w$ be a (propositional-logic) !!{valuation}
	of the !!{propositional variable}s. We write $w(\Obj{\pvar{A}})=\True$ when the interpretation
	that consititutes world $w$ assigns $\True$ to $\Obj{\pvar{A}}$, $w(\Obj{\pvar{A}})=\False$
	if it assigns it false. Our set of worlds $W$ is the set of all !!{valuation}s.
	How do we define an accessibility relation $R$ and a modal interpretation
	$V$ to get the desired model $\tuple{W,R,V}$?
	\item Which modal logic does logical necessity obey on that picture? (Give
	the strongest of the standard modal logics \textbf{K, D}, \textbf{T},
	\textbf{B}, \textbf{S4}, \textbf{S5} that applies.) Briefly justify
	your answer.
	\item Are the following true in your model:
	\begin{enumerate}
		\item $\Box \Obj{\pvar{A}} \lif\Box\Box \Obj{\pvar{A}}$
		\item $\Diamond\lnot \Obj{\pvar{A}}$.
		\item $\Box\Diamond\lnot \Obj{\pvar{A}}$.
	\end{enumerate}
	\item Consider the uniform substitution rule:

	If $!A$ is a logical necessity then for any sentence letter $\Obj{\pvar{A}}$
	that appears in $!A$ and any formula $!B$, the result of substituting $!B$
	for $\Obj{\pvar{A}}$ in $!A$, namely $\Subst{!A}{!B}{\Obj{\pvar{A}}}$ is a logical necessity
	too.

	Explain why the rule \emph{fails} in the picture we just developped. 

\end{enumerate}


\end{document}