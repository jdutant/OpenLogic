% Part: exercise-booklet
% Chapter: propositional-modal-logic
% Section: understanding

\documentclass[../../../include/open-logic-section]{subfiles}

\begin{document}

\olfileid{exb}{pml}{und}

\olsection{Check Your Understanding}

\begin{prob}
    Define $\Diamond$ in terms of $\Box$. Define $\Box$ in terms of $\Diamond$.
\end{prob}

\begin{prob}
    Formalize in the best way you can the following in the language
    of propositional modal logic:
    \begin{itemize}
        \item Omar has to study or get a job, but he doesn't have to do both.
        \item She must be at home. And if she can't be running if she's at home.
        Therefore she can't be running.
        \item It's possible that God exists necessarily.
    \end{itemize}
\end{prob}

\begin{prob}[Possibility vs Contingency]
    What is the difference between \emph{possibility} and \emph{contingency}?

    $\Diamond !A$ stands for `it is possible that $!A$'. Let $\lindet !A$
    stand for `it is contingent that $!A$'. Define $\lindet$ in terms of $\Diamond$.

    Say which of the following are correct (the answer is the same whether
    you use $\Log{K}$ or standard stronger systems like $\Log{S5}$):
    \begin{itemize}
        \item $\Box !A \Entails \lnot \Diamond !A$
        \item $\Box !A \Entails \lnot \lindet !A$
        \item $\Diamond !A \Entails \Diamond \lnot !A$
        \item $\lindet !A \Entails \lindet \lnot !A$
        \item $\Diamond !A \Entails \lnot\Box !A$
        \item $\lindet !A \Entails \lnot\Box !A$
    \end{itemize}
\end{prob}

\begin{prob}[Distributing $\Box$ and $\Diamond$ over $\lor$ and $\land$]
    \ollabel{moddistrib}
    Below are some inferences that are valid and some that are not 
    valid. Can you explain why each of the following is the case? 
    For valid ones, check that you see how the semantics ensures that it holds. 
    For invalid ones, check that you can think of a counterexample.
    \begin{itemize}
        \item $\Box(!A\land!B) \Entails{\Log{K}} \Box !A \land \Box !B$
        \item $\Box !A \land \Box !B \Entails{\Log{K}} \Box(!A \land !B)$
        \item $\Diamond(!A\land!B) \Entails{\Log{K}} \Diamond !A \land \Diamond !B$
        \item $\Diamond(!A)\land\Diamond(!B) \Entails/{\Log{K}} \Diamond (!A \land !B)$
        \item $\Box(!A\lor!B) \Entails/{\Log{K}} \Box !A \lor \Box !B$
        \item $\Box !A \lor \Box !B \Entails/{\Log{K}} \Box (!A \lor !B)$
        \item $\Diamond(!A\lor!B) \Entails{\Log{K}} \Diamond !A \lor \Diamond !B$
        \item $\Diamond !A \lor \Diamond !B \Entails/{\Log{K} \Diamond(!A\lor!B)}$ 
    \end{itemize}
\end{prob}

\begin{prob}
    Check that the claims of \olref{moddistrib} still hold when we replace $\Log{K}$
    by any stronger system up to $\Log{S5}$:
    \begin{itemize}
        \item We know that the validites in $\Log{K}$ are still valid 
        in $\Log{S5}$. (Why?)
        \item For the invalidities, it is enough to show that they are
        still not valid in $\Log{S5}$. To do this think of countermodels
        to each in $\Log{S5}$.
        \item Once we know that they are invalid in $\Log{S5}$, we know
        they are invalid in weaker systems, including
         $\Log{D},\Log{T},\Log{S4},\Log{B}$. (Why?)
    \end{itemize}

\end{prob}

\end{document}

