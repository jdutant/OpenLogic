% Part: first-order-logic
% Chapter: syntax-and-semantics
% Section: negative-free-logic

\documentclass[../../../include/open-logic-section]{subfiles}

\begin{document}

\olfileid{frl}{syn}{pfl}

\olsection{Positive Free Logic}

The guiding idea of positive free logic is that atomic claims do not have
existence committments. In natural language the following may seem true:

\begin{enumerate}
  \item Pegasus is Pegasus.
  \item Pegasus is a winged horse.
  \item Some children are waiting for Santa Claus.
  \item The round square is a square.
\end{enumerate}

If they are true, however, they should not entail that Pegasus, 
Santa Claus or the round square exist. Moreover, the following seem 
false:

\begin{enumerate}
  \item Pegasus is Santa Claus.
  \item Santa Claus is a winged horse.
  \item Some children are waiting for Pegasus.
\end{enumerate}

Accordingly, positive free logic allows that, where $a$ is empty in a 
!!{structure} $\Struct M$ and $F$ some predicate, $\Atom{F}{a}$ may be 
either true or false in $\Struct M$. So $\Atom{F}{a}$ does not 
entail $\lfrexists a$. Moreover, it allows that predicates behave 
differently with different empty !!{constant}s: for instance, where 
$a$ and $b$ are empty in a !!{structure} $\Struct M$ and $F$ some 
predicate, we may have $\Atom{F}{a}$ true and $\Atom{F}{b}$ false
in $\Struct M$. We will have $\eq[a][a]$ and $\eq[b][b]$ but 
$\eq[a][b]$ may be either true or false in a given $\Struct M$. In short,
the truth of falsity of atomic !!{formula}s with empty !!{constant}s is
\emph{non-trivial} in positive free logic.

\begin{explain}
A semantics for positive free logic must therefore find ways of distinguishing
true and false atomic !!{formula}s with empty !!{constant}s. The most
common way to do so is by using a \emph{dual-domain} semantics. On this
kind of semantics, all empty !!{constant}s referents, but these referents
are treated as ``non-existing'' and left out of the domain over which
quantifiers operate. These referents can be used to give non-trivial
truth conditions for identity claims (where $a$ and $b$ are empty,
$\eq[a][b]$ holds in a model iff their non-existing referents are the 
same). The empty !!{constant}s' referents can also be part of the extension 
of predicates, which in turns determines the the truth of falsity of claim 
using these !!{constant}s.
\end{explain}

\begin{defn}[!!^{structure}s]
In the positive system of free logic, $\Log NFL$,
\article{structure} \emph{!!{structure}}~$\Struct M$ for a language
$\Lang L_{\Log{FL}}$ of free logic consists of:
\begin{enumerate}
\item \emph{(Inner) Domain:} a set, $\Domain M$,
\item \emph{Outer Domain:} a non-empty set, $\OuterDomain M$ that
	includes $\Domain M$ ($\Domain M \subseteq \OuterDomain M $),
\item \emph{Interpretation of !!{constant}s:} for any !!{constant}~$c$ of
  $\Lang L_{\Log{FL}}$, its interpretation $\Assign{c}{M}$ is !!a{element} of 
  $\OuterDomain M$,
\item \emph{Interpretation of !!{predicate}s:} for each $n$-place
  !!{predicate}~$R$ of $\Lang L_{\Log{FL}}$ (other than $\eq$), an $n$-place
  relation $\Assign{R}{M} \subseteq \OuterDomain M^n$
\tagitem{fnTerms}{\item \emph{Interpretation of !!{function}s:} for each $n$-place
  !!{function}~$f$ of $\Lang L_{\Log{FL}}$, an $n$-place function $\Assign{f}{M}
  \colon \OuterDomain M^n \to \OuterDomain M$.}{}
\end{enumerate}
The system of positive free logic is \emph{inclusive} if $\Domain M$ is
allowed to be empty. There are several such !!{structure}s, as they may
still differ on the !!{element}s of $\OuterDomain M$ outside of $M$ and the 
interpretations of constants and predicates. We call a !!{structure} with
empty $\Domain M$ an \emph{empty structure}. 
\end{defn}

\begin{intro}
Our outer domain includes the inner domain. Some authors call instead "outer" 
domain a set of objects that are not in the (inner) domain, that is the set
$\OuterDomain M - \Domain M$.  
\end{intro}

\begin{defn}[Variable Assignment]
A \emph{variable assignment}~$s$ for a !!{structure}~$\Struct{M}$ is a
  function such that for any !!{variable}~$x$ of
  $\Lang L_{\Log{FL}}$, $s(x) \in \OuterDomain M$. 
\end{defn}

\begin{explain}
Interpretations and variable assignments are total functions: they 
assign an object to every !!{constant} and !!{variable}, though 
not necessarily an existing object (in) the inner domain. Interpretations
are required to be total because that is needed to give non-trivial
truth conditions to sentences with empty !!{constant}s. We allow
variable assignements to assign non-existing objects to !!{variable}s in
order to ensure that there are assignements even in empty
 !!{structure}s. For even in empty !!{structure}s the definition ensures 
 that $\Domain{M_O}$ is not empty and can be used to assign values to
 !!{variable}s.
\end{explain}

\begin{defn}[!!^{value} of Terms]
If $t$ is a term of the language~$\Lang L_{\Log{FL}}$, $\Struct M$ is a
!!{structure} for~$\Lang L_{\Log{FL}}$, and $s$ is a !!{variable} assignment
for~$\Struct M$, the \emph{!!{value}}~$\Value{t}{M}[s]$ is defined as
follows:
\begin{enumerate}
\item \indcase{t}{c}{$\Value{\indfrm}{M}[s] = \Assign{\indcomplex}{M}$.}
\item \indcase{t}{x}{$\Value{\indfrm}{M}[s] = s(\indcomplex)$.}
\tagitem{fnTerms}{\indcase{t}{\Atom{f}{t_1, \ldots, t_n}}{
\[
\Value{\indfrm}{M}[s] = \Assign{f}{M}(\Value{t_1}{M}[s], \ldots,
\Value{t_n}{M}[s])\textrm{, if defined}.
\]}
}{}
\end{enumerate}
\end{defn}

\begin{defn}[Assignement variants]
  If $s$ is a !!{variable} assignment for a !!{structure}~$\Struct M$,
  and $o$ an object in the \emph{inner} $\Domain M$, and $v$ any variable, the
  \emph{assignment variant} $s(o/v)$ is the assignement that assigns $o$
  to $s$ and agrees with $s$ on every other variable. That is, for any
  variable $u$:
  $$
  s(o/v)(u)=\begin{cases}
    o & \text{if $u$ is $v$},\\
    s(u) & \text{otherwise}.  
  \end{cases}
  $$
  We call $s(o/v)$ a \emph{$v$-variant} of $s$.
\end{defn}

\begin{explain}
Note that while an assignment $s$ can assign any object in the outer 
domain to $x$, the assignement variants $s(o/x)$ only assign
\emph{inner} domain objects to $x$. This will ensure that our 
quantifiers only range over the \emph{inner} domain. 

If $\Struct M$ has an empty inner domain, then there are assignments
in $s$, but no assignment variant $s(o/x)$ of $s$ for any $x$. This
will have the desired result that in that !!{structure},
$\lforall[x][!A]$ !!{formula}s are all trivially true and
$\lexists[x][!A]$ all trivially false. 
\end{explain}

\begin{defn}[Satisfaction]
\ollabel{defn:satisfaction}
Satisfaction of a !!{formula}~$!A$ in a !!{structure}~$\Struct M$
relative to a !!{variable} assignment~$s$, in symbols:
$\Sat{M}{!A}[s]$, is defined recursively as follows. (We write
$\Sat/{M}{!A}[s]$ to mean ``not $\Sat{M}{!A}[s]$.'')
\begin{enumerate}
\tagitem{prvFalse}{%
  \indcase{!A}{\lfalse}{$\Sat/{M}{\indfrm}[s]$.}}{}

\tagitem{prvTrue}{%
  \indcase{!A}{\ltrue}{$\Sat{M}{\indfrm}[s]$.}}{}

\item \indcase{!A}{\Atom{R}{t_1, \dots, t_n}}{$\Sat{M}{\indfrm}[s]$
  iff $\langle \Value{t_1}{M}[s], \dots, \Value{t_n}{M}[s] \rangle \in
  \Assign{R}{M}$.}

\item \indcase{!A}{\Atom{\lfrexists}{t}}{$\Sat{M}{\indfrm}[s]$ iff 
  $\Value{t}{M}[s] \in \Domain M$.}

\item \indcase{!A}{\eq[t_1][t_2]}{$\Sat{M}{\indfrm}[s]$ iff 
  $\Value{t_1}{M}[s] = \Value{t_2}{M}[s]$.}

\tagitem{prvNot}{%
  \indcase{!A}{\lnot !B}{$\Sat{M}{\indfrm}[s]$ iff
    $\Sat/{M}{!B}[s]$.}}{}

\tagitem{prvAnd}{%
  \indcase{!A}{(!B \land !C)}{$\Sat{M}{\indfrm}[s]$ iff $\Sat{M}{!B}[s]$
    and $\Sat{M}{!C}[s]$.}}{}

\tagitem{prvOr}{%
  \indcase{!A}{(!B \lor !C)}{$\Sat{M}{\indfrm}[s]$ iff
    $\Sat{M}{!A}[s]$ or $\Sat{M}{!B}[s]$ (or both).}}{}

\tagitem{prvIf}{%
  \indcase{!A}{(!B \lif !C)}{$\Sat{M}{\indfrm}[s]$ iff $\Sat/{M}{!B}[s]$
    or $\Sat{M}{!C}[s]$ (or both).}}{}

\tagitem{prvIff}{%
  \indcase{!A}{(!B \liff !C)}{$\Sat{M}{\indfrm}[s]$ iff either both
    $\Sat{M}{!B}[s]$ and $\Sat{M}{!C}[s]$, or neither $\Sat{M}{!B}[s]$
    nor $\Sat{M}{!C}[s]$.}}{}

\tagitem{prvAll}{%
  \indcase{!A}{\lforall[x][!B]}{$\Sat{M}{\indfrm}[s]$ iff for every 
  object $o$ in the \emph{inner} domain $\Domain M$, 
  $\Sat{M}{!B}[s(o/x)]$.}}{}

\tagitem{prvEx}{%
  \indcase{!A}{\lexists[x][!B]}{$\Sat{M}{\indfrm}[s]$ iff for some 
  object $o$ in the \emph{inner} domain $\Domain M$, 
  $\Sat{M}{!B}[s(o/x)]$}}{}
\end{enumerate}
\end{defn}

Truth and validity are defined as usual.

\begin{defn}[Truth]
Where $!A$ is a closed !!{formula}, we say that $A$ is \emph{true in}, or
\emph{satisfied by} a !!{structure} $\Struct M$, written $\Sat{M}{!A}$ iff 
$A$ is satisfied relative to any assignement in $\Struct M$:
$$\Sat{M}{!A}\textrm{ iff }\Sat{M}{!A}[s]\textrm{ for any }s\textrm{ in }\Struct M$$
\end{defn}

\begin{defn}[Validity]
A !!{formula} $!A$ is \emph{valid}, $\Entails !A$, iff $\Sat{M}{!A}$ for every
!!{structure}~$\Struct M$.
\end{defn}

\begin{ex}
If $\Sat{M}{Fa}$ then $\Sat{M}{\lfrexists a}$.
\end{ex}

\begin{ex}
If $\Entails/_{\Log{PFL}} \eq[a][a]\lif\lfrexists a$.
\end{ex}

\begin{ex}
$\Entails/_{\Log{PFL}} \lexists x (x = x)$.
\end{ex}

\end{document}
