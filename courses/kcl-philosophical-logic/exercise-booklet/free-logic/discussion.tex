% Part: exercise-booklet
% Chapter: free-logic
% Section: discussion

\documentclass[../../../include/open-logic-section]{subfiles}

\begin{document}

\olfileid{exb}{frl}{dis}

\olsection{Discussion}

\begin{prob}
Give an English equivalent of each of the following and explain
 briefly why both are valid in standard first-order logic.
\begin{enumerate}
\item $\vDash \lexists[x][(\eq[x][x])]$
\item $\vDash \lexists[x][(\eq[x][a])]$
\end{enumerate}

\noindent Suppose we extend first-order logic with a necessity
operator $\Box$ with the  
necessitation property:
\begin{defn}[Necessitation rule]
If $\Entails !A$, then $\Entails \Box !A$
\end{defn}

\begin{enumerate}
	\item[3.] Applying the necessitation rule to the two valid
	formulas of first-order logic above. Which validities do we get?
	(Give only two.) 
	\item[4.] Explain briefly why one would not want to accept the
	resulting validities.
\end{enumerate}

\begin{ans}
\begin{enumerate}
\item !!^{structure}s of first-order logic have a non-empty domain. In
	any such structure $\lexists[x][(\eq[x][x])]$ is true (since,
	unlike in free logics, it is guaranteed that there are
	assignements, and any assignments will trivially assign the same
	object to $x$ as they assign to $x$). So
	$\lexists[x][(\eq[x][x])]$ is valid in first-order logic. 
\item In !!{structure}s of first-order logic the interpretation
	assigns an object of the  domain to every !!{constant}. Hence in
	any !!{structure} the interpretation of $a$ is an object of the
	domain, so $\eq[x][a]$ will be true relative to some assignment
	(namely, any assignment that assigns to $x$ the object that $a$
	denotes). Hence $\lexists[x][(\eq[x][a])]$ valid in first-order
	logic. 
\item $\Box\lexists[x][(\eq[x][x])]$ and
$\Box\lexists[x][(\eq[x][a])]$.
\item $\Box\lexists[x][(\eq[x][x])]$ says that it is necessary that
	there is something rather than nothing. But we may think that it
	is false: it could have been that there is nothing. \\
	Suppose that $a$ is a name for me. $\Box\lexists[x][(\eq[x][a])]$
	says that necessarily there is something that is me. But I could
	have failed to exist; that is, it could have been the case that
	nothing is me. (I do not necessarily exist.)
\end{enumerate}
\end{ans}
\end{prob}

\begin{prob}[advanced, optional]
Negative free logic is committed to primitive predicates.
\begin{enumerate}
	\item Explain what that means.
	\item Explain why that is the case.
	\item Give a reason to think that this is a problem.
\end{enumerate}
\begin{ans}
\item In negative free logic we cannot introduce a predicate ``nadult'' 
	that means ``not an adult'' (i.e. by the definition $\Atom{\Obj N}{t} \liff 
	\lnot \Atom{\Obj A}{t}$ where $A$ stands for ``is an adult'' and 
	$N$ for ``is a nadult''). Negative free logic must say that 
	``Bob is a nadult'' is not genuinely of the form $\Atom{\Obj N}{b}$ because
	``nadult'' is not an elementary or primitive predicate.
\item That is because in negative free logic formulas of the form $Fa$ are
	existence-entailing but  formulas of the form $\lnot \Atom{\Obj F}{a}$ are not. Now
	if ``Bob is a nadult'' is definitionally equivalent to
	``Bob is not an adult'' and the latter is not existence-entailing, 
	then the former is not existence-entailing either. Hence the former
	could not be expressed in negative free logic as $\Atom{\Obj N}{a}$, 
	because that formula is existence-entailing in negative free logic.
\item If we think that logic is formal, independent of subject matter,
	then logic should not draw differences between predicates. If logic is
	formal, independent of subject matter, then logical truths should be 
	truth \emph{no matter what the non-logical terms mean}. But negative
	free logic cannot preserve the idea that logical truths are 
	independent of subject matter, because it must make a difference
	between predicates like ``adult'' and ``nadult''. 
\end{ans}
\end{prob}

\begin{prob}[advanced, optional]
Consider the following formula:
$$\lforall[x][(\Atom{\Obj F}{x} \lif \lfrexists x)]$$
\begin{enumerate}
	\item Explain why this is valid in positive and negative free logic.
	\item Give a reason to think that this is a problem.
\end{enumerate}
\begin{ans}
\begin{enumerate}
\item Any assignment (in a !!{structure} of either positive or
	negative free logic) assigns to $x$ some object of the domain
	(negative free logic) or inner domain (positive free logic). In
	both cases, $\lfrexists x$ is true relative to that assignement.
	So $\lforall[x][(\Atom{\Obj F}{x} \lif \lfrexists x)]$ is true in
	any !!{structure} of positive or negative free logic.
\item In metaphysics, we want to say that certain predicates denote
	\emph{sufficient conditions for existence}: whatever thinks exist,
	whatever is material exists, and so on. On the face of it, these
	are expressed by $\lforall[x][(\Atom{\Obj F}{x} \lif \lfrexists
	x)]$ or $\Box\lforall[x][(\Atom{\Obj F}{x} \lif \lfrexists x)]$,
	where $\Obj F$ stands for the predicate in question. However, as the
	above result shows, these claims turn out to be logical truths for
	\emph{any} predicate in free logic (for the second, in a modal
	free logic with the rule that every logical truth is necessary).
	So we cannot express the idea that some predicates, as opposed to
	others, state sufficient conditions for existence. It would be
	wrong, for instance, to say that ``everything that is material
	exists, but it's not the case that everything that is a ghost
	exists'' if that is translated in a free logic by:
	$$\lforall[x][(\Atom{\Obj M}{x} \lif \lfrexists x)] \land \lnot
    \lforall[x][(\Atom{\Obj G}{x} \lif \lfrexists x)]$$ or, in a modal
    free logic: $$\Box \lforall[x][(\Atom{\Obj M}{x} \lif \lfrexists
    x)] \land \lnot \Box \lforall[x][(\Atom{\Obj G}{x} \lif \lfrexists
    x)]$$ for the second conjunct would be a logical falsehood. 
\end{enumerate}
\end{ans}
\end{prob}


\end{document}