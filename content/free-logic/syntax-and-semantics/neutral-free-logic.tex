% Part: first-order-logic
% Chapter: syntax-and-semantics
% Section: negative-free-logic

\documentclass[../../../include/open-logic-section]{subfiles}

\begin{document}

\olfileid{frl}{syn}{neu}

\olsection{Neutral Free Logic}

The guiding idea of neutral free logic is that atomic claims with empty
terms are neither true nor false: they produce \emph{truth-value gaps}. 
Neutral free logic agrees with negative free logic and classical first-order
logic on the idea that atomic claims have existence committments. Where $F$ 
is some predicate other than $\lfrexists$, $\Atom{F}{a}$ cannot be 
true if $a$ is empty. But instead of treating atomic
claims involving empty terms as false, it treats them as neither true nor 
false. 

\begin{explain}
One way to motivate neutral free logic is to consider pairs of claims like
the following: 

\begin{enumerate}
  \item Pegasus is a horse.
  \item Pegasus is not a horse. 
\end{enumerate}

It is difficult to endorse either claim. Saying that Pegasus is a horse 
suggests that there is horse called Pegasus. But saying that Pegasus isn't
a horse suggests that there is a thing that is not a horse. Either way sounds
wrong. But all bivalent logicians must say that one of these is true and
the other false. Positive free logicians may (but need not to) say the first is true. 
Negative free logicians must say that the second is true. Classical logicians 
too must say that the second is true --- unless they deny these sentences express
claims at all. Neutral free logic, by contrast, says that neither is true.
\end{explain}

Semantics for neutral free logic must therefore accomodates truth-valueless
!!{formula}s. But there are several ways to do so that yield different 
systems of logic. The systems all agree on these points:

\begin{itemize}
	\item If a !!{formula} doesn't involve empty terms, it is evaluated
	as in classical first-order logic. 
	\item If an atomic !!{formula} $\Atom{R}{t_1,\ldots,t_n}$ (where $R$
	is some $n$-ary predicate other than $\lfrexists$) contains a
	term $t_i$ that is empty (in a !!{structure}), it is neither true nor 
	false (in a !!{structure}).
	\item If a !!{formula} $!A$ is neither true nor false, then its negation
	$\lnot !A$ is neither true nor false either, and conversely. 
\end{itemize}

Moreover, all can use the same !!{structure}, namely the same as
negative free logic. 

\begin{defn}
In a neutral system of free logic, \article{structure} 
\emph{!!{structure}}~$\Struct M$ for a language
$\Lang L_{\Log{FL}}$ of free logic consists of:
\begin{enumerate}
\item \emph{Domain:} a set, $\Domain M$,
\item \emph{Interpretation of !!{constant}s:} for any !!{constant}~$c$ of
  $\Lang L_{\Log{FL}}$, its interpretation $\Assign{c}{M}$ is !!a{element} of 
  $\Domain M$ or undefined,
\item \emph{Interpretation of !!{predicate}s:} for each $n$-place
  !!{predicate}~$R$ of $\Lang L_{\Log{FL}}$ (other than $\eq$), an $n$-place
  relation $\Assign{R}{M} \subseteq \Domain{M}^n$
\item \emph{Interpretation of !!{function}s:} for each $n$-place
  !!{function}~$f$ of $\Lang L_{\Log{FL}}$, an $n$-place function $\Assign{f}{M}
  \colon \Domain{M}^n \to \Domain{M}$
\end{enumerate}
The system is \emph{inclusive} if it includes the \emph{empty !!{structure}}
for which $\Domain M$ is empty. 
\end{defn}

But they diverge on how to define satisfaction and truth on that !!{structure}.
The main division is between \emph{truth-functional} systems, on the one hand,
and \emph{supervaluationism}, on the other. The options are detailed in chapters
on many-valued logics. 

\end{document}
