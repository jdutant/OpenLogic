% Part: normal-modal-logic
% Chapter: natural-deduction 
% Section: PL-rules

\documentclass[../../../include/open-logic-section]{subfiles}

\begin{document}

\olfileid{nml}{nd}{plr}

\olsection{Simplifying propositional logic rules}

\subsection{Rule of propositional consequence \Log{PL}}

We introduce a rule to omit purely propositional logic steps. 

\begin{defish}
When $!B$ can be derived from $!A_1,\ldots,!A_n$ ($n\geq 0$) using
propositional logic rules alone, we may simply write: 
\begin{prooftree}
    \AxiomC{}\DeduceC{$!A_1$}
    \AxiomC{}\DeduceC{$\ldots$}
    \AxiomC{}\DeduceC{$!A_n$}
    \RightLabel{\Log{PL}}
    \TrinaryInfC{$!B$}
\end{prooftree}
\end{defish}

\begin{ex}
The following !!{derivation} of $!A\lor!B,\lnot!A\Proves{\Log{K}}!B$:
\begin{prooftree}
\AxiomC{$!A\lor!B$}
    \AxiomC{$\Discharge{!A}{1}$}
    \AxiomC{$\lnot!A$}
    \RightLabel{\Elim{\lnot}}
    \BinaryInfC{$!B$}
            \AxiomC{$!B$}
    \DischargeRule{\Elim{\lor}}{1}
    \TrinaryInfC{$!B$}
\end{prooftree}
may be abbreviated:
\begin{prooftree}
    \AxiomC{$!A\lor!B$}
        \AxiomC{$\lnot!A$}
        \RightLabel{\Log{PL}}
    \BinaryInfC{$!B$}
\end{prooftree}
    
\end{ex}

\end{document}