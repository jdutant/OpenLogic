% Part: normal-modal-logic
% Chapter: natural-deduction 
% Section: derived-rules-for-K.tex

\documentclass[../../../include/open-logic-section]{subfiles}

\begin{document}

\olfileid{nml}{nd}{drl}

\olsection{Derived rules for \Log{K}}

To following rules can be derived from the core ones, so they are 
superfluous. However, we may use them to simplify our !!{derivation}s.

\iftag{prvDiamond}{
\subsubsection{Duality rules}
We have six further Duality rules. They all apply the same
procedure: move a negation `through' a modal operator and switch the
main modal operator of a !!{formula}, and eliminate a double
negation if it arises. Thus $\Box\lnot$ can be converted
to $\lnot\Diamond$ and conversely, and $\lnot\Diamond\lnot$ can be
converted to $\Box$. 

\begin{defish}
    \begin{tabular}{cccc}
        \AxiomC{}\DeduceC{$\lnot\Diamond!A$}
        \RightLabel{Dual}
        \UnaryInfC{$\Box\lnot!A$}
        \DisplayProof
    &
        \AxiomC{}\DeduceC{$\Box\lnot!A$}
        \RightLabel{Dual}
        \UnaryInfC{$\lnot\Diamond!A$}
        \DisplayProof
    &
        \AxiomC{}\DeduceC{$\Diamond\lnot!A$}
        \RightLabel{Dual}
        \UnaryInfC{$\lnot\Box!A$}
        \DisplayProof
    &
        \AxiomC{}\DeduceC{$\Box\lnot!A$}
        \RightLabel{Dual}
        \UnaryInfC{$\lnot\Diamond!A$}
        \DisplayProof
    \\[3em]
        \AxiomC{}\DeduceC{$\lnot\Diamond\lnot!A$}
        \RightLabel{Dual}
        \UnaryInfC{$\Box!A$}
        \DisplayProof
    &
        \AxiomC{}\DeduceC{$\Box!A$}
        \RightLabel{Dual}
        \UnaryInfC{$\lnot\Diamond\lnot!A$}
        \DisplayProof

    \end{tabular}
    
\end{defish}

}

\iftag{prvDiamond}{
    \subsubsection{Generalized $\Box$\Ax{K}, \Intro{\Diamond}\Ax{K}}
}{
    \subsubsection{Generalized $\Box$\Ax{K}}
}

Where $k\geq 0$:

\begin{defish}

\AxiomC{}\DeduceC{$\Box!A_1$}
    \AxiomC{}\DeduceC{$\ldots$}
    \AxiomC{}\DeduceC{$\Box!A_k$}
    \AxiomC{$\Discharge{!A_1}{n},\ldots,\Discharge{!A_k}{n}$ \emph{at most}}\DeduceC{$!B$}
    \DischargeRule{$\Box$\Ax{K}}{n}
    \QuaternaryInfC{$\Box!B$}
\DisplayProof

\iftag{prvDiamond}{
\bigskip
\AxiomC{}\DeduceC{$\Diamond!A$}
\AxiomC{}\DeduceC{$\Box!B_1$}
\AxiomC{}\DeduceC{\ldots}
\AxiomC{}\DeduceC{$\Box!B_k$}
\AxiomC{$\Discharge{!A}{n},\Discharge{!B_1}{n},\ldots,\Discharge{!B_k}{n}$ \emph{at most}}\DeduceC{$!C$}
\DischargeRule{$\Diamond$\Ax{K}}{n}
\QuinaryInfC{$\Diamond!C$}
\DisplayProof
}{} 

\end{defish}

Since we allow $k=0$, the $\Box$ premises are optional in both rules, 
but the $\Diamond!A$ premise is mandatory in the $\Diamond$\Ax{K} 
rule. Therefore the first three !!{derivation}s below are correct but
the fourth incorrect:

\bigskip\noindent
\begin{tabular}{cccc}
    \AxiomC{$\Box!A$}
        \AxiomC{$\Discharge{!A}{n}$}
        \DeduceC{$!B$}
    \DischargeRule{$\Box$\Ax{K}}{n}
    \BinaryInfC{$\Box!B$}
    \DisplayProof
&
    \AxiomC{\emph{no undisch. as.}}
    \DeduceC{$!B$}
    \RightLabel{$\Box$\Ax{K}}
    \UnaryInfC{$\Box!B$}
    \DisplayProof
&
    \AxiomC{$\Diamond!A$}
        \AxiomC{$\Discharge{!A}{n}$}
        \DeduceC{$!B$}
    \DischargeRule{$\Box$\Ax{K}}{n}
    \BinaryInfC{$\Diamond!B$}
    \DisplayProof
&
    \AxiomC{\emph{no undisch. as.}}
    \DeduceC{$!B$}
    \RightLabel{{\color{red}Incorrect}}
    \UnaryInfC{$\Diamond!B$}
    \DisplayProof
\end{tabular}
\bigskip

\begin{note}
We do \emph{not} have a rule allowing the aggregation of possibilities:
\begin{prooftree}
    \AxiomC{}
    \DeduceC{$\Diamond!A$}
    \AxiomC{}
    \DeduceC{$\Diamond!B$}
        \AxiomC{$\Discharge{!A}{n}, \Discharge{!B}{n}$\emph{ at most}}
        \DeduceC{$!C$}
    \DischargeRule{{\color{red}Incorrect}}{n}
    \TrinaryInfC{$\Diamond!C$}
\end{prooftree}
Such a rule would not be sound, since $\Diamond!A,\Diamond!B 
\Entails/[\Log{K}] \Diamond{!A\land!B}$.
\end{note}

\begin{explain}
The $\Diamond$\Ax{K} rule captures the idea
that \emph{possibility is closed under single-premise necessary
consequence}, namely that if $!C$ is a necessary consequence of $!A$,
then $\Diamond!C$ is a logical consequence of $\Diamond!A$. To clarify this idea,
start with the related idea that possibility is closed under single-premise
\emph{logical} consequence: if $!C$ is a \emph{logical} consequence of 
$!A$, then $\Diamond!C$ is a logical consequence of $\Diamond!A$. 
This is a special case of \Intro{\Diamond}\Ax{K} with $\Box!B$ missing.
Then, define \emph{necessary} consequence as follows:
\begin{quote}
\emph{Necessary consequence}. $!C$ is a necessary consequence of $!A$
iff $!A\lif!C$ is necessary. (Equivalently, iff $!C$ is a logical
consequence of $!A$ \emph{together with} what is necessary.)
\end{quote}
This idea would be captured by the following rule:
\begin{prooftree}
    \AxiomC{}\DeduceC{$\Diamond!A$}
    \AxiomC{}\DeduceC{$\Box(!A\lif!C)$}
    \BinaryInfC{$\Diamond!C$}
\end{prooftree}
Which is equivalent to $\Diamond$\Ax{K} given the $\Box$\Ax{K}
rule. Our $\Diamond$\Ax{K} is more elegant than the rule above
because it does not involve a specific propositional operator ($\lif$). 

Why `single premise' consequence? While we accept the idea that
logical consequences of \emph{multiple} necessities are necessary, we
reject the idea that logical consequences of \emph{multiple}
possibilities are possible. For instance, if $!A$ is contingent it is
possible that $!A$ and also possible that $\lnot!A$. Yet their
conjunction, $!A\land\lnot!A$, is still impossible. Therefore even
though $!A$ and $\lnot!A$ \emph{jointly} logically entail
$!A\land\lnot!A$, we do not want to infer that $\Diamond!A$ and
$\Diamond\lnot!A$ jointly entail $\Diamond(!A\land\lnot!A)$. \emph{A
fortiori}, we reject the principle according to which necessary joint
consequences of \emph{multiple} possibilities are themselves
possibile. We only accept the principle that necessary consequences of
\emph{one} possibility are possible.
\end{explain}

\iftag{prvDiamond}{
    \subsubsection{Rule $\Diamond\lfalse$}

The following rules are also derived:

\begin{defish}
\AxiomC{}
\DeduceC{$\Diamond\lfalse$}
\RightLabel{$\Diamond\lfalse$}
\UnaryInfC{$\lfalse$}
\DisplayProof
\hfill
\AxiomC{}
\DeduceC{$\Diamond\lfalse$}
\RightLabel{$\Diamond\lfalse$}
\UnaryInfC{$!A$}
\DisplayProof
\end{defish}

}{}

\end{document}
