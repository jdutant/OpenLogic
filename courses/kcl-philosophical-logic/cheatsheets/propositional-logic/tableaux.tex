
\section{Tableaux}

\subsection{Rules for $\lnot$}

\begin{defish}
\AxiomC{\sFmla{\True}{\lnot !A}}
\RightLabel{\TRule{\True}{\lnot}}
\UnaryInfC{\sFmla{\False}{!A}}
\DisplayProof
\hfill
\AxiomC{\sFmla{\False}{\lnot !A}}
\RightLabel{\TRule{\False}{\lnot}}
\UnaryInfC{\sFmla{\True}{!A}}
\DisplayProof
\end{defish}

\subsection{Rules for $\land$}

\begin{defish}\noindent
\AxiomC{\sFmla{\True}{!A \land !B}}
\RightLabel{\TRule{\True}{\land}}
\UnaryInfC{\sFmla{\True}{!A}}
\noLine
\UnaryInfC{\sFmla{\True}{!B}}
\DisplayProof
\hfill
\AxiomC{\sFmla{\False}{!A \land !B}}
\RightLabel{\TRule{\False}{\land}}
\UnaryInfC{$\sFmla{\False}{!A} \quad \mid \quad \sFmla{\False}{!B}$}
\DisplayProof
\end{defish}

\subsection{Rules for $\lor$}

\begin{defish}
\AxiomC{\sFmla{\True}{!A \lor !B}}
\RightLabel{\TRule{\True}{\lor}}
\UnaryInfC{$\sFmla{\True}{!A} \quad \mid \quad \sFmla{\True}{!B}$}
\DisplayProof
\hfill
\AxiomC{\sFmla{\False}{!A \lor !B}}
\RightLabel{\TRule{\False}{\lor}}
\UnaryInfC{\sFmla{\False}{!A}}
\noLine
\UnaryInfC{\sFmla{\False}{!B}}
\DisplayProof
\end{defish}

\subsection{Rules for $\lif$}

\begin{defish}
\AxiomC{\sFmla{\True}{!A \lif !B}}
\RightLabel{\TRule{\True}{\lif}}
\UnaryInfC{$\sFmla{\False}{!A} \quad \mid \quad \sFmla{\True}{!B}$}
\DisplayProof
\hfill
\AxiomC{\sFmla{\False}{!A \lif !B}}
\RightLabel{\TRule{\False}{\lif}}
\UnaryInfC{\sFmla{\True}{!A}}
\noLine
\UnaryInfC{\sFmla{\False}{!B}}
\DisplayProof
\end{defish}

\subsection{The Cut Rule}

\begin{defish}
\AxiomC{}
\RightLabel{\Cut}
\UnaryInfC{$\sFmla{\True}{!A} \quad \mid \quad \sFmla{\False}{!A}$}
\DisplayProof
\end{defish}

Note: the Cut Rule is not to be used: a tableau closes with the rule iff it closes without it. Its only purpose is to simplify proofs of soundness and completeness (by allowing us to combine tableaus).
