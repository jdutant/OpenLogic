% Part: formatives
% Chapter: two
% Section: quantified-modal-logic

\documentclass[../../../include/open-logic-section]{subfiles}

\begin{document}

\olfileid{for}{two}{qmod}

\newpage
\olsection{Quantified Modal Logic and Free Logic}

\emph{Notation}. \Log{SQML} (`Simple Quantified Modal Logic', Sider) 
natural deduction proofs can use any rules of first order logic plus 
modal logic $\Log{K}$. \Log{SQML} models have a \emph{fixed} domain of 
objects.

\begin{prob}
    Answer the following.
    \begin{enumerate}
    \item 
    Formalize the following in the language of Quantified Modal Logic 
    using Russell's theory of definite descriptions. Show that the 
    sentence is ambiguous by giving \emph{two} formalizations. 
    \begin{quote}
        The red crab is necessarily red.
    \end{quote}
    

    \item
    One of these formalizations is a logical truth: give a natural 
    deduction proof of it in \Log{SQML}. 
    \end{enumerate}

\end{prob}

\begin{prob}
    Give a proof of the following in \Log{SQML}:
    $$\vdash_{\Log{SQML}}\Box\lexists[x][\eq[x][a]]$$
    
    Should we reject this consequence of \Log{SQML}? If so, how can
    we avoid it? If not, why not? Briefly justify your answer (<500 words). 

\end{prob}

\begin{prob}
    Say whether the arguments below are valid in \emph{negative} free logic.
    If yes, give a natural deduction proof. Otherwise, 
    give a counterexample.
    \begin{enumerate}
        \item $\lforall[x][\Obj{G}xb] \vdash \Obj{G}ab$
        \item $\lforall[x][\Obj{G}xb] \vdash \lexists[y]\lexists[x][\Obj{G}xy]$
    \end{enumerate}
\end{prob}

\end{document}