% Part: first-order-logic
% Chapter: syntax-and-semantics
% Section: negative-free-logic

\documentclass[../../../include/open-logic-section]{subfiles}

\begin{document}

\olfileid{frl}{syn}{nfl}

\olsection{Negative Free Logic}

The guiding idea of negative free logic is that atomic claims have
existence committments. The following inference seems correct:

\begin{enumerate}
  \item René thinks.
  \item Therefore, René exists.
\end{enumerate}

This suggests that the claim that René thinks does require that \emph{there is 
something}, René, which is said to think.  Accordingly, negative free logic 
considers that (where $F$ is any predicate) $\Atom{F}{a}$ holds only 
if $a$ is not empty, hence only if $\lfrexists a$ holds. 

\begin{explain}
The corresponding !!{structure}s are the most straightforward way to 
modify those of first-order logic to remove their existence assumptions.
We allow interpretations that do not assign any object to some !!{constant}s,
and (for an inclusive negative free logic) we allow the domain to be empty.
\end{explain}

\begin{defn}[!!^{structure}s]
In the negative system of free logic, $\Log NFL$,
\article{structure} \emph{!!{structure}}~$\Struct M$ for a language
$\Lang L_{\Log{FL}}$ of free logic consists of:
\begin{enumerate}
\item \emph{Domain:} a set, $\Domain M$,
\item \emph{Interpretation of !!{constant}s:} for any !!{constant}~$c$ of
  $\Lang L_{\Log{FL}}$, its interpretation $\Assign{c}{M}$ is !!a{element} of 
  $\Domain M$ or undefined,
\item \emph{Interpretation of !!{predicate}s:} for each $n$-place
  !!{predicate}~$R$ of $\Lang L_{\Log{FL}}$ (other than $\eq$), an $n$-place
  relation $\Assign{R}{M} \subseteq \Domain{M}^n$
\tagitem{fnTerms}{\item \emph{Interpretation of !!{function}s:} for each $n$-place
  !!{function}~$f$ of $\Lang L_{\Log{FL}}$, an $n$-place function $\Assign{f}{M}
  \colon \Domain{M}^n \to \Domain{M}$}{}
\end{enumerate}
The system of negative free logic is \emph{inclusive} if $\Domain M$ is
allowed to be empty. There is only one such !!{structure}, for interpretations
can only be defined one way when the domain is empty. We call it the 
\emph{empty !!{structure}}.
\end{defn}

\begin{prob}[The empty !!{structure}]
Describe the interpretation of !!{constant}s and predicates in 
an !!{structure} whose domain is the empty set.
\end{prob}

!!^{variable} assignements are defined as before, except that we allow
partial assignments that fail to assign anything to some (or even all)
!!{variable}s. 

\begin{defn}[Variable Assignment]
A \emph{variable assignment}~$s$ for a !!{structure}~$\Struct{M}$ is a
(partial) function such that for any !!{variable}~$x$ of
$\Lang L_{\Log{FL}}$, $s(x) \in \Domain M$ if defined. In addition, we stipulate
an \emph{empty variable assignement} $s^{*}$ that doesn't assign any value to
any variable.
\end{defn}

\begin{defn}[!!^{value} of Terms]
If $t$ is a term of the language~$\Lang L_{\Log{FL}}$, $\Struct M$ is a
!!{structure} for~$\Lang L_{\Log{FL}}$, and $s$ is a !!{variable} assignment
for~$\Struct M$, the \emph{!!{value}}~$\Value{t}{M}[s]$ is defined as
follows:
\begin{enumerate}
\item \indcase{t}{c}{$\Value{\indfrm}{M}[s] = \Assign{\indcomplex}{M}$, if defined.}
\item \indcase{t}{x}{$\Value{\indfrm}{M}[s] = s(\indcomplex)$, if defined.}
\tagitem{fnTerms}{\indcase{t}{\Atom{f}{t_1, \ldots, t_n}}{
\[
\Value{\indfrm}{M}[s] = \Assign{f}{M}(\Value{t_1}{M}[s], \ldots,
\Value{t_n}{M}[s])\textrm{, if defined}.
\]}
}{}
\end{enumerate}
\end{defn}

\begin{defn}[$x$-Variant]
If $s$ is a !!{variable} assignment for a !!{structure}~$\Struct M$, we
call !!a{variable} assignment $s'$ an \emph{$x$-variant} of $s$ iff:
\begin{enumerate}
\item it differs from $s$ at most in what it assigns for $s$, that is 
either $\Value{y}{M}[s'] = \Value{y}{M}[s]$, or both are undefined, for
any variable $y$ other than $x$.
\item it is defined for $x$, that is, $\Value{y}{M}[s'] \in \Domain M$.
We write $\varAssign{s'}{s}{x}$ when $s'$ is an $x$-variant of $s$.
\end{enumerate}
\end{defn}

Note that an $x$-variant of some !!{variable} assignment $s$ must be 
defined for $x$, that is, it must assign an existing object to $x$. 
If, for instance, $s$ is undefined for $x$, then $s$ is not an $x$-variant
of itself. Only those variants of $s$ who do assign an object to $x$ 
are $x$-variants of $s$. 

\begin{explain}
The only reason to let !!{variable} assignments go undefined for some
!!{variable}s is to cater  for the empty !!{structure}. We need to have 
!!{variable} assignments in  every !!{structure} because we define 
truth in a !!{structure} in terms of truth relative to an !!{variable} assignement. In the 
empty !!{structure} there are no !!{element}s of the domain to assign 
to !!{variable}. So the only way to have a  !!{variable} assignment 
in the empty !!{structure} is to allow assignments
that are not defined for all (or any) !!{variable}.
\end{explain}

\begin{defn}[Satisfaction]
\ollabel{defn:satisfaction}
Satisfaction of a !!{formula}~$!A$ in a !!{structure}~$\Struct M$
relative to a !!{variable} assignment~$s$, in symbols:
$\Sat{M}{!A}[s]$, is defined recursively as follows. (We write
$\Sat/{M}{!A}[s]$ to mean ``not $\Sat{M}{!A}[s]$.'')
\begin{enumerate}
\tagitem{prvFalse}{%
  \indcase{!A}{\lfalse}{$\Sat/{M}{\indfrm}[s]$.}}{}

\tagitem{prvTrue}{%
  \indcase{!A}{\ltrue}{$\Sat{M}{\indfrm}[s]$.}}{}

\item \indcase{!A}{\Atom{R}{t_1, \dots, t_n}}{$\Sat{M}{\indfrm}[s]$
  iff $\Value{t_i}{M}[s]$ is defined for any $1\leq i \leq n$ and 
  $\langle \Value{t_1}{M}[s], \dots, \Value{t_n}{M}[s] \rangle \in
  \Assign{R}{M}$.}

\item \indcase{!A}{\Atom{\lfrexists}{t}}{$\Sat{M}{\indfrm}[s]$ iff 
  $\Value{t}{M}[s]$ is defined.}

\item \indcase{!A}{\eq[t_1][t_2]}{$\Sat{M}{\indfrm}[s]$ iff 
  $\Value{t_1}{M}[s], \Value{t_2}{M}[s]$ are both defined and
  $\Value{t_1}{M}[s] = \Value{t_2}{M}[s]$.}

\tagitem{prvNot}{%
  \indcase{!A}{\lnot !B}{$\Sat{M}{\indfrm}[s]$ iff
    $\Sat/{M}{!B}[s]$.}}{}

\tagitem{prvAnd}{%
  \indcase{!A}{(!B \land !C)}{$\Sat{M}{\indfrm}[s]$ iff $\Sat{M}{!B}[s]$
    and $\Sat{M}{!C}[s]$.}}{}

\tagitem{prvOr}{%
  \indcase{!A}{(!B \lor !C)}{$\Sat{M}{\indfrm}[s]$ iff
    $\Sat{M}{!A}[s]$ or $\Sat{M}{!B}[s]$ (or both).}}{}

\tagitem{prvIf}{%
  \indcase{!A}{(!B \lif !C)}{$\Sat{M}{\indfrm}[s]$ iff $\Sat/{M}{!B}[s]$
    or $\Sat{M}{!C}[s]$ (or both).}}{}

\tagitem{prvIff}{%
  \indcase{!A}{(!B \liff !C)}{$\Sat{M}{\indfrm}[s]$ iff either both
    $\Sat{M}{!B}[s]$ and $\Sat{M}{!C}[s]$, or neither $\Sat{M}{!B}[s]$
    nor $\Sat{M}{!C}[s]$.}}{}

\tagitem{prvAll}{%
  \indcase{!A}{\lforall[x][!B]}{$\Sat{M}{\indfrm}[s]$ iff for every
    $x$-variant~$s'$ of $s$ \emph{for which $s'(x)$ is defined}, $\Sat{M}{!B}[s']$.}}{}

\tagitem{prvEx}{%
  \indcase{!A}{\lexists[x][!B]}{$\Sat{M}{\indfrm}[s]$ iff there is an
    $x$-variant~$s'$ of $s$ \emph{for which $s'(x)$ is defined} such that $\Sat{M}{!B}[s']$.}}{}
\end{enumerate}
\end{defn}

Truth and validity are defined as in first-order logic.

\begin{defn}[Truth]
Where $!A$ is a closed !!{formula}, we say that $A$ is \emph{true in}, or
\emph{satisfied by} a !!{structure} $\Struct M$, written $\Sat{M}{!A}$ iff 
$A$ is satisfied relative to any assignement in $\Struct M$:
$$\Sat{M}{!A}\textrm{ iff }\Sat{M}{!A}[s]\textrm{ for any }s\textrm{ in }\Struct M$$
\end{defn}

\begin{defn}[Validity]
  A !!{formula} $!A$ is \emph{valid}, $\Entails !A$, iff $\Sat{M}{!A}$ for every
  !!{structure}~$\Struct M$.
\end{defn}

\begin{prop}\ollabel{prop:if-sat-Fa-sat-exists}
  If $\Sat{M}{Fa}$ then $\Sat{M}{\lfrexists a}$.
\end{prop}

\begin{proof}
  Exercise.
\end{proof}

\begin{prob}
  Prove \olref[frl][syn][nfl]{prop:if-sat-Fa-sat-exists}
\end{prob}

\begin{prop}\ollabel{prop:if-sat-id-sat-exists}
  If $\Sat{M}{\eq[a][a]}$ then $\Sat{M}{\lfrexists a}$.
\end{prop}

\begin{proof}
  Exercise.
\end{proof}

\begin{prob}
  Prove \olref[frl][syn][nfl]{prop:if-sat-id-sat-exists}
\end{prob}

\begin{prop}
  $\Entails/ \lexists x (x = x)$.
\end{prop}

\begin{prob}
Show that $\Entails/ \lexists x (x = x)$.
\end{prob}

\end{document}
