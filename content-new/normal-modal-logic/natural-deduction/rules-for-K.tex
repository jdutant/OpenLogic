% Part: normal-modal-logic
% Chapter: natural-deduction 
% Section: rules-for-K

\documentclass[../../../include/open-logic-section]{subfiles}

\begin{document}

\olfileid{nml}{nd}{rul}

\olsection{Rules for \Log{K}}

For the modal operator $\Box$, we add the \emph{Normality} rule 
$\Box$\Ax{K}:
\begin{defish}
\AxiomC{}\DeduceC{$\Box!A$}
\AxiomC{}\DeduceC{$\Box!B$}
\AxiomC{$\Discharge{!A}{n},\Discharge{!B}{n}$ \emph{at
most}}
\DeduceC{$!C$} 
\DischargeRule{$\Box$\Ax{K}}{n}
\TrinaryInfC{$\Box!C$} 
\DisplayProof
\end{defish}

The rule plays both the role of an introduction rule (for $\Box!C$)
and of an elimination rule (for $\Box!A$ and $\Box!B$). We thus call
it $\Box$\Ax{K} rather than \Intro{\Box}\Ax{K} or \Elim{\Box}\Ax{K}. 

At the step when $\Box$\Ax{K} is applied, the sub-!!{derivation} on 
the right must have at most assumptions $!A,!B$ left undischarged. 
It may have other assumptions, but they must be discharged by the 
time we reach this step.

The rule can be applied with just one premise $\Box!A$ 
or even no premise:

\bigskip
\begin{tabular}{cc}
    \AxiomC{}\DeduceC{$\Box!A$}
    \AxiomC{$\Discharge{!A}{n}$ \emph{at
    most}}
    \DeduceC{$!C$} 
    \DischargeRule{$\Box$\Ax{K}}{n}
    \BinaryInfC{$\Box!C$} 
    \DisplayProof
&
    \AxiomC{\emph{no undischarged assumption}}
    \DeduceC{$!C$}
    \RightLabel{$\Box$\Ax{K}}
    \UnaryInfC{$\Box!C$} 
    \DisplayProof
\end{tabular}
\bigskip

\iftag{prvDiamond}{
For the modal operator $\Diamond$ we add the \emph{Duality} rules
below. These amount to defining $\Diamond!A$ as $\lnot\Box\lnot!A$.
\begin{defish}
        \AxiomC{}\DeduceC{$\lnot\Box\lnot!A$}
        \RightLabel{\Intro{\Diamond}K}
        \UnaryInfC{$\Diamond!A$}
        \DisplayProof
    \hfill
        \AxiomC{}\DeduceC{$\Diamond!A$}
        \RightLabel{\Elim{\Diamond}K}
        \UnaryInfC{$\lnot\Box\lnot!A$}
        \DisplayProof
\end{defish}
}{}

\begin{explain}
The $\Box$\Ax{K} rule is called \emph{Normality} because it
characterizes how necessity behaves in so-called \emph{normal modal
logics}. The core idea of normal modal logics is that necessity is
\emph{closed under multi-premise logical consequence}, namely, 
that if some premises logically entail a conclusion, then the 
necessity of those premises entail the necessity of that conclusion:

\begin{quote}
\emph{Normality.} If $\Gamma\Entails!A$ then
$\Box\Gamma\Entails\Box\!A$. (Where $\Box\Gamma$ is the result of
prefixing each !!{formula} in $\Gamma$ with a $\Box$ operator, i.e.
the set $\Setabs{\Box!B}{!B\in\Gamma}$.)
\end{quote} 

Our $\Box$\Ax{K} rule captures that idea. The sub-!!{derivation} on
the right establishes that $!A,!B\Entails!C$. It is crucial that it
has \emph{no undischarged assumptions other than} $!A,!B$. Otherwise
it would not establish that $!C$ is a logical consequence of $!A,!B$
\emph{alone}. Once $!A,!B\Entails!C$ is established, according to
Normality $\Box!C$ is a logical consequence of $\Box!A$, $\Box!B$.
Therefore we can discharge the assumptions $!A,!B$ on the right, and
turn the two sub-!!{derivation}s of $\Box!A$, $\Box!B$ on the left
into a !!{derivation} of $\Box!C$. Note that the sub-!!{derivation}s
of $\Box!A$, $\Box!B$ may rest on undischarged assumptions; we then
have a !!{derivation} of $\Box!C$ from these assumptions.

The $\Box$\Ax{K} rule directly captures the idea that logical
consequences of \emph{up to two} necessities are necessary. But
Normality states that logical consequences of \emph{any number} of
necessities are necessary. Is our rule enough? Yes, because repeated
applications of $\Box$\Ax{K} ensure that logical consequences of
\emph{any finite number} of necessities are necessary. Could it happen
that $!A$ is the logical consequence of an \emph{infinite} number of
necessities, without being a logical consequence of any finite subset
of those? If that was so, logical consequence would not be compact.
When logical consequence is not compact, a complete !!{derivation}
system cannot be given; our rule would still be the best we can
achieve. As it happens, logical consequence in \Ax{K} models is
compact, and the rule is sufficient for completeness.

\iftag{prvDiamond}{ The $\Diamond$ rules capture the idea that $\Diamond$ is
the \emph{dual} of $\Box$: the negation of a
necessity is is the possibility of its negation ($\lnot\Box$ is
$\Diamond\lnot$) and the negation of a possibility is the necessity of
its negation ($\lnot\Diamond$ is $\Box\lnot$). 
}{}

\end{document}
