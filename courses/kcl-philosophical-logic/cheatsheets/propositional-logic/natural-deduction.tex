% Part: cheat-sheets
% Chapter: propositional-logic
% Section: natural deduction

\documentclass[../../../../include/open-logic-section]{subfiles}

\begin{document}

\olfileid{chs}{pl}{nd}

\newpage % for 'all the rules at once'
\olsection{Natural Deduction}

\begin{figure}
{\setlength\extrarowheight{2em} 
\begin{tabular}{llll}
    \AxiomC{$!A$}\DisplayProof (\emph{Assumption})
    & \AxiomC{}\DeduceC{$!A$}
    \AxiomC{}\DeduceC{$\lnot!A$}
    \RightLabel{\Intro{\bot}} \BinaryInfC{$\bot$}\DisplayProof
    & \AxiomC{$\Discharge{\mathcolor{gray}{!A}}{n}$}\DeduceC{$\bot$}
    \DischargeRule{\Elim{\bot}}{n} \UnaryInfC{$\lnot!A$}\DisplayProof 
    & \AxiomC{$\Discharge{\mathcolor{gray}{\lnot!A}}{n}$}\DeduceC{$\bot$}
    \DischargeRule{$\bot_C$}{n} \UnaryInfC{$!A$}\DisplayProof
    \\
    \AxiomC{}\DeduceC{$!A$}
    \AxiomC{}\DeduceC{$!B$}
    \RightLabel{\Intro{\land}}
    \BinaryInfC{$!A\land!B$}\DisplayProof
    & \AxiomC{}\DeduceC{$!A\land!B$}
    \RightLabel{\Elim{\land}}
    \UnaryInfC{$!A$}\DisplayProof
    & \AxiomC{}\DeduceC{$!A\land!B$}
    \RightLabel{\Elim{\land}}
    \UnaryInfC{$!B$}\DisplayProof
    \\
    \AxiomC{}\DeduceC{$!A$}
    \RightLabel{\Intro{\lor}}
    \UnaryInfC{$!B\lor!A$}\DisplayProof
    & \AxiomC{}\DeduceC{$!B$}
    \RightLabel{\Intro{\lor}}
    \UnaryInfC{$!B\lor!A$}\DisplayProof
    & \multicolumn{2}{l}{
    \AxiomC{}\DeduceC{$!A\lor!B$}
    \AxiomC{$\Discharge{\mathcolor{gray}{!A}}{n}$}\DeduceC{$!C$}
    \AxiomC{$\Discharge{\mathcolor{gray}{!B}}{n}$}\DeduceC{$!C$}
    \DischargeRule{\Elim{\lor}}{n}
    \TrinaryInfC{$!C$}\DisplayProof
    }
    \\
    \AxiomC{$\Discharge{\mathcolor{gray}{!A}}{n}$}\DeduceC{$!B$}
    \DischargeRule{\Intro{\lif}}{n}
    \UnaryInfC{$!A\lif!B$}\DisplayProof
    & \AxiomC{}\DeduceC{$!A$}
    \AxiomC{}\DeduceC{$!A\lif!B$}
    \RightLabel{\Elim{\lif}}
    \BinaryInfC{$!B$}\DisplayProof
\end{tabular}
\\[1em] \noindent
\begin{tabular}{lll}
    \AxiomC{$\Discharge{\mathcolor{gray}{!A}}{n}$}\DeduceC{$!B$}
    \AxiomC{$\Discharge{\mathcolor{gray}{!B}}{n}$}\DeduceC{$!A$}
    \DischargeRule{\Intro{\liff}}{n}
    \BinaryInfC{$!A\liff!B$}\DisplayProof
    & \AxiomC{}\DeduceC{$!A$}
    \AxiomC{}\DeduceC{$!A\liff!B$}
    \RightLabel{\Elim{\liff}}
    \BinaryInfC{$!B$}\DisplayProof
    & \AxiomC{}\DeduceC{$!B$}
    \AxiomC{}\DeduceC{$!A\liff!B$}
    \RightLabel{\Elim{\liff}}
    \BinaryInfC{$!A$}\DisplayProof
    \\
\end{tabular}
\\[1em] \noindent
\emph{You may also use the derived rules:}
\\[1em] \noindent
\begin{tabular}{llll}
    \AxiomC{$\Discharge{\mathcolor{gray}{!A}}{n}$}\DeduceC{$!B$}
    \AxiomC{$\Discharge{\mathcolor{gray}{!A}}{n}$}\DeduceC{$\lnot!B$}
    \DischargeRule{\Intro{\lnot}}{n} \BinaryInfC{$\lnot!A$}\DisplayProof
    & \AxiomC{$\Discharge{\mathcolor{gray}{\lnot!A}}{n}$}\DeduceC{$!B$}
    \AxiomC{$\Discharge{\mathcolor{gray}{\lnot!A}}{n}$}\DeduceC{$\lnot!B$}
    \DischargeRule{\Elim{\lnot}}{n}
    \BinaryInfC{$!A$}\DisplayProof
    \\
    \AxiomC{}\DeduceC{$\bot$}
    \RightLabel{$\bot_I$ aka EFQ}
    \UnaryInfC{$!B$}\DisplayProof
    & \AxiomC{}\DeduceC{$!A$}
    \AxiomC{}\DeduceC{$\lnot!A$}
    \RightLabel{EFQ}
    \BinaryInfC{$!B$}\DisplayProof
    & \AxiomC{}\DeduceC{$\lnot\lnot!A$}
    \RightLabel{DNE}
    \UnaryInfC{$!A$}\DisplayProof
\end{tabular}
}
\caption{Natural Deduction Rules for Propositional Logic.}
\label{NatDedPropRules}
\end{figure}

An overview of the rules is given in Figure \ref{NatDedPropRules}.
We describe them in detail below, using the following notation:

\bigskip \noindent
{\renewcommand{\arraystretch}{1.2}
\begin{tabular}{cp{.75\textwidth}}

    $!A,!B,!C$ 
    
    & stand for any formula.
   
    \\

    \AxiomC{}\DeduceC{$!A$}\DisplayProof

    & \begin{minipage}[c]{.75\textwidth}
        stands for a proof whose conclusion is $!A$. 
    
    The proof may consist in just one line ($!A$, see the
    assumption rule) or more, and it may have undischarged
    assumptions.        
    \end{minipage}

    \\

    \AxiomC{$\mathcolor{gray}{!A}$}\DeduceC{$!B$}\DisplayProof
    \AxiomC{$[\mathcolor{gray}{!A}]$}\DeduceC{$!B$}\DisplayProof

    & \begin{minipage}[c]{.75\textwidth}
    The first stands for a proof whose conclusion is $!B$ and
    whose assumptions \emph{may, but need not} include $!A$. The
    second stands for the same proof, but with the assumption of
    $!A$ discharged \emph{if present}. If $!A$ was not
    present, it is the same proof with nothing discharged
    \end{minipage}
    
\end{tabular}
}
\bigskip

\noindent 
Exam conventions:

\begin{itemize}
\item \emph{Is it mandatory to discharge?} No. Some systems (e.g. Hablach)
require you to discharge when you can; others (e.g. King's
Philosophical Logic) don't. The rules below are formulated with 
mandatory discharge (`... you \emph{must} discharge ...'), which I 
recommend as it makes your proofs cleaner. But either system is
acceptable in exam, so you don't have to discharge every assumption 
that can be discharged.

\item  \emph{Is it mandatory to number the discharging steps?} No. 
Below, rules with discharge include a number associating discharged
assumptions with the step at which they are discharged. It makes 
proofs more readable but is not required in exam.

\end{itemize}

\subsection*{Assumption Rule} 

\begin{tabular}{l}
     For any formula $!A$,\\
     \AxiomC{$!A$}\DisplayProof\\
     is a proof.
\end{tabular}

\subsection*{Connective Rules}

I give two sets of rules for negation, plus some derived rules. Make
sure you know at least the set of Negation with Falsum: it's enough
to do any proofs. If you learn the others, you can freely use them
in exam.

\subsubsection*{Negation with Falsum}

KCL Philosophical Logic's rules with Negation and Falsum. I have
relabelled them \Intro{\bot} (instead of \Elim{\lnot}) and \Elim{\bot}
(instead of \Intro{\lnot}) to avoid confusion with the rules without
Falsum given further down. In exam labels are optional and any correct
label is fine.

\begin{defish}
\noindent \textbf{Falsum Introduction}
\nopagebreak \smallskip \\ \noindent
\begin{tabular}{ll@{\hskip 3em}l}
    \multicolumn{2}{l}{Given proofs:} & this is a proof:\\
    \AxiomC{}\DeduceC{$!A$}\DisplayProof &
    \AxiomC{}\DeduceC{$\lnot!A$}\DisplayProof &
    \AxiomC{}\DeduceC{$!A$}
    \AxiomC{}\DeduceC{$\lnot!A$}
    \RightLabel{\Intro{\bot}} \BinaryInfC{$\bot$}\DisplayProof
\end{tabular}    
\end{defish}

\begin{defish}
\noindent \textbf{Reductio Ad Absurdum (aka Reductio)}
\nopagebreak \smallskip \\ \noindent
\begin{tabular}{l@{\hskip 4em}lp{.375\textwidth}}
    Given proof: & this is a proof:\\
    \AxiomC{$\mathcolor{gray}{!A}$}\DeduceC{$\bot$}\DisplayProof &
    \AxiomC{$\Discharge{\mathcolor{gray}{!A}}{n}$}\DeduceC{$\bot$}
    \DischargeRule{\Elim{\bot}}{n} \UnaryInfC{$\lnot!A$}\DisplayProof
    & \emph{The assumption $!A$ need not be present in either subproof.
    If present, it must be discharged.}
\end{tabular}
\end{defish}

\begin{defish}
\noindent \textbf{Indirect Proof}
\nopagebreak \smallskip \\ \noindent
\begin{tabular}{l@{\hskip 4em}lp{.375\textwidth}}
    Given proof: & this is a proof:\\
    \AxiomC{$\mathcolor{gray}{\lnot!A}$}\DeduceC{$\bot$}\DisplayProof &
    \AxiomC{$\Discharge{\mathcolor{gray}{\lnot!A}}{n}$}\DeduceC{$\bot$}
    \DischargeRule{$\bot_C$}{n} \UnaryInfC{$!A$}\DisplayProof
    & \emph{The assumption $\lnot!A$ need not be present in either subproof.
    If present, it must be discharged.}
\end{tabular}
\end{defish}

%% NOTE
$\bot_C$ stands for `Classical Falsum rule'. In the notes a fourth rule
rule is given, $\bot_I$ (for `Intuitionistic Falsum rule'). We don't add
it here because it's merely a special case of $\bot_C$ where the
assumption $\lnot!A$ isn't present and, hence, nothing is discharged.
$\bot_I$ is also called Ex Falso Quodlibet or Explosion; see below 
for details.
%% END OF NOTE

\subsubsection*{Conjunction}

\begin{defish}
\noindent \textbf{Conjunction Introduction}
\nopagebreak \smallskip \\ \noindent
\begin{tabular}{ll@{\hskip 4em}l}
    \multicolumn{2}{l}{Given proofs:} & this is a proof:\\
    \AxiomC{}\DeduceC{$!A$}\DisplayProof 
  & \AxiomC{}\DeduceC{$!B$}\DisplayProof 
  & \AxiomC{}\DeduceC{$!A$}
  \AxiomC{}\DeduceC{$!B$}
  \RightLabel{\Intro{\land}}
  \BinaryInfC{$!A\land!B$}\DisplayProof
\end{tabular}
\end{defish}

\begin{defish}
\noindent \textbf{Conjunction Elimination}
\nopagebreak \smallskip \\ \noindent
\begin{tabular}{l@{\hskip 3em}l@{\hskip 2em}l}
    Given proof: & this is a proof: & and this too:\\
    \AxiomC{}\DeduceC{$!A\land!B$}\DisplayProof
    &  \AxiomC{}\DeduceC{$!A\land!B$}
       \RightLabel{\Elim{\land}}
       \UnaryInfC{$!A$}\DisplayProof
    &  \AxiomC{}\DeduceC{$!A\land!B$}
       \RightLabel{\Elim{\land}}
       \UnaryInfC{$!B$}\DisplayProof
\end{tabular}
\end{defish}

\subsubsection*{Disjunction}

\begin{defish}
\noindent \textbf{Disjunction Introduction}
\nopagebreak \smallskip \\ \noindent    
\begin{tabular}{l@{\hskip 4em}l@{\hskip 2em}l}
    Given proof: & this is a proof: & and this too:\\
    \AxiomC{}\DeduceC{$!A$}\DisplayProof
    &  \AxiomC{}\DeduceC{$!A$}
        \RightLabel{\Intro{\lor}}
        \UnaryInfC{$!A\lor!B$}\DisplayProof
     &  \AxiomC{}\DeduceC{$!A$}
        \RightLabel{\Intro{\lor}}
        \UnaryInfC{$!B\lor!A$}\DisplayProof
\end{tabular}
\end{defish}

\begin{defish}
\noindent \textbf{Disjunction Elimination}
\nopagebreak \smallskip \\ \noindent    
\begin{tabular}{lll@{\hskip 3em}l}
    \multicolumn{3}{l}{Given proofs:} & this is a proof:\\
    \AxiomC{}\DeduceC{$!A\lor!B$}\DisplayProof 
  & \AxiomC{$\mathcolor{gray}{!A}$}\DeduceC{$!C$}\DisplayProof
  & \AxiomC{$\mathcolor{gray}{!B}$}\DeduceC{$!C$}\DisplayProof
  & \AxiomC{}\DeduceC{$!A\lor!B$}
        \AxiomC{$\Discharge{\mathcolor{gray}{!A}}{n}$}\DeduceC{$!C$}
            \AxiomC{$\Discharge{\mathcolor{gray}{!B}}{n}$}\DeduceC{$!C$}
        \DischargeRule{\Elim{\lor}}{n}
        \TrinaryInfC{$!C$}\DisplayProof \\
\end{tabular}
\bigskip

\noindent \emph{The assumption $!A$ need not be present in the second
subproof. If present, it must be discharged. Same for the assumption
$!B$ in the third subproof.}
\end{defish}

\subsubsection*{Material implication}

\begin{defish}
\noindent \textbf{Conditional Proof}
\nopagebreak \smallskip \\ \noindent        
\begin{tabular}{l@{\hskip 4em}lp{.375\textwidth}}
    Given proof: & this is a proof:\\
    \AxiomC{$\mathcolor{gray}{!A}$}\DeduceC{$!B$}\DisplayProof 
  & \AxiomC{$\Discharge{\mathcolor{gray}{!A}}{n}$}\DeduceC{$!B$}
    \DischargeRule{\Intro{\lif}}{n}
    \UnaryInfC{$!A\lif!B$}\DisplayProof
        & \emph{The assumption $!A$ need not be present. If present, it must be discharged.}
\end{tabular}
\end{defish}

\begin{defish}
\noindent \textbf{Modus Ponens}
\nopagebreak \smallskip \\ \noindent        
\begin{tabular}{ll@{\hskip 2em}lp{.375\textwidth}}
    \multicolumn{2}{l}{Given proofs:} & this is a proof:\\
    \AxiomC{}\DeduceC{$!A$}\DisplayProof 
  & \AxiomC{}\DeduceC{$!A\lif!B$}\DisplayProof
  & \AxiomC{}\DeduceC{$!A$}
            \AxiomC{}\DeduceC{$!A\lif!B$}
        \RightLabel{\Elim{\lif}}
        \BinaryInfC{$!B$}\DisplayProof
    & \emph{When that is more convenient, you may place instead the $!A$ subproof on the right and the $!A\lif!B$ subproof on the left.}
\end{tabular}
\end{defish}

\subsubsection*{Material equivalence}

\begin{defish}
\noindent \textbf{Material Equivalence Introduction}
\nopagebreak \smallskip \\ \noindent
\begin{tabular}{ll@{\hskip 2em}lp{.375\textwidth}}
    \multicolumn{2}{l}{Given proofs:} & this is a proof:\\
    \AxiomC{$\mathcolor{gray}{!A}$}\DeduceC{$!B$}\DisplayProof 
  & \AxiomC{$\mathcolor{gray}{!B}$}\DeduceC{$!A$}\DisplayProof 
  & \AxiomC{$\Discharge{\mathcolor{gray}{!A}}{n}$}\DeduceC{$!B$}
    \AxiomC{$\Discharge{\mathcolor{gray}{!B}}{n}$}\DeduceC{$!A$}
    \DischargeRule{\Intro{\liff}}{n}
    \BinaryInfC{$!A\liff!B$}\DisplayProof 
    & \emph{As with other rules, the subproof assumptions need not be present, and are discharged if present.}
\end{tabular}
\end{defish}    

\begin{defish}
\noindent \textbf{Material Equivalence Elimination}
\nopagebreak \smallskip \\ \noindent
\begin{tabular}{ll@{\hskip 2em}lp{.375\textwidth}}
    \multicolumn{2}{l}{Given proofs:} & this is a proof:\\
    \AxiomC{}\DeduceC{$!A$}\DisplayProof
  & \AxiomC{}\DeduceC{$!A\liff!B$}\DisplayProof
  & \AxiomC{}\DeduceC{$!A$}
    \AxiomC{}\DeduceC{$!A\liff!B$}
    \RightLabel{\Elim{\liff}}
    \BinaryInfC{$!B$}\DisplayProof
    & \emph{You can swap the left-right order of subproofs if convenient.}
\end{tabular}
\bigskip

\noindent
\begin{tabular}{ll@{\hskip 2em}lp{.375\textwidth}}
    \multicolumn{2}{l}{Given proofs:} & this is a proof:\\
    \AxiomC{}\DeduceC{$!B$}\DisplayProof
  & \AxiomC{}\DeduceC{$!A\liff!B$}\DisplayProof
  & \AxiomC{}\DeduceC{$!B$}
    \AxiomC{}\DeduceC{$!A\liff!B$}
    \RightLabel{\Elim{\liff}}
    \BinaryInfC{$!A$}\DisplayProof
\end{tabular}
\end{defish}

\subsubsection*{Optional: Negation without Falsum}

Halbach's system. You may use these rules too.

\begin{defish}
\noindent \textbf{Reductio Ad Absurdum without Falsum}
\nopagebreak \smallskip \\ \noindent
\begin{tabular}{ll@{\hskip 3em}lp{.375\textwidth}}
    \multicolumn{2}{l}{Given proofs:} & this is a proof:\\
    \AxiomC{$\mathcolor{gray}{!A}$}\DeduceC{$!B$}\DisplayProof &
  \AxiomC{$\mathcolor{gray}{!A}$}\DeduceC{$\lnot!B$}\DisplayProof &
  \AxiomC{$\Discharge{\mathcolor{gray}{!A}}{n}$}\DeduceC{$!B$}
  \AxiomC{$\Discharge{\mathcolor{gray}{!A}}{n}$}\DeduceC{$\lnot!B$}
  \DischargeRule{\Intro{\lnot}}{n} \BinaryInfC{$\lnot!A$}\DisplayProof
  & \emph{The assumption $!A$ need not be present in either subproof.
  If present, it must be discharged.}
\end{tabular}
\end{defish}

\begin{defish}
\noindent \textbf{Indirect Proof without Falsum}
\nopagebreak \smallskip \\ \noindent
\begin{tabular}{ll@{\hskip 2em}lp{.325\textwidth}}
    \multicolumn{2}{l}{Given proofs:} & this is a proof:\\
    \AxiomC{$\mathcolor{gray}{\lnot!A$}}\DeduceC{$!B$}\DisplayProof 
  & \AxiomC{$\mathcolor{gray}{\lnot!A}$}\DeduceC{$\lnot!B$}\DisplayProof 
  & \AxiomC{$\Discharge{\mathcolor{gray}{\lnot!A}}{n}$}\DeduceC{$!B$}
    \AxiomC{$\Discharge{\mathcolor{gray}{\lnot!A}}{n}$}\DeduceC{$\lnot!B$}
    \DischargeRule{\Elim{\lnot}}{n}
    \BinaryInfC{$!A$}\DisplayProof
    & \emph{The assumption $\lnot!A$ need not be present in either subproof. If present, it must be discharged.}
\end{tabular}
\end{defish}

This Reductio ad Absurdum rule (\Intro{\lnot}) here is derivable from
the one with Falsum (\Elim{\bot}) together with \Intro{\bot}:
\begin{prooftree}
\AxiomC{$\Discharge{\mathcolor{gray}{!A}}{n}$}
\DeduceC{$!B$}
        \AxiomC{$\Discharge{\mathcolor{gray}{!A}}{n}$}
        \DeduceC{$\lnot!B$}
    \RightLabel{\Intro{\bot}}
    \BinaryInfC{$\bot$}
    \DischargeRule{\Elim{\bot}}{n}
    \UnaryInfC{$\lnot!A$}
\end{prooftree}

This Indirect Proof rule (\Elim{\lnot}) is derivable from the one with
Falsum ($\bot_C$) together with \Intro{\bot}:
\begin{prooftree}
    \AxiomC{$\Discharge{\mathcolor{gray}{\lnot!A}}{n}$}\DeduceC{$!B$}
    \AxiomC{$\Discharge{\mathcolor{gray}{\lnot!A}}{n}$}\DeduceC{$\lnot!B$}
    \RightLabel{\Intro{\bot}}
    \BinaryInfC{$\bot$}
    \DischargeRule{$\bot_C$}{n}
    \UnaryInfC{$!A$}
\end{prooftree}

\subsubsection*{Optional: further negation rules}

You may use the following rules too. 

\begin{defish}
\noindent \textbf{Ex Falso Quodlibet (aka Ex Falso or Explosion)}
\nopagebreak \smallskip \\ \noindent
\begin{tabular}{l@{\hskip 3em}l}
    Given proof: & this is a proof:\\
    \AxiomC{}\DeduceC{$\bot$}\DisplayProof &
    \AxiomC{}\DeduceC{$\bot$}
    \RightLabel{$\bot_I$}
    \UnaryInfC{$!B$}\DisplayProof
\end{tabular}
\end{defish}

Ex Falso ($\bot_I$) can be derived from Indirect Proof
($\bot_C$): it is simply the case where we don't discharge. The 
opposite isn't true: you can't derive Indirect Proof from Ex Falso.

\begin{defish}
\noindent \textbf{Ex Falso Quodlibet (aka Ex Falso or Explosion) without Falsum}
\nopagebreak \smallskip \\ \noindent
\begin{tabular}{ll@{\hskip 3em}l}
    \multicolumn{2}{l}{Given proofs:} & this is a proof:\\
    \AxiomC{}\DeduceC{$!A$}\DisplayProof &
    \AxiomC{}\DeduceC{$\lnot!A$}\DisplayProof &
    \AxiomC{}\DeduceC{$!A$}
    \AxiomC{}\DeduceC{$\lnot!A$}
    \RightLabel{EFQ}
    \BinaryInfC{$!B$}\DisplayProof
\end{tabular}
\end{defish}

Likewise, Ex Falso without Falsum (EFQ) can be derived from Indirect Proof 
without Falsum ($\Elim{\lnot}$): it is the no-discharge case of the latter. 
But the opposite is not true; so we couldn't replace Indirect Proof 
with Ex Falso in this version either. 
\begin{defish}
\noindent \textbf{Double Negation Elimination}
\nopagebreak \smallskip \\ \noindent
\begin{tabular}{l@{\hskip 3em}l}
    Given proof: & this is a proof:\\
    \AxiomC{}\DeduceC{$\lnot\lnot!A$}\DisplayProof &
    \AxiomC{}\DeduceC{$\lnot\lnot!A$}
    \RightLabel{DNE}
    \UnaryInfC{$!A$}\DisplayProof
\end{tabular}
\end{defish}

This rule is equivalent to Indirect Proof, given the other negation
rules. DNE is !!{derivable} from Indirect Proof with Falsum
Introduction:
\begin{prooftree}\def\extraVskip{4pt}
    \AxiomC{}\DeduceC{$\lnot\lnot!A$}
        \AxiomC{$\Discharge{\lnot!A}{n}$}
    \RightLabel{\Intro{\bot} (Falsum Intro)}
    \BinaryInfC{$\bot$}
    \DischargeRule{$\bot_C$ (Indirect Proof)}{n}
    \UnaryInfC{$!A$}
\end{prooftree}
(Note at step $n$, applying $\bot_C$ to conclude $!A$, we discharge
assumptions of $\lnot!A$; since we only have one on the right, we only
discharge there.) Conversely, Indirect Proof is derivable from
DNE with Reductio Ad Absurdum:
\begin{prooftree}\def\extraVskip{4pt}
    \AxiomC{$\Discharge{\lnot!A}{n}$}
    \DeduceC{$\bot$}
    \DischargeRule{\Intro{\lnot} (Reductio)}{n}
    \UnaryInfC{$\lnot\lnot!A$}
    \RightLabel{DNE}
    \UnaryInfC{$!A$}
\end{prooftree}
Thus either of Indirect Proof or DNE can be added to the other negation 
rules to get a full system, known as Classical Propositional Logic. 
But at least one is needed: you could not derive them from the other.

\paragraph{Remark} Negation and falsum rules belong to four groups:

\begin{enumerate}
\item Falsum Introduction. From $p$ and $\lnot p$ absurdity follows.
\item Reductio Ad Absurdum. If assuming $p$ leads to absurdity, then
not-$p$ follows.
\item Ex Falso. From absurdity anything follows.
\item Indirect Proof and DNE. If $\lnot p$ leads to absurdity, then
$p$ follows.
\end{enumerate}

(1)+(2) constitute a minimal logic of negation 
where we can get from contradictions to \emph{negative} conclusions.
It proves the Law of Non-Contradiction:
\begin{equation}\tag{Law of Non-Contradiction}
\lnot(!A\land\lnot!A)
\end{equation}
But it doesn't give us the Ex Falso rules, nor the Disjunctive Syllogism
rule:
\begin{prooftree}
\AxiomC{$!A\lor !B$}
        \AxiomC{$\lnot!A$}
    \RightLabel{Disjunctive Syllogism}
    \BinaryInfC{$!B$} 
\end{prooftree}

(1)-(2)-(3) constitute a stronger logic of negation, Intuitionistic Logic. 
With it we can derive not only the Law of Non-Contradiction, but 
also Disjunctive Elimination. It doesn't derive (4), 
however; hence not Double Negation Elimination. Nor does it derive
the Law of Excluded Middle: 
\begin{equation}\tag{Law of Excluded Middle}
    \lnot(!A\land\lnot!A)
\end{equation}

(1)-(2)-(3)-(4) is the strongest logic of negation, Classical Logic.
In it we have !!{derivation}s of all the theorems and rules mentioned
above.

\end{document}

