% Part: free-logic
% Chapter: tableaux
% Section: common-rules

\documentclass[../../../include/open-logic-section]{subfiles}

\begin{document}

\olfileid{frl}{tab}{cor}

\olsection{Common Rules}

The rules below are common to all systems of free logic. 

\subsection{Rules for $\lforall$}

\begin{defish}
\AxiomC{\sFmla{\True}{\lforall[x][!A(x)]}}
\RightLabel{\TRule{\True}{\lforall}}
\UnaryInfC{$\sFmla{\False}{\lfrexists t} \quad \mid \quad \sFmla{\True}{!A(t)}$}
\DisplayProof
\hfill
\AxiomC{\sFmla{\False}{\lforall[x][!A(x)]}}
\RightLabel{\TRule{\False}{\lforall}}
\UnaryInfC{\sFmla{\True}{\lfrexists a}}
\noLine
\UnaryInfC{\sFmla{\False}{!A(a)}}
\DisplayProof
\end{defish}

In \TRule{\True}{\lforall}, $t$ is \iftag{fnTerms}{a closed term (i.e., one without
variables)}{!!a{constant}}. In \TRule{\False}{\lforall}, $a$~is !!a{constant} 
which must not occur anywhere in the branch above \TRule{\False}{\lforall}
rule. We call $a$ the \emph{eigenvariable} of the \TRule{\False}{\forall}
inference.

\subsection{Rules for $\lexists$}

\begin{defish}
\AxiomC{\sFmla{\True}{\lexists[x][!A(x)]}}
\RightLabel{\TRule{\True}{\lexists}}
\UnaryInfC{\sFmla{\True}{\lfrexists a}}
\noLine
\UnaryInfC{\sFmla{\True}{!A(a)}}
\DisplayProof
\hfill
\AxiomC{\sFmla{\False}{\lexists[x][!A(x)]}}
\RightLabel{\TRule{\False}{\lexists}}
\UnaryInfC{$\sFmla{\False}{\lfrexists t} \quad \mid \quad \sFmla{\False}{!A(t)}$}
\DisplayProof
\end{defish}

Again, $t$~is $t$ is \iftag{fnTerms}{a closed term}{!!a{constant}}, and $a$~is 
!!a{constant} which does not occur in the branch above 
the~\TRule{\False}{\lexists} rule. We call
$a$ the \emph{eigenvariable} of the \TRule{\False}{\lexists} inference.

The condition that an eigenvariable not occur in the branch above 
the \TRule{\False}{\lforall} or \TRule{\True}{\lexists} inference is called the
\emph{eigenvariable condition}.

\subsection{Rule for $\eq$}

The rule for $\eq$ is ($t$, $t_1$, and $t_2$ are closed terms):

\begin{defish}
\AxiomC{\sFmla{\True}{\eq[t_1][t_2]}}
\noLine
\UnaryInfC{\sFmla{\True}{!A(t_1)}}
\RightLabel{$\TRule{\True}{\eq}$}
\UnaryInfC{\sFmla{\True}{!A(t_2)}}
\DisplayProof
\hfill
\AxiomC{\sFmla{\True}{\eq[t_1][t_2]}}
\noLine
\UnaryInfC{\sFmla{\False}{!A(t_1)}}
\RightLabel{$\TRule{\False}{\eq}$}
\UnaryInfC{\sFmla{\False}{!A(t_2)}}
\DisplayProof
\end{defish}
Note that in contrast to all the other rules, $\TRule{\True}{\eq}$ and
$\TRule{\False}{\eq}$ require that \emph{two} signed !!{formula}s
already appear on the branch, namely both $\sFmla{\True}{\eq[t_1][t_2]}$
and $\sFmla{S}{!A(t_1)}$.


\end{document}
