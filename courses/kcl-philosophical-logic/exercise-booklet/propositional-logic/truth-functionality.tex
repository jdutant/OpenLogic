% Part: exercise-booklet
% Chapter: propositional-logic
% Section: truth-functionality

\documentclass[../../../include/open-logic-section]{subfiles}

\begin{document}

\olfileid{exb}{pl}{tfn}

\olsection{Truth-functionality}

\begin{prob}
    \citep[1.1 item 1]{MacFarlane-2020-PhilosophicalLogicContemporary}
    Give an example of a truth-functional connective other than the
usual ones (conjunction, disjunction, negation, material conditional
and bi-conditional). Explain what makes it truth-functional. Give an
example of a non-truth-functional connective, and show that it is not
truth-functional.

    \begin{ans} One possible answer among many:
        \begin{itemize}
        \item `neither... nor' is a connective connective $\circ$ such
        that $!A\circ!B$ is true just if neither $!A$ not $!B$ is
        true. Since we can give its truth value as a function of the
        truth values of $!A$ and $!B$, it is truth-functional.
        \item 'It is necessary that' is not truth-functional. We can
        find two claims $!A$, $!B$ with the same truth value, for
        instance `water is water' and `England is a democracy' such
        that `it is necessary that $!A$' is true but `it is necessary
        that $!B$' is not. Therefore `it is necessary that' is not
        truth-functional
    \end{itemize}
    
    \end{ans}
\end{prob}

\begin{prob}
    A $n$-ary truth function is a function $n$ truth values to
    a truth-values. For instance, conjunction is a truth function that
    maps two truth values to $\ltrue$ if both are true, and to $\lfalse$
    otherwise.

    There are four \emph{unary} truth functions. Describe them all. 
    The propositional language only has one (which?). Why is that enough?

    \begin{ans}
     The four possible unary truth functions are described by the 
     table below:

        \bigskip
        \begin{tabular}{l|cccc}
            $\Obj{\pvar{A}}$    & $f_1$   & $f_2$   & $f_3$   & $f_4$ \\
            $\True$       & $\True$ & $\True$ & $\False$ & $\False$ \\ 
            $\False$      & $\True$ & $\False$ & $\True$ & $\False$ \\
        \end{tabular}
        \bigskip

    Our language has $\lnot$ to express $f_3$. The others can be 
    expressed indirectly: $f_1$ with the zero-ary connective $\top$,
    $f_4$ with the zero-ary connective $\bot$, and $f_2$ using no 
    connective and simply writing $\Obj{\pvar{A}}$.
    \end{ans}
\end{prob}

\end{document}
