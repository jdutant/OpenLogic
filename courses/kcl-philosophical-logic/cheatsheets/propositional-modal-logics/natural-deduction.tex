% Part: cheat-sheets
% Chapter: modal-propositional-logics
% Section: natural-deduction

\documentclass[../../../../include/open-logic-section]{subfiles}

\begin{document}

\olfileid{chs}{pml}{nd}

\olsection{Natural Deduction}

\begin{figure}
    \noindent
    {\setlength\extrarowheight{3em}
    \begin{tabular}{cc}
        \AxiomC{}\DeduceC{$\Box!A$}
        \AxiomC{}\DeduceC{$\Box!B$}
        \AxiomC{$\Discharge{!A}{n},\Discharge{!B}{n}$ \emph{at most}}\DeduceC{$!C$}
        \DischargeRule{\Intro{\Box}}{n}
        \TrinaryInfC{$\Box!C$}
        \DisplayProof
    &   \AxiomC{}\DeduceC{$\Diamond!A$}
        \AxiomC{}\DeduceC{$\Box!B$}
        \AxiomC{$\Discharge{!A}{n},\Discharge{!B}{n}$ \emph{at most}}\DeduceC{$!C$}
        \DischargeRule{\Intro{\Diamond}}{n}
        \TrinaryInfC{$\Diamond!C$}
        \DisplayProof
    \\
    &   \AxiomC{}\DeduceC{$\Diamond\lfalse$}
        \RightLabel{\Elim{\Diamond}}
        \UnaryInfC{$\lfalse$}
        \DisplayProof
    \end{tabular}
    }
    \\[1em]
    \emph{In \Intro{\Box} and \Intro{\Diamond} the !!{derivation} on the right 
    has at most $!A,!B$ undischarged assumptions before applying the rule.}
    \\[1em]
    \emph{You may also use the derived rules:}
    \\[1em] \noindent
    {\setlength\extrarowheight{3em} 
    \begin{tabular}{ll}
        \AxiomC{}\DeduceC{$\Box!A_1$}
        \AxiomC{}\DeduceC{$\ldots$}
        \AxiomC{}\DeduceC{$\Box!A_k$}
        \AxiomC{$\Discharge{!A_1}{n},\ldots,\Discharge{!A_k}{n}$ \emph{at most}}\DeduceC{$!B$}
        \DischargeRule{\Intro{\Box}}{n}
        \QuaternaryInfC{$\Box!B$}
        \DisplayProof
        &   \AxiomC{\emph{no assumption}}\DeduceC{$!A$}
        \RightLabel{Nec}
        \UnaryInfC{$\Box!A$}
        \DisplayProof
    \\
        \multicolumn{2}{l}{
        \AxiomC{}\DeduceC{$\Diamond!A$}
        \AxiomC{}\DeduceC{$\Box!B_1$}
        \AxiomC{}\DeduceC{\ldots}
        \AxiomC{}\DeduceC{$\Box!B_k$}
        \AxiomC{$\Discharge{!A}{n},\Discharge{!B_1}{n},\ldots,\Discharge{!B_k}{n}$ \emph{at most}}\DeduceC{$!C$}
        \DischargeRule{\Intro{\Diamond}}{n}
        \QuinaryInfC{$\Diamond!C$}
        \DisplayProof
        }
    \end{tabular}
    }
    \\[1em] \noindent

\caption{Natural Deduction rules for Propositional Modal Logic K.}
\ollabel{natdedpmlk}
\end{figure}

\begin{figure}
    \begin{center} 
    {\setlength\extrarowheight{3em} 
    \begin{tabular}{ll}
        \AxiomC{}\DeduceC{$\Box!A$}
        \RightLabel{\Ax{D}}
        \UnaryInfC{$\Diamond!A$}
        \DisplayProof
    \\
        \AxiomC{}\DeduceC{$\Box!A$}
        \RightLabel{$\Box$\Ax{T}}
        \UnaryInfC{$!A$}
        \DisplayProof
    &   \AxiomC{}\DeduceC{$!A$}
        \RightLabel{$\Diamond$\Ax{T}}
        \UnaryInfC{$\Diamond!A$}
        \DisplayProof
    \\
        \AxiomC{}\DeduceC{$\Box!A$}
        \RightLabel{$\Box$\Ax{4}}
        \UnaryInfC{$\Box\Box!A$}
        \DisplayProof
    &   \AxiomC{}\DeduceC{$\Diamond\Diamond!A$}
        \RightLabel{$\Diamond$\Ax{4}}
        \UnaryInfC{$\Diamond!A$}
        \DisplayProof
    \\
        \AxiomC{}\DeduceC{$!A$}
        \RightLabel{$\Box$\Ax{B}}
        \UnaryInfC{$\Box\Diamond!A$}
        \DisplayProof
    &   \AxiomC{}\DeduceC{$\Diamond\Box!A$}
        \RightLabel{$\Diamond$\Ax{B}}
        \UnaryInfC{$!A$}
        \DisplayProof
    \\
        \AxiomC{}\DeduceC{$\Diamond!A$}
        \RightLabel{$\Box$\Ax{5}}
        \UnaryInfC{$\Box\Diamond!A$}
        \DisplayProof
    &   \AxiomC{}\DeduceC{$\Box!A$}
        \RightLabel{$\Diamond$\Ax{5}}
        \UnaryInfC{$\Diamond\Box!A$}
        \DisplayProof

    \end{tabular}
    }
    \end{center}

    \caption{Natural Deduction rules for stronger modal logics.}
\ollabel{natdedspml}
\end{figure}



\end{document}