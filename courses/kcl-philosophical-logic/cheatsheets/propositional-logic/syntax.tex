% Part: cheat-sheets
% Chapter: propositional-logic
% Section: syntax

\documentclass[../../../../include/open-logic-section]{subfiles}

\begin{document}

\olfileid{chs}{pl}{syn}

\olsection{Syntax}

\begin{defn}[Formula]
    \ollabel{defn:formulas}
    The set~$\Frm[L_0]$ of \emph{!!{formula}s} of propositional logic
    is defined inductively as follows:
    \begin{enumerate}
    \tagitem{prvFalse}{$\lfalse$ is an atomic !!{formula}.}{}
    
    \tagitem{prvTrue}{$\ltrue$ is an atomic !!{formula}.}{}
    
    \item Every !!{propositional variable}~$\Obj p_i$ is an atomic
      !!{formula}.
    
    \tagitem{prvNot}{If $!A$ is !!a{formula}, then $\lnot !A$ is
      !!{formula}.}{}
    
    \tagitem{prvAnd}{If $!A$ and $!B$ are !!{formula}s, then $(!A \land
      !B)$ is a !!{formula}.}{}
    
    \tagitem{prvOr}{If $!A$ and $!B$ are !!{formula}s, then $(!A \lor !B)$
      is a !!{formula}.}{}
    
    \tagitem{prvIf}{If $!A$ and $!B$ are !!{formula}s, then $(!A \lif !B)$
      is a !!{formula}.}{}
    
    \tagitem{prvIff}{If $!A$ and $!B$ are !!{formula}s, then $(!A
      \liff !B)$ is a !!{formula}.}{}
    
    \tagitem{limitClause}{Nothing else is a !!{formula}.}{}
    \end{enumerate}
\end{defn}

\iftag{defNot,defOr,defAnd,defIf,defIff,defTrue,defFalse,defEx,defAll}{%
In addition to the primitive connectives introduced
above, we also use the following \emph{defined} symbols:
\startycommalist
  \iftag{defNot}{\ycomma $\lnot$ (negation)}{}%
  \iftag{defAnd}{\ycomma $\land$ (conjunction)}{}%
  \iftag{defOr}{\ycomma $\lor$ (disjunction)}{}%
  \iftag{defIf}{\ycomma $\lif$ (!!{conditional})}{}%
  \iftag{defIff}{\ycomma $\liff$ (!!{biconditional})}{}%
  \iftag{defFalse}{\ycomma !!{falsity}~$\lfalse$}{}%
  \iftag{defTrue}{\ycomma !!{truth}~$\ltrue$}}{}.


You may be familiar with different terminology and symbols than the
    ones we use above. Logic texts (and teachers) commonly use 
    $\sim$, $\neg$, or~!\ for ``negation'', $\wedge$, $\cdot$, or $\&$
    for ``conjunction''.  Commonly used symbols for the ``conditional'' or
    ``implication'' are $\rightarrow$, $\Rightarrow$, and $\supset$.
    \iftag{prvIff,defIff}{Symbols for ``biconditional,'' ``bi-implication,''
      or ``(material) equivalence'' are $\leftrightarrow$,
      $\Leftrightarrow$, and $\equiv$.}{}
    \iftag{prvFalse,defFalse}{The $\lfalse$ symbol is variously called
      ``falsity,'' ``falsum,'', ``absurdity,'' or ``bottom.''}{}
    \iftag{prvTrue,defTrue}{The $\ltrue$ symbol is variously called
      ``truth,'' ``verum,'' or ``top.''}{}

\begin{defn}[Main connective]
    The \emph{main connective} of a non-atomic !!{formula} is the last
    connective introduced in the construction of the formula.
\end{defn}

For instance, in $\lnot \PVarP_0 \lor (\PVarP_1 \lif \PVarP_1)$, the
main connective is $\lor$.
    
\end{document}