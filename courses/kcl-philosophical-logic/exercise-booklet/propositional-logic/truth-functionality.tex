% Part: exercise-booklet
% Chapter: propositional-logic
% Section: truth-functionality

\documentclass[../../../include/open-logic-section]{subfiles}

\begin{document}

\olfileid{exb}{pl}{tfn}

\olsection{Truth-functionality}

\begin{prob}
    \citep[1.1 item 1]{MacFarlane-2020-PhilosophicalLogicContemporary}
    Give an example of a truth-functional connective other than the
usual ones (conjunction, disjunction, negation, material conditional
and bi-conditional). Explain what makes it truth-functional. Give an
example of a non-truth-functional connective, and show that it is not
truth-functional.
\end{prob}

\begin{prob}
    A $n$-ary truth function is a function $n$ truth values to
    a truth-values. For instance, conjunction is a truth function that
    maps two truth values to $\ltrue$ if both are true, and to $\lfalse$
    otherwise.

    There are four \emph{unary} truth functions. Describe them all. 
    The propositional language only has one (which?). Why is that enough?
\end{prob}

\end{document}
