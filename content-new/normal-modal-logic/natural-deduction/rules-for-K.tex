% Part: normal-modal-logic
% Chapter: natural-deduction
% Section: rules-for-K

\documentclass[../../../include/open-logic-section]{subfiles}

\begin{document}

\olfileid{nml}{nd}{rul}

\olsection{Rules for \Log{K}}

We accept all the rules of propositional logic: assumption and rules
for propositional connectives. 

For the modal operator $\Box$, we add:
\begin{defish}
\AxiomC{}\DeduceC{$\Box!A$}
\AxiomC{}\DeduceC{$\Box!B$}
\AxiomC{$\Discharge{!A}{n},\Discharge{!B}{n}$ \emph{at most}}\DeduceC{$!C$}
\DischargeRule{\Intro{\Box}}{n}
\TrinaryInfC{$\Box!C$}
\DisplayProof
\end{defish}

\iftag{prvDiamond}{
For the modal operator $\Diamond$, we add:

\begin{defish}
\begin{tabular}{cc}
    \AxiomC{}\DeduceC{$\Diamond!A$}
    \AxiomC{}\DeduceC{$\Box!B$}
    \AxiomC{$\Discharge{!A}{n},\Discharge{!B}{n}$ \emph{at most}}\DeduceC{$!C$}
    \DischargeRule{\Intro{\Diamond}}{n}
    \TrinaryInfC{$\Diamond!C$}
    \DisplayProof

    &
    \AxiomC{}\DeduceC{$\Diamond\lfalse$}
    \RightLabel{\Elim{\Diamond}}
    \UnaryInfC{$\lfalse$}
    \DisplayProof
\end{tabular}
\end{defish}

}
{} % end of \iftag{prvDiamond}{

\begin{explain}
Normal modal logics, i.e. $\Log{K}$ and stronger systems, all share the 
idea that \emph{logical consequences of all that is necessary are necessary}.
$!A\lor !B$ is a logical consequence of $!A$, so from $\Box !A$ we 
should infer $\Box (!A\lor !B)$. $!A\land !B$ is a logical consequence 
of $!A$ and $!B$, so if we have both $\Box !A$ and $\Box !B$ we 
shoul dinfer $\Box(!A\land !B)$. 

The rule above captures this idea. It concludes $\Box!C$ from:
\begin{itemize}
\item !!^{derivation}s of $\Box!A$ and of $\Box!B$, and
\item A !!{derivation} $!C$ from \emph{at most} the assumptions of
$!A$, $!B$. If we have such a !!{derivation} then $!C$ is a logical
consequence of $!A$, $!B$. It is important that the !!{derivation} on
right at most uses $!A$ and $!B$: if it used a further assumption,
then we would not have shown that $!C$ is a logical consequences of
$!A$ and $!B$.
\end{itemize}
On the right, $!A$, $!B$ are only temporary assumptions to show
that we can derive $!C$ from these alone. They are therefore discharged
when we conclude $!C$.

The rule directly captures the idea that logical consequences of \emph{two}
necessities are necessary. What about logical consequences of \emph{three}
or more necessities? Repeated applications of the 
rule will ensure that they are necessary too.

There is no $\Box$ elimination rule. Or, perhaps a better way to think about 
this: the rule is both a way to eliminate $\Box$ (in $\Box!A$ 
and $\Box!B$) and to introduce $\Box$ (in $\Box!C$).
\end{explain}

\begin{ex}
A special application of the rule is when the !!{derivation} of $!C$
has no assumptions at all. We can then append $\Box!C$ to the
!!{derivation}. That in turns counts as a !!{derivation} of $\Box!C$
without assumptions, so we can append $\Box\Box!C$, and so on.
Generalizing, our system obeys the \emph{Necessitation} rule:
\begin{quote}
    Nec. If $\Proves !A$, then $\Proves \Box!A$.
\end{quote}
Thus logical truths are necessary, necessarily necessary, and so on.
(Note that this is \emph{not} equivalent to the principle that \emph{all}
necessities are necessarily necessary, i.e. $\Box!A\lif\Box\Box!A$.)
\end{ex}

\iftag{prvDiamond}{ Our \Elim{\Diamond} rule captures the idea that
what is logically false is not possible. Note that this rule is forced
upon us if we accept two ideas. (1) The idea that logical consequences
of all that is necessary are necessary (which motivates our
\Intro{\Box} rule). (2) The idea that what is necessary not so is not
possible:
$$
\Box\lnot !A\lif\lnot\Diamond !A
$$
For suppose $!A$ is logically false. Then $\lnot!A$ is logically true.
By idea (1), $\Box\lnot!A$ should hold. By (2), $\lnot\Diamond !A$ 
should hold.

Our \Intro{\Diamond} rule is less immediately obvious. First, one thing 
is worth clearing up: we do \emph{not} want a rule that says that
logical consequences of \emph{all} possibilities are possible. 
Suppose for instance that $!A$ is \emph{contingent}: $!A$ is possible,
but $\lnot!A$ is possible too. That should \emph{not} entail that $!A\land\lnot!A$
is possible---even though $!A\land\lnot!A$ is a logical consequence
of $!A$ and $\lnot!A$ \emph{together}. In general, possibilities
need not aggregate: if $!A$ is possible and $!B$ is possible, it need
not be the case that $!A\land !B$ is possible. Necessities, by contrast,
aggregate: if $!A$ is necessary and $!B$ is necessary, then 
 $!A\land !B$ is necessary. 

 

Suppose $!A$ is a logical falsehood; then $\lnot!A$

One way to justify them is the following fact.

\begin{prop}
    Given the $\Intro{\Box}}$ rule, the \Intro{\Diamond},
    \Elim{\Diamond} rules are inter-!!{derivable} with the \emph{Duality} 
    rules:
    
    \bigskip
    \begin{tabular}{cc}
    \end{tabular}



\end{prop}
\begin{quote}
    \item Given the $\Intro{\Box}}$ rule, the \Intro{\Diamond},
    \Elim{\Diamond} rules are necessary and sufficient to ensure
    the \emph{Duality} of $\Box$ and $\Diamond$:
    $$\Box\lnot!A\Proves \Diamond\lnot!A$$
    $$\Diamond\lnot!A\Proves \lnot\Box!A$$
\end{quote}
Thus, if we want to uphold both Duality and our $\Box$ rule, we 
need to accept those rules for $\Diamond$.

\begin{prob}

\end{prob}

All normal modal logics 
For the modal operator $\Diamond$, we add:

\begin{defish}
\begin{tabular}{cc}
    \AxiomC{}\DeduceC{$\Diamond!A$}
    \AxiomC{}\DeduceC{$\Box!B$}
    \AxiomC{$\Discharge{!A}{n},\Discharge{!B}{n}$ \emph{at most}}\DeduceC{$!C$}
    \DischargeRule{\Intro{\Diamond}}{n}
    \TrinaryInfC{$\Diamond!C$}
    \DisplayProof

    &
    \AxiomC{}\DeduceC{$\Diamond\lfalse$}
    \RightLabel{\Elim{\Diamond}}
    \UnaryInfC{$\lfalse$}
    \DisplayProof
\end{tabular}
\end{defish}

}
{} % end of \iftag{prvDiamond}{



% \begin{figure}
%     \noindent
%     {\setlength\extrarowheight{3em}
%     \begin{tabular}{cc}
%         \AxiomC{}\DeduceC{$\Box!A$}
%         \AxiomC{}\DeduceC{$\Box!B$}
%         \AxiomC{$\Discharge{!A}{n},\Discharge{!B}{n}$ \emph{at most}}\DeduceC{$!C$}
%         \DischargeRule{\Intro{\Box}}{n}
%         \TrinaryInfC{$\Box!C$}
%         \DisplayProof
%     &   \AxiomC{}\DeduceC{$\Diamond!A$}
%         \AxiomC{}\DeduceC{$\Box!B$}
%         \AxiomC{$\Discharge{!A}{n},\Discharge{!B}{n}$ \emph{at most}}\DeduceC{$!C$}
%         \DischargeRule{\Intro{\Diamond}}{n}
%         \TrinaryInfC{$\Diamond!C$}
%         \DisplayProof
%     \\
%     &   \AxiomC{}\DeduceC{$\Diamond\lfalse$}
%         \RightLabel{\Elim{\Diamond}}
%         \UnaryInfC{$\lfalse$}
%         \DisplayProof
%     \end{tabular}
%     }
%     \\[1em]
%     \emph{In \Intro{\Box} and \Intro{\Diamond} the !!{derivation} on the right 
%     has at most $!A,!B$ undischarged assumptions before applying the rule.}
%     \\[1em]
%     \emph{You may also use the derived rules:}
%     \\[1em] \noindent
%     {\setlength\extrarowheight{3em} 
%     \begin{tabular}{ll}
%         \AxiomC{}\DeduceC{$\Box!A_1$}
%         \AxiomC{}\DeduceC{$\ldots$}
%         \AxiomC{}\DeduceC{$\Box!A_k$}
%         \AxiomC{$\Discharge{!A_1}{n},\ldots,\Discharge{!A_k}{n}$ \emph{at most}}\DeduceC{$!B$}
%         \DischargeRule{\Intro{\Box}}{n}
%         \QuaternaryInfC{$\Box!B$}
%         \DisplayProof
%         &   \AxiomC{\emph{no assumption}}\DeduceC{$!A$}
%         \RightLabel{Nec}
%         \UnaryInfC{$\Box!A$}
%         \DisplayProof
%     \\
%         \multicolumn{2}{l}{
%         \AxiomC{}\DeduceC{$\Diamond!A$}
%         \AxiomC{}\DeduceC{$\Box!B_1$}
%         \AxiomC{}\DeduceC{\ldots}
%         \AxiomC{}\DeduceC{$\Box!B_k$}
%         \AxiomC{$\Discharge{!A}{n},\Discharge{!B_1}{n},\ldots,\Discharge{!B_k}{n}$ \emph{at most}}\DeduceC{$!C$}
%         \DischargeRule{\Intro{\Diamond}}{n}
%         \QuinaryInfC{$\Diamond!C$}
%         \DisplayProof
%         }
%     \end{tabular}
%     }
%     \\[1em] \noindent

% \caption{Natural Deduction rules for Propositional Modal Logic K.}
% \ollabel{natdedpmlk}
% \end{figure}

% \begin{figure}
%     \begin{center} 
%     {\setlength\extrarowheight{3em} 
%     \begin{tabular}{ll}
%         \AxiomC{}\DeduceC{$\Box!A$}
%         \RightLabel{\Ax{D}}
%         \UnaryInfC{$\Diamond!A$}
%         \DisplayProof
%     \\
%         \AxiomC{}\DeduceC{$\Box!A$}
%         \RightLabel{$\Box$\Ax{T}}
%         \UnaryInfC{$!A$}
%         \DisplayProof
%     &   \AxiomC{}\DeduceC{$!A$}
%         \RightLabel{$\Diamond$\Ax{T}}
%         \UnaryInfC{$\Diamond!A$}
%         \DisplayProof
%     \\
%         \AxiomC{}\DeduceC{$\Box!A$}
%         \RightLabel{$\Box$\Ax{4}}
%         \UnaryInfC{$\Box\Box!A$}
%         \DisplayProof
%     &   \AxiomC{}\DeduceC{$\Diamond\Diamond!A$}
%         \RightLabel{$\Diamond$\Ax{4}}
%         \UnaryInfC{$\Diamond!A$}
%         \DisplayProof
%     \\
%         \AxiomC{}\DeduceC{$!A$}
%         \RightLabel{$\Box$\Ax{B}}
%         \UnaryInfC{$\Box\Diamond!A$}
%         \DisplayProof
%     &   \AxiomC{}\DeduceC{$\Diamond\Box!A$}
%         \RightLabel{$\Diamond$\Ax{B}}
%         \UnaryInfC{$!A$}
%         \DisplayProof
%     \\
%         \AxiomC{}\DeduceC{$\Diamond!A$}
%         \RightLabel{$\Box$\Ax{5}}
%         \UnaryInfC{$\Box\Diamond!A$}
%         \DisplayProof
%     &   \AxiomC{}\DeduceC{$\Box!A$}
%         \RightLabel{$\Diamond$\Ax{5}}
%         \UnaryInfC{$\Diamond\Box!A$}
%         \DisplayProof

%     \end{tabular}
%     }
%     \end{center}

%     \caption{Natural Deduction rules for stronger modal logics.}
% \ollabel{natdedspml}
% \end{figure}



\end{document}
