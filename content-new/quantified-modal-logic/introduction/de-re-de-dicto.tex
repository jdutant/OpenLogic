% Part: quantified-modal-logic
% Chapter: introduction
% Section: de-re-de-dicto

\documentclass[../../../include/open-logic-section]{subfiles}

\begin{document}

\olfileid{qml}{int}{drdd}

\olsection{De re vs. de dicto}

This section introduces the distinction between \emph{de re} vs
\emph{de dicto} modal claims and the related notion of 
\emph{quantifying in a modal context}.

These notions crucial to understand Quantified Modal Logic. In fact,
Quine's influential attack on Quantified Modal Logic can be summarized
thus: (a) \emph{de dicto} modal claims are intelligible, but they are
to be understood as metalinguistic claims, and (b) \emph{de re} modal
claims don't make sense, and (c) because \emph{de re} modal claims
don't make sense, quantifying in modal contexts doesn't make sense.

Western medieval logicians drew a distinction between `necessity of
claims' (necessity \emph{de dicto}) and `necessity of things' 
(\emph{de re}). Consider:

\begin{itemize}
\item Every blacksmith is necessarily a blacksmith.
\end{itemize}

Taken in one sense, this says something true: namely, that it couldn't
be that there are blacksmiths who are not blacksmith. In another sense,
it says something (arguably) false: namely, that those people who are
blacksmiths couldn't have had any other profession. Some 
Medieval logicians saw the former as attributing necessity to a claim
---it says that claim `a blacksmith is a blacksmith' is a necessity---
and the latter as attributing necessity to a thing---those people
are claimed to be `necessarily blacksmiths'. In contemporary logic 
we still use the `\emph{de dicto}` and `\emph{de re}' terminology
but characterize the two readings using different \emph{scopes} for 
`necessity' instead:

\begin{enumerate}
\item $\Box\lforall[x][(\Atom{B}{x}\lif\Atom{B}{x})]$.\\ 
It is necessary that every blacksmith is a blacksmith (\emph{de
dicto}, true).
\item $\lforall[x][\Atom{B}{x}\lif\Box\Atom{B}{x}]$.\\
Every blacksmith is something that is necessarily a blacksmith
(\emph{de re}, false).
\end{enumerate}

The crucial difference between these two claims is that, in the former,
the `$\Box$' is applied to a closed sentence, while in the latter,
the `$\Box$' is applied to a open !!{formula}. Consider applying
 $\Box$ a !!{formula} with a constant or a free !!{variable}:

\begin{itemize}
    \item $\Box\Atom{F}{a}$
    \item $\Box\Atom{F}{x}$
\end{itemize}

Read $F$ as `\dots flies' and `$a$' as a name for Alice. The first
says that Alice necessarily flies. The second says, relative to the
assignement of some object to $x$, that \emph{it} necessarily flies.
In both cases we are assigning a necessary feature to specific
individuals (`things', \emph{rea}). These are \emph{de re} modal
claims. By contrast consider applying $\Box$ to a sentence without 
free !!{variable}s or constants, say $\lexists[x][\Atom{F}{x}]$:

\begin{itemize}
    \item $\Box\lexists[x][\Box\Atom{F}{x}]$
\end{itemize}

This says that necessarily, there is a flying thing. Here we say that 
a certain fact has to be the case (it has to be that there is a flying
thing). We're not attributing a necessary property to any object. 
That is a \emph{de dicto} modal claim.

The distinction between \emph{de re} or \emph{de dicto} modal !!{formula}s
is drawn as follows.

\begin{defn}[De re vs de dicto]
\ollabel{de-re-de-dicto}
!!^a{formula} is \emph{modal} iff it contains $\Box$ or 
$\Diamond$. Modal !!{formula}s are either \emph{de re} or \emph{de dicto}:
\begin{itemize}
\item If $!A$ contains a free variable or constant, then $\Box!A$, $\Diamond!A$
are \emph{de re} modal !!{formula}s,
\item Any !!{formula} that contains a \emph{de re} modal !!{formula} is
itself a \emph{de re} modal !!{formula}.
\item Other modal !!{formula}s are \emph{de dicto}
\end{itemize}

\end{defn}

\emph{Quantifying in a modal context} is using a quantifier to bind
!!a{variable} within a \emph{de re} modal claim: 

\begin{itemize}
\item $\lexists[x][\Box\Atom{H}{x}]$.\\
Something is such that \emph{it} is necessarily human.
\end{itemize}

Here the quantifier $\lexists$ binds !!a{variable} that is within 
the \emph{de re} modal claim $\Box\Atom{H}{x}$. It quantifies `across'
the modal operator, so to say. 

A claim quantifying into modal context such as `Something is such that
it is necessarily human' does not contain constants (names) nor free
!!{variable}s. But like simpler \emph{de re} modal claims, it involves
ascribing modal properties to things: it says that there is something
or other such that \emph{it} has the property of being necessarily
human. Thus quantifying in modal context makes sense only if simpler
\emph{de re} modal claims make sense. That is our formal definition
of \emph{de re} modal claims counts such claims as \emph{de re} too.

\end{document}