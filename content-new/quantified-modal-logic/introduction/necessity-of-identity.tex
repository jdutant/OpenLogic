% Part: quantified-modal-logic
% Chapter: introduction
% Section: necessity-of-identity

\documentclass[../../../include/open-logic-section]{subfiles}

\begin{document}

\olfileid{qml}{int}{noi}

\olsection{Necessity of Identity}

Combining standard first-order logic with identity and modal logic 
allows us to prove the necessity of identity. 

\begin{defn}[Necessity of identity]
The necessity of identity is the claim that true identities are 
necessary. For particular constants $a,b$, it is the claim that 
if $a$ and $b$ are one and the same thing, it is necessarily so:
$$\eq[a][b]\lif\Box\eq[a][b]$$ 

We can also express it as a universal claim. If a thing and a thing
are one and the same, it is necessary that  are necessarily one and
the same: 
$$\lforall[x][\lforall[y][(\eq[x][y]\lif\Box\eq[x][y])]]$$
\end{defn}

(Avoid saying: `if \emph{two things} are identical \dots'. Calling $a$
 and $b$ \emph{two} things entails that they are distinct.)

According to the necessity of identity, it is never the case that 
some thing $a$ is in fact some thing $b$ but also 
could have not been $b$. To illustrate it denies claims such as:
\begin{itemize}
\item Hesperus is Phosphorus but could have failed to be Phosphorus.
\item Clark Kent is Superman but could have failed to be Superman.
\item Slab (a block of marble) is Oscar (a statue) but could have 
failed to be Oscar. 
\end{itemize} 

Informally, the proof rests on two claims:
\begin{enumerate}
    \item It is necessary that $a$ is $a$. (Where $a$ can be any name.)
    \item If $a$ is $b$ and it is necessary that $a$ is $a$, then it 
    is necesasry that $a$ is $b$.
\end{enumerate}
The first claim can be further derived from Reflexivity and the
Necessitation rule and the second is an instance of Leibniz's Law.
\begin{itemize}
    \item Reflexivity. For any name $a$, `$a$ is $a$' is a logical truth.
    \item Necessitation rule. If $!A$ is a logical truth, $\Box!A$ is
    a logical truth.
    \item Lebiniz's law. If $a$ is $b$ and $!A[a/c]$ is so, then 
    $!A[b/c]$ is so.
\end{itemize}

Formally:
\begin{prooftree}
    \AxiomC{}
    \RightLabel{\Intro{\eq}}
    \UnaryInfC{$\eq[a][a]$}
    \RightLabel{$\Box$\Ax{K}}
    \UnaryInfC{$\Box\eq[a][a]$}
\AxiomC{$\Discharge{\eq[a][b]}{1}$}
\RightLabel{\Elim{\eq}}
\BinaryInfC{$\Box\eq[a][b]$}
\DischargeRule{\Elim{\lif}}{1}
\UnaryInfC{$\eq[a][b]\lif\Box\eq[a][b]$}
\RightLabel{\Intro{\lforall}}
\UnaryInfC{$\lforall[y][(\eq[a][y]\lif\Box\eq[a][y])]$}
\RightLabel{\Intro{\lforall}}
\UnaryInfC{$\lforall[x][(\lforall[y][\eq[a][y]\lif\Box\eq[a][y]])]$}
\end{prooftree}
As the formal !!{derivation} makes clear the proof only rests on 
the Reflexivity of Identity, Lebiniz's Law (applied to de re 
modal claims) and Necessitation. 

We can extend the proof with \Intro{\lforall} steps to establish the 
universal version of the Necessity of Identity:
\begin{prooftree}
    \AxiomC{}
    \RightLabel{\Intro{\eq}}
    \UnaryInfC{$\eq[a][a]$}
    \RightLabel{$\Box$\Ax{K}}
    \UnaryInfC{$\Box\eq[a][a]$}
\AxiomC{$\Discharge{\eq[a][b]}{1}$}
\RightLabel{\Elim{\eq}}
\BinaryInfC{$\Box\eq[a][b]$}
\DischargeRule{\Elim{\lif}}{1}
\UnaryInfC{$\eq[a][b]\lif\Box\eq[a][b]$}
\RightLabel{\Intro{\lforall}}
\UnaryInfC{$\lforall[y][(\eq[a][y]\lif\Box\eq[a][y])]$}
\RightLabel{\Intro{\lforall}}
\UnaryInfC{$\lforall[x][(\lforall[y][\eq[a][y]\lif\Box\eq[a][y]])]$}
\end{prooftree}

Since the necessity of identity follows from just three principles
(Reflexivity, the Necessitation rule and Leibniz's law), anyone who
finds fault with the necessity of identity must reject at least one of
these three principles. 

Note that \emph{de re} modal claims are central to the proof: $a$ is
necessarily $a$ (necessitation of reflexivity) and if $a$ is $b$, then
from $a$ being necessarily $a$ we can infer than $b$ is necessarily
$a$ too (Leibniz's law applied to \emph{de re} modal claims). Those
suspicious of \emph{de re} modal claims (like Quine) are suspicious of
the proof too.

Suppose, on the other hand, that we accept the proof's conclusion. 
This has two important consequences: the necessary a posteriori 
and the distinction between necessity and logicality. 

Necessary a posteriori. In Hume and Kant we find the idea that 
experience cannot teach us that something is necessarily so. Experience
only teaches us what is, it doesn't show that something couldn't have 
been otherwise. Hence no necessary truth is discovered empirically
(`a posteriori').

If the !!{derivation} is correct, Hume's and Kant's idea is (arguably) 
wrong. For (arguably) observation can teach us that certain identities
are true: that Hesperus is Phosphorus, for instance. If the !!{derivation}
is correct, we know on its basis that if Hesperus is Phosphorus, it is 
necessary that Hesperus is Phosphorus. Putting the two together, we 
learn that  it is necessary that Hesperus is Phosphorus. That is a 
true necessity that is discovered a posteriori (on the basis of 
experience): a necessary a posteriori. 

Necessity vs Logicality. You may find modal notions (what could have 
been or not been the case, possible world, \dots) obscure, but logical
ones (validity) clear. If so, you could be tempted by the idea that 
necessity is just logicality. We can put that idea in a principle:

\begin{itemize}
\item (Necessity as Logicality). It is necessary that $!A$ iff
$\ulcorner!A\urcorner$ is a logical truth.
\end{itemize}

For instance: it is necessary that Alice is Alice iff `Alice is Alice'
is a logical truth. Note that on the left side we're \emph{using} the
sentence Alice and Alice. On the right side we're \emph{mentioning}
the sentence, i.e. talking about it, and saying that it is a logical
truth (it is true relative to any interpretation of the non-logical
vocabulary in it). (The corner quotes $\ulcorner\ldots\urcorner$ are a piece of notation 
introduced by Quine to mean, for any sentence $!A$, a name of that 
sentence.)

If the proof of the necessity of identity is correct, then Necessity 
as Logicality is false because its right to left direction fails. 
That is, from some $a$, $b$ such that $a$ is in fact $b$ (`Hesperus'
and `Phosphorus', say), we have: 

\begin{itemize}
\item It's necessary that $a$ is $b$.
\item But, `$a=b$' isn't a logical truth.
\end{itemize}

We still have the converse: if $\ulcorner!A\urcorner$ is a logical 
truth, then it is necessary that $!A$.

\emph{Rough historical background.} Logical Positivists (like Carnap) and 
philosophers close to them (like Wittgenstein, Carnap) hoped to 
understand necessity in terms of logical truth. For a bit of background:
traditional metaphysics (Aristotle, Descartes
and post-Cartesian Rationalists like Spinoza and Leibniz) abounds in
modal claims about what could be or could not be the case. Aristotelians
endorse essentialism, for instance: things have an essence or nature 
that puts limits on what they could be (Socrates is essentially human,
hence could not have been a lamp). At the
beginning of the XXth century, neo-Kantians and Logical Positivists
are suspicious of that kind of metaphysics (like Kant and Hume
earlier). But Logical Positivists are happy with the notion of 
\emph{logical truth}, for instance, that $!A\lif!A$ expresses a 
truth, whatever one puts for $!A$. Their hypothesis is that all 
genuine necessities will turn out to be (something like) 
logical truths, and that others are non-sensical. Thus we find the idea
that necessity is logicality in Carnap and Quine. Carnap's approach was
constructive: he tried to account for necessity in terms of 
'state-descriptions', which anticipate the possible worlds semantics
of Kripke and others. Quine's was sceptical: he saw that if quantified
modal logic made sense, de re modality had to make sense, and that if
de re modality made sense, then necessity couldn't be logicality (as
shown by the proof of necessity of identity); he concluded that 
quantified modal logic made no sense.

\end{document}
