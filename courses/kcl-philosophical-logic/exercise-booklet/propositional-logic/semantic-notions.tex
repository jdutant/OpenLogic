% Part: exercise-booklet
% Chapter: propositional-logic
% Section: semantic-notions

\documentclass[../../../include/open-logic-section]{subfiles}

\begin{document}

\olfileid{exb}{pl}{sem}

\olsection{Semantic Notions}

\begin{prob}[Satisfiability and validity]
    If the set of !!{formula}s $!A_1, !A_2,\ldots, !A_n$ is unsatisfiable, 
    then the argument with premises $!A_1,!A_2,\ldots!A_{n-1}$ 
    and conclusion $\lnot !A_n$ is valid. Explain why that is so.

    \begin{ans}
    Suppose $!A_1, !A_2,\ldots, !A_n$ is unsatisfiable and 
    let $\pAssign{v}$ be any !!{valuation} such that $\pSat{v}{!A_i}$
    for all $1<i<n-1$. Since $!A_1, !A_2,\ldots, !A_n$ is unsatisfiable,
    it's not the case that $\pSat{v}{!A_i}$ for all  $1<i<n$. Therefore it 
    must be that $\pSat/{v}{!A_n}$ and $\pSat{v}{!A_n}$. Generalizing on $pAssign{v}$,
    we have $!A_1, !A_2,\ldots, !A_{n_1}\Entails \lnot!A_n$.
    
    \emph{Here is a less formal version, also acceptable.} If 
    $!A_1, !A_2,\ldots, !A_n$ is unsatisfiable then no !!{valuation}
    makes all of them true. So any structure that makes all of 
    $!A_1, !A_2,\ldots, !A_{n-1}$ must make $!A_n$ false hence
    $\lnot!A_n$ true. Therefore any !!{valuation} that makes all of 
    $!A_1,\ldots,!A_{n_1}$ true makes $\lnot!A_n$ true too. So, the 
    argument from $!A_1,\ldots,!A_{n_1}$ to $\lnot!A_n$ is valid.

    \emph{Comment.} The answer shows that \emph{any} !!{valuation} that
    would make all of $!A_1, !A_2,\ldots, !A_{n-1}$ true would make
    $\lnot!A_n$ true. That's enough to show that the argument is valid.
    It does not assume that \emph{there actually exists} a !!{valuation}
    that makes all of them true---that is why we say `\emph{any}', not
    `some'. For all we know, it may be that 
    $!A_1, !A_2,\ldots, !A_{n-1}$ are themselves unsatisfiable.

    \end{ans}
\end{prob}

\begin{prob}
    \citep[1.1 item 4]{MacFarlane-2020-PhilosophicalLogicContemporary}
    Is the following set of !!{formula}s satisfiable?
    $$
    \Obj{\pvar{A}}\lif\Obj{\pvar{B}}, 
    \Obj{\pvar{B}}\lif\Obj{\pvar{C}},
    \lnot\Obj{\pvar{C}}\lif\lnot\Obj{\pvar{A}}
    $$

    \begin{ans}
    The set is satisfiable. Consider $\pAssign{v}$ with 
    $\pAssign{v}(\Obj{\pvar{A}})=\pAssign{v}(\Obj{\pvar{B}})
    =\pAssign{v}(\Obj{\pvar{C}})=\True$.
    It is easy to see that $\pSat{v}{\Obj{\pvar{A}}\lif\Obj{\pvar{B}}}$, 
    $\pSat{v}{\Obj{\pvar{B}}\lif\Obj{\pvar{C}}}$, and
    $\pSat{v}{\lnot\Obj{\pvar{C}}\lif\lnot\Obj{\pvar{A}}}$. Therefore the 
    these !!{formula}s are jointly satisfiable.
    \end{ans}
\end{prob}

\begin{prob}[Validity of arguments]
    \citep[1.1 item 6]{MacFarlane-2020-PhilosophicalLogicContemporary}
    Say whether the following arguments are valid. Justify your answer using
    truth tables, natural deduction proofs or informal reasoning.

    \smallskip\noindent
    \begin{tabular}{rlrl}
    1. 
        & \AxiomC{$\Obj{\pvar{A}}\lif(\Obj{\pvar{B}}\land\lnot\Obj{\pvar{B}})$}
          \UnaryInfC{$\lnot\Obj{\pvar{A}}\lor\Obj{\pvar{C}}$}
          \DisplayProof
    & 2.
        & \AxiomC{$\Obj{\pvar{A}}\lif(\Obj{\pvar{B}}\lif\pvar{C})$}
          \UnaryInfC{$\Obj{\pvar{C}}\lif(\lnot\Obj{\pvar{B}}\lif\lnot\Obj{\pvar{A}})$}
          \DisplayProof
    \end{tabular}

    \begin{ans} Informally:
        \begin{itemize}
        \item Valid. Informal justification. Suppose
        $\Obj{\pvar{A}}\lif(\Obj{\pvar{B}}\land\lnot\Obj{\pvar{B}})$
        is true on some !!{valuation} $\pAssign{v}$. Since
        $\Obj{\pvar{B}}\land\lnot\Obj{\pvar{B}}$ is not true in any
        !!{valuation}, $\Obj{\pvar{A}}$ is false on $\pAssign{v}$,
        therefore $\lnot\Obj{\pvar{A}}\lor\Obj{\pvar{C}}$ is true on
        $\pAssign{v}$.

        Natural deduction proof:
        \begin{prooftree}
        \AxiomC{$\Obj{\pvar{A}}\lif(\Obj{\pvar{B}}\land\lnot\Obj{\pvar{B}})$}
            \AxiomC{$\Discharge{\Obj{\pvar{A}}}{1}$}
        \RightLabel{\Elim{\lif}}
        \BinaryInfC{$\Obj{\pvar{B}}\land\lnot\Obj{\pvar{B}}$}
        \RightLabel{\Elim{\land}}
        \UnaryInfC{$\Obj{\pvar{B}}$}
                \AxiomC{$\Obj{\pvar{A}}\lif(\Obj{\pvar{B}}\land\lnot\Obj{\pvar{B}})$}
                    \AxiomC{$\Discharge{\Obj{\pvar{A}}}{1}$}
                \RightLabel{\Elim{\lif}}
                \BinaryInfC{$\Obj{\pvar{B}}\land\lnot\Obj{\pvar{B}}$}
                \RightLabel{\Intro{\land}}
                \UnaryInfC{$\lnot\Obj{\pvar{B}}$}
        \DischargeRule{\Elim{\lnot}}{1}
        \BinaryInfC{$\lnot\Obj{\pvar{A}}$}
        \RightLabel{\Intro{\lor}}
        \UnaryInfC{$\lnot\Obj{\pvar{A}}\lor\Obj{\pvar{C}}$}
        \end{prooftree}


        \item Not valid. Counterexample 
        $\pAssign{v}(\Obj{\pvar{A}})=\pAssign{v}(\Obj{\pvar{C}})=\True$, 
        $\pAssign{v}(\Obj{\pvar{B}})=\False$.
        
        \emph{More explicit answer}. The conclusion 
        $\Obj{\pvar{C}}\lif(\lnot\Obj{\pvar{B}}\lif\lnot\Obj{\pvar{A}})$
        is false on a !!{valuation} $\pAssign{v}$ iff 
        $\pAssign{v}(\Obj{\pvar{C}})=\True$ and 
        $\pSat/{v}{\lnot\Obj{\pvar{B}}\lif\lnot\Obj{\pvar{A}}}$ 
        hence $\pSat{v}{\lnot\Obj{\pvar{B}}}$ and 
        $\pSat/{v}{\lnot\Obj{\pvar{A}}}$, iff
        $\pAssign{v}(\Obj{\pvar{A}})=\True$, 
        $\pAssign{v}(\Obj{\pvar{B}})=\False$
        and $\pAssign{v}(\Obj{\pvar{C}})=\True$. Let $\pAssign{v}$ 
        be such a valuation. We have $\pSat{v}{\Obj{\pvar{C}}}$
        hence $\pSat{v}{\Obj{\pvar{B}}\lif\Obj{\pvar{C}}}$
        hence $\pSat{v}{\Obj{\pvar{A}}\lif(\Obj{\pvar{B}}\lif\Obj{\pvar{C}})}$.
        Since the premise is true on $\pAssign{v}$ but the conclusion
        false, the argument is not valid.
        
        \end{itemize}
    \end{ans}
\end{prob}



