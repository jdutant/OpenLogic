% Part: exercise-booklet
% Chapter: free-logic
% Section: semantics

\documentclass[../../../include/open-logic-section]{subfiles}

\begin{document}

\olfileid{exb}{frl}{sem}

\olsection{Semantics}

\begin{prob}[Semantics]
Explain why the following are not valid in any of the free logics.
\begin{enumerate}
\item $\lexists[x][(\eq[a][x])]$
\item $\lexists[x][(\eq[x][x])]$
\end{enumerate}
\begin{ans}
The simplest acceptable answers just describe the !!{structure}s that
make each formula not true. 
\begin{enumerate}
\item In negative free logic, there are !!{structure}s in which the
	interpretation of $a$ is undefined. In those !!{structure}s,
	$\lexists[x][(\eq[a][x])]$ is false. So the formula is not valid
	in negative free logic.\\
	In positive free logic, there are !!{structure}s in which the
	interpretation of $a$ is an object of the \emph{outer} domain not
	in the (inner) domain. In those !!{structure}s,
	$\lexists[x][(\eq[a][x])]$ is false. So the formula is not valid
	in positive free logic.
    \iftag{neuFrl}{\\
	In neutral free logic, there are !!{structure}s in which the
	interpretation of $a$ is not an object of the (inner) domain. In
	those !!{structure}s, $\eq[a][x]$ is $\Undef$ relative to any
	assignment of an object of the (inner) to $x$. Hence
	$\lexists[x][(\eq[a][x])]$ is $\Undef$. So the formula is not
	valid in neutral free logic.}{}
\item In negative free logic, there are !!{structure}s with an empty
	domain. In those !!{structure}s $\lexists[x][(\eq[x][x])]$ is
	false. So the formula is not valid in negative free logic. \\
	In positive free logic, there are !!{structure}s with an empty
	\emph{inner} domain. In those !!{structure}s,
	$\lexists[x][(\eq[x][x])]$ is false. So the formula is not valid
	in positive free logic. \iftag{neuFrl}{\\
	In neutral free logic, there are !!{structure}s with an empty
	(inner) domain. In those !!{structure}s,
	$\lexists[x][(\eq[x][x])]$ is false. So the formula is not valid
	in neutral free logic.}{}
\end{enumerate}
More rigorous answers say more about the assignments that make the
formulas not true in the relevant !!{structure}s. 
\begin{enumerate}
\item Negative free logic. $\lexists[x][(\eq[a][x])]$ is true in
	!!a{structure} of NFL iff there is an assignment that assigns to
	$x$ that an object of the domain that the !!{structure}'s
	interpretation assigns to $a$. But in negative free logic, the
	interpretation of $a$ can be undefined (the interpretation does
	not assign any object to $a$). If so, there is no assignement that
	assigns to $x$ an object of the domain that the !!{structure}'s
	interpretation assigns to $a$.\\
	Positive free logic. $\lexists[x][(\eq[a][x])]$ is true in
	!!a{structure} of PFL iff there is an assignement that assigns to
	$x$ that an object of the \emph{inner} domain that the
	!!{structure}'s interpretation assigns to $a$. But in positive
	free logic, the interpretation of $a$ can be an object of the
	\emph{outer} domain that is not in the inner domain. If so, there
	is no assignement that assigns to $x$ an object of the domain that
	the !!{structure}'s interpretation assigns to $a$.
    \iftag{neuFrl}{\\
	Neutral free logic. In neutral free logic, a formula $!A(a)$ is
	true if the interpretation assigns $a$ to an object of the domain
	and $!A(a)$ is true of it, false if the if the interpretation
	assigns $a$ to an object of the domain and $!A(a)$ is false of it,
	but undefined if the interpretation of $a$ is undefined (or in the
	outer domain, if we're using !!a{structure} of PFL). Thus if in
	some !!{structure} the interpretation of $a$ is undefined (or an
	object of the outerdomain), there is no assignement of object in
	the (inner) domain that makes $\eq[a][x]$ true, so
	$\lexists[x][(\eq[a][x])]$ $\lexists[x][(\eq[a][x])]$ is true
	!!a{structure} }
\item $\lexists[x][(\eq[x][x])]$ is true in a model of free logic (of
	any kind) iff there is some assignement under which $x$ and $x$
	are assigned to the same object. But assignements must assign to
	variables objects of the domain (negative free logic) or inner
	domain (positive free logic) (and either of these options in
	neutral free logic). And !!{structure}s of free logics may have an
	empty (inner) domain. On those !!{structure}s there are no
	assignments at all, so $\lexists[x][(\eq[x][x])]$ is false on all
	three logics. So the formula is not valid in any of the free
	logics.
\end{enumerate}
\end{ans}
\end{prob}

\begin{prob}[Existence]
Explain why no formula of the form $\lexists[x][!A]$ is valid in any
of the (inclusive) free logics. 
\iftag{neuFrl}{(For neutral free logic assume a truth-preservation
notion of validity.)}
\begin{ans}
In all free logics, a formula of the form $\lexists[x][!A]$ is true in
a !!{structure} only if \emph{there is an assignment} that assigns
objects of the \emph{(inner) domain} to each variable relative to
which $!A$ is true. Inclusive free logics allow !!{structure}s with an
empty (inner) domain. In those !!{structure}s \emph{there are no
assignments}, so formulas of the form $\lexists[x][!A]$ are false in
those !!{structure}s. 

\iftag{neuFrl}{\emph{Comment}. Even in neutral free logic
$\lexists[x][!A]$ is false when the domain is empty. For
$\lexists[x][!A]$ to be $\Undef$ in a neutral free logic, it must be
that \emph{there is an assigment} that assigns $x$ to some object
relative to which $!A$ is false. For instance, you may have an (inner)
domain with one individual, and $a$ undefined (or assigned to an
object of the outer domain). Then $\lexists[x][(\eq[a][x])]$ is
$\Undef$.}
\end{ans}
\end{prob}

\begin{prob}[Validity in negative and positive free logics]
For each of the formulas below, say whether it is valid in negative
free logic and whether it is valid in positive free logic. Briefly
justify your answers.
\begin{enumerate}
\item $\eq[a][a]$.\\
\emph{Example answer}. Positive: yes. Negative: no. In positive free
logic, $\eq[t_{1}][t_{2}]$ is true as long as $t_{1}$ and $t_{2}$ have
the same denotation, so $\eq[a][a]$ is guaranteed to hold. In negative
FL, $\eq[t_{1}][t_{2}]$ is false if the denotation of $t_{1}$ or
$t_{2}$ is in the outer domain, so $\eq[a][a]$ may fail. 
\item $\lforall[x][(\lfrexists x)]$. 
\item $\Atom{\Obj F}{a} \lif \lfrexists a$.
\item $\eq/[a][b] \lif \lnot\lfrexists a$.
\item $\lnot\lfrexists a \lif \eq/[a][b]$.
\item $\lforall[x][\Atom{\Obj G}{xb}] \lif (\lfrexists a\lif \Atom{\Obj G}{ab})$.
\item $\eq[a][b]\lif(\Atom{\Obj F}{a}\lif \Atom{\Obj F}{b})$.
\item $\lfrexists a\land\lnot \lfrexists b\lif\lnot \Atom{\Obj G}{ab}$.
\end{enumerate}
\begin{ans}
\begin{enumerate}
\item Positive: yes. Negative: no. In positive FL, $\eq[t_{1}][t_{2}]$
	is true as long as $t_{1}$ and $t_{2}$ have the same denotation,
	so $\eq[a][a]$ is guaranteed to hold. In negative FL,
	$\eq[t_{1}][t_{2}]$ is false if the denotation of $t_{1}$ or
	$t_{2}$ is in the outer domain, so $\eq[a][a]$ may fail. 
\item Valid in negative and positive FL. If the (inner) domain is
	empty, formulas of the form $\lforall[x][!A]$ are trivially true.
	If not, $\lforall[x][!A]$ if true if $!A$ is true relative to any
	assignement that assigns to $x$ and object of the domain (negative
	free logic) or inner domain (positive free logic). Since (in both
	free logics) $\lfrexists x$ is true relative to any assignment
	that assigns to $x$ an object of the domain (negative free logic)
	or inner domain (positive free logic, $\forall[x][(\lfrexists x)]$
	is true. So $\forall[x][(\lfrexists x)]$ is true on any
	!!{structure} of positive or negative free logic. 
\item Valid in negative FL, not valid in positive FL. In negative FL,
	$Fa$ is false if the  interpretation of $a$ is undefined, and
	$\lfrexists a$ is true if the interpretation of $a$ is defined. So
	$Fa\lif \lfrexists a$ is true in any !!{structure} of negative
	FL.\\ 
	In positive free logic, $Fa$ is true and $\lfrexists a$ false when
	the interpretation of  $a$ is an object of the \emph{outer} domain
	that is not in the inner domain (but within the interpretation of
	$F$). Hence $Fa\lif \lfrexists a$ is not true in all
	!!{structure}s of positive FL. 
\item Not valid in negative or positive FL. Consider !!a{structure} in
	which the  interpretations of $a$ and $b$ are distinct objects of
	the domain (negative free logic) or of the \emph{inner} domain
	(positive free logic). Then $\eq/[a][b]$ is true but $\lnot
	\lfrexists a$ is false. So the formula is not valid in either
	logic. 
\item Valid in negative FL, not valid in positive FL. In negative FL,
	either the interpretation of $a$ is defined, and $\lnot \lfrexists
	a$ is false, or it is undefined, and $\eq/[a][b]$ is true (no
	matter what the interpretation of $b$ is). So $\lnot \lfrexists
	a\lif\eq/[a][b]$ is true in any !!{structure} of negative
	FL.\\ 
	In positive FL, $\eq[a][b]$ are true in !!a{structure} in which
	the interpretation of $a$ and $b$ is the same object of the
	\emph{outer} domain that is not in the inner domain. If so, $\lnot
	\lfrexists a $ is true but $\eq/[a][b]$ is false. So the
	formula is not true in all !!{structure}s of positive FL.
\item Valid in negative and positive FL. Suppose $\lforall[x][Gxb]$ is
	true in some !!{structure} of these logics. Then $Gxb$ is true
	relative to any assignment that assigns $x$ to some object of the
	domain (negative free logic) or of the inner domain (positive free
	logic). If so, either the interpretation of $a$ is not an object
	of the (inner) domain, or $Gab$ is true. Either way, if
	$\lforall[x][Gxb]$ is true $\lfrexists a \lif Gab$ is true.
\item Valid in negative and positive FL. If $\eq[a][b]$ is true in a
	!!{structure}, then the interpretations of $a$ and of $b$ are both
	defined and the same object of the domain (negative free logic),
	or both the same object of the outer domain (positive free logic).
	Thus the interpretation of $a$ is in the interpretation of $F$
	just if the interpretation of $b$ is, hence $Fa \lif Fb$ is true.
	So there is no !!{structure} in which $\eq[a][b]$ is true and $Fa
	\lif Fb$ not true.
\item Valid in negative FL, not valid in positive FL.\\
	In negative FL, if $\lfrexists a \land \lnot \lfrexists b$ then
	the interpretation of $b$ is undefined. If so, $Gab$ is false.
	Hence $\lfrexists a \land \lnot \lfrexists b \lif Gab$ is true in
	any !!{structure} of negative FL.\\
	In positive FL, $\lfrexists a \land \lnot \lfrexists b$ is true in
	a $\Struct{M}$ then $\Assign{a}{M}$ is an object of the inner
	domain and $\Assign{b}{M}$ is an object of the outer domain that
	is not in the inner domain. Still, $\langle
	\Assign{a}{M},\Assign{b}{M} \rangle$ may be $\Assign{G}{M}$, in
	which case $Gab$ is true in $\Struct{M}$. So the formula is not
	true in all !!{structure}s of positive FL.
\end{enumerate}
\end{ans}
\end{prob}

\end{document}