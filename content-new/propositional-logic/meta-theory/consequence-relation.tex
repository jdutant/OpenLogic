% Part: propositional-logic
% Chapter: propositional-logic
% Section: consequence-relation

\documentclass[../../../include/open-logic-section]{subfiles}

\begin{document}

\olfileid{pl}{mth}{csq}

\olsection{Consequence Relation}

Logics define consequence relations. In propositional logic we have
two: semantic consequence ($\Entails$) and !!{derivability} ($\Proves$),
which can be shown to coincide (Soundness and Completeness).

In this section we consider consequence relations at a more general
level, using propositional logic as a concrete example.

\subsection{Consequence relation}

A logic can be thought of as a pair of a language $\Lang{L}$ and 
a \emph{consequence relation} $\Proves$ over that language.
The language $\Lang{L}$  is a set of !!{formula}s. The consequence relation is a
relation between sets of !!{formula}s and !!{formula}s.

For instance, in propositional logic, between the set
$\{\Obj{\pvar{A}},\Obj{\pvar{A}}\lif\Obj{\pvar{B}}\}$ and the
!!{formula} $\Obj{\pvar{B}}$, and between the empty set and the
formula $\Obj{\pvar{A}}\lor\lnot\Obj{\pvar{A}}$. 

Note that we use $\Proves$ for any consequence relation here: semantic 
consequence ($\Entails$) or !!{derivability} ($\Proves$ proper). 
In propositional logic and many others, the two coincide. 

We use two notations for consequence relations:
\begin{itemize}
    \item $\Gamma\Proves !A$ states that $!A$ is a consequence of 
    set of !!{formula}s $\Gamma$. When $\Gamma$ is the empty set, 
    we may simply write $\Proves !A$ instead of
    $\emptyset\Proves !A$.

    \item $\mathsf{Con}(\Gamma)$ denotes the set of all !!{formula}s that
    are consequences of !!{formula}s $\Gamma$:
    $$\mathsf{Con}(\Gamma)=\Setabs{!A\in\Lang{L}}{\Gamma\Proves !A}$$

\end{itemize}

\subsection{Main Structural Properties of Consequence Relations}

Consequence in propositional logic has three notable features:
\begin{enumerate}
    \item Reflexivity. $!A$ is a consequence of any premise set
    containing $!A$.
    \item Monotonicity. If $!A$ is a consequence of premise set $\Gamma$, 
    it's a consequence of any set $\Delta$ that includes $\Gamma$.
    \item Cut or Transitivity. If $!A$ is a consequence of consequences of
    a set $\Gamma$, it's a consequence of $\Gamma$. (That is, $!A$ is a 
    consequence of $\Delta$, and every !!{formula} in $\Delta$ is a 
    consequence of $\Gamma$, then $!A$ is a consequence of $\Gamma$.)
\end{enumerate}

These properties are called \emph{structural} because they can be 
described without reference to the syntax of $\Lang{L}$. (By contrast,
the property that if $\Gamma\Proves !A$, then $\Gamma\Proves !A\lor!B$
isn't structural because it makes reference to a specific connective, $\lor$.)

\begin{explain}
Monotonicity is a feature of \emph{deductive} arguments. With
inductive, abductive (inference to the best explanation) or
probabilistic reasoning, monotonicity fails. For instance, a
conclusion may \emph{be probable} given a premise $!A$ alone but not
probable given $!A$ and a further premise $!B$. In standard logics, by
contrast, consequence is monotonic: if a consequence already follows
from a set of premises, it also follows no matter what further
premises we add. (It is called `Monotonicity' because in mathematical
terms it can be expressed as the property that $\mathsf{Con}$ is a
monotonically increasing function of subsets of $\Lang{L}$.)

Cut means that intermediate steps are inessential. In ordinary reasoning,
we often derive intermediate conclusions ('lemmas') first, from which
we derive in term ultimate conclusions.We then say that the conclusions
follow directly from our initial hypotheses, not merely from the initial
hypotheses \emph{together with} the lemmas. 
\end{explain}

Here are Reflexivity, Monotoncity and Cut stated using the 
two notations for consequence:

\smallskip \noindent
{\small
\begin{tabular}{ccc}
        & $\Proves$ notation & $\mathsf{Con}$ notation \\
        \hline
    Reflexivity & If $!A\in\Gamma$ then $\Gamma\Proves !A$ & $\Gamma\subseteq\mathsf{Con}(\Gamma)$ \\
    Montonicity & If $\Gamma\Proves !A$
    $\Gamma\cup\Delta\Proves!A$ & If $\Gamma\subseteq\Delta$ then
    $\mathsf{Con}(\Delta)\subseteq\mathsf{Con}(\Gamma)$ \\
    Cut & If $\Gamma\Proves !A$ and $\Delta,!A\Proves !B$ then $\Gamma,\Delta\Proves !B$ & $\mathsf{Con}(\mathsf{Con}(\Gamma))=\Gamma$ \\
\end{tabular}
}
\smallskip

\begin{explain}
We could also have defined Reflexivity with:
\begin{quote}
    $\{!A\}\Proves !A$ for every $!A$. 
\end{quote}
The two definitions are equivalent given Monotonicity. 

We could also have defined Cut with: 
\begin{quote}
    If $\Gamma\Proves !A_i$ for each $!A_i$ in $\Delta$ and $\Delta\Proves !B$,
    then $\Gamma\Proves !B$. 
\end{quote}
The two definitions are equivalent given Reflexivity and Monotonicity.
\end{explain}

Reflexivity, Monotoncity and Cut aren't only features of consequence in
propositional logic. They are features of consequence of all standard
logics. Some logicians consider that a consequence relations can't
really be called logical if it doesn't have these features.

That being said, logicians do study systems without the structural 
rules (e.g. non-monotonic Logics, subtructural logics).

\subsection{The Substitution Rule}

The Substitution Rule (\olref[pl][mth][imp]{thm:subst-rule}) 
isn't structural, but it is another core property
of consequence in propositional logic that many consider essential 
for a consequence relation to be properly `logical'.

\subsection{Further Structural Properties}

Further structural properties of consequence relations are implicit in
the decision to treat premises as \emph{sets}. With sets order and
repetition don't matter. Therefore, when we say that a consequence
relation $\Proves$ is a relation between \emph{sets} of !!{formula}s
and !!{formula}s, we are committed to the idea that order and
repetition of premises don't matter. For instance, we cannot say that
$!A,!A,!B\Proves !C$ without accepting that $!B,!A\Proves !C$, for
$\{!A,!A,!B\}$ and $\{!B,!A\}$ are one and the same set. 

\smallskip\noindent
\begin{tabular}{cc}
    Repetition doesn't matter 
        & $\Gamma,!A,!A\Proves!B$ iff $\Gamma,!A\Proves!B$ \\
    Order doesn't matter
        & $\Gamma,!A,!B,\Delta\Proves!C$ iff $\Gamma,!B,!A,\Delta\Proves!C$\\
\end{tabular}

Again, these features are present in all standard logics.
\emph{Substructural} logics explore systems where they don't, i.e. 
where order and/or repetition matters, or even when $!A$ doesn't 
follow from $!A$. 

\end{document}



