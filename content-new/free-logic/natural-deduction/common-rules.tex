% Part: free-logic
% Chapter: natural-deduction
% Section: common-rules

\documentclass[../../../include/open-logic-section]{subfiles}

\begin{document}

\olfileid{frl}{nd}{cor}

\olsection{Common Rules}

Free logic is an alternative to (standard) first order logic with
identity. Its proof system replaces the rules of first order logic
with identity with weaker ones. There are two possible systems:
negative free logic \Log{NFL} and positive free logic \Log{PFL}. Both
share a common core.

\subsection{Overview}

\emph{Propositional Logic rules}. All rules of propositional logic are
included. To simplify !!{derivation}s, we may use a derived \Log{PL}
rule to skip purely propositional steps. 

\begin{defish}
    \emph{All propositional logic rules are allowed.}

    \emph{When $!B$ can be derived from $!A_1,\ldots,!A_n$ using
    propositional logic rules alone:}
        \begin{prooftree}
            \AxiomC{}\DeduceC{$!A_1$}
            \AxiomC{}\DeduceC{$\ldots$}
            \AxiomC{}\DeduceC{$!A_n$}
            \RightLabel{\Log{PL}}
            \TrinaryInfC{$!B$}
        \end{prooftree}
\end{defish}

\emph{Quantifier and Identity rules}
An overview of the common free logic rules for quantifiers and 
identity is in \olref{natdedfreecore}. These rules replace those 
of first order logic with identity. Only Leibniz's law (\Elim{\eq})
is unchanged.

Do not learn them rote. Rather, you should understand in each case why
the standard first order logic rule must be modified as it is. These
modifications are explained below. Train yourself to rediscover the
rules on your own a few times until you have learned them.

\begin{figure}
    \begin{defish}

        \emph{Free Logic Quantifier Rules}
       
        \smallskip\noindent
        \begin{tabular}{ll}
            \AxiomC{}\DeduceC{$\lfrexists c$}
            \AxiomC{}\DeduceC{$\lforall[v][!A]$}
            \RightLabel{\Elim{\lforall} \Log{FL}}
            \BinaryInfC{$!A[c/v]$}
            \DisplayProof
        &
            \AxiomC{}\DeduceC{$\lfrexists c$}
            \AxiomC{}\DeduceC{$!A[c/v]$}
            \RightLabel{\Intro{\lexists} \Log{FL}}
            \BinaryInfC{$\lexists[v][!A]$}
            \DisplayProof
        \end{tabular} 
        
        \smallskip\noindent
        {\setlength\extrarowheight{3em} 
        \begin{tabular}{cp{10em}}
            \AxiomC{$\Discharge{\lfrexists c}{n}$}\noLine
            \UnaryInfC{$\mathcal{D}$}\noLine
            \UnaryInfC{$!A[c/v]$}
            \DischargeRule{\Intro{\lforall} \Log{FL}}{n}
            \UnaryInfC{$\lforall[v][!A]$}
            \DisplayProof
        
        &   
            \emph{Restrictions:}
        
            - $c$ not in undischarged assumptions of $\mathcal{D}$
                other than $\lfrexists c$.
        
            - $c$ not in $!A$
        
        \\
        
            \AxiomC{$\lexists[v][!A]$}
                \AxiomC{$\Discharge{\lfrexists c}{n},\Discharge{!A[c/v]}{n}$}
                \noLine
                \UnaryInfC{$\mathcal{D}$}
                \UnaryInfC{$!B$}
            \DischargeRule{\Elim{\lexists} \Log{FL}}{n}
            \BinaryInfC{$!B$}
            \DisplayProof
        
        & 
            \emph{Restrictions:}
        
            - $c$ no in undischarged assumptions of $\mathcal{D}$
              other than $!A[c/v]$ and $\lfrexists c$.
        
            - $c$ not in $!A$
        
            - $c$ not in $!B$
        
        \\
            
        \end{tabular}
        }   

        \smallskip\noindent
        \emph{Free Logic Identity Rules}
       
        \smallskip\noindent
        \begin{tabular}{ll}
            \AxiomC{$\lfrexists c$}
            \RightLabel{\Intro{\eq} \Log{FL}}
            \UnaryInfC{$\eq[c][c]$}
            \DisplayProof
            &
            \AxiomC{$\eq[c][d]$}
                \AxiomC{$!A[c/v]$}
            \RightLabel{\Elim{\eq}}
            \BinaryInfC{$!A[d/v]$}
            \DisplayProof                
        \end{tabular}

    \end{defish}
    \caption{Natural Deduction rules common to \Log{NFL} and \Log{PFL}.}
    \ollabel{natdedfreecore}
\end{figure}

We also the following derived rules for $\lfrexists$. They allow you 
to replace $\lfrexists c$ by $\lexists[v][\eq[c][v]]$ and conversely.

\begin{defish}
    \begin{tabular}{ll}
        \AxiomC{$\lfrexists c$}
        \RightLabel{\Elim{\lfrexists}}
        \UnaryInfC{$\lexists[v][\eq[c][v]]$}
        \DisplayProof
        &
        \AxiomC{$\lexists[v][\eq[c][v]]$}
        \RightLabel{\Intro{\lfrexists}}
        \UnaryInfC{$\lfrexists c$}
        \DisplayProof
    \end{tabular}
\end{defish}

\subsection{Explanation of the new quantifier rules}

\subsubsection{Free Universal Instantiation}

In standard first order logic, we may instantiate $\lforall[x][\Obj{F}x]$
directly with $\Obj{F}c$, for any constant $c$. In Free Logic, that 
is not enough. We need the extra premise that $c$ exists, $\lfrexists c$.

This explains our \Elim{\lforall} \Log{FL} rule.
\begin{prooftree}
\AxiomC{}\DeduceC{$\lfrexists c$}
\AxiomC{}\DeduceC{$\lforall[v][!A]$}
\RightLabel{\Elim{\lforall} \Log{FL}}
\BinaryInfC{$!A[c/v]$}
\end{prooftree}

\subsubsection{Free Existential Generalization}

In standard first order logic, to show that $\lexists[x][\Obj{F}x]$ 
it is enough to derive $\Obj{F}c$ for some constant $c$. 

In free logic, that is enough. We also need that $c$ exists. To see
this, recall that free logic allows empty constants. If $c$ is empty,
the claim that nothing is $c$ is true:
$$\lnot\lexists[x][\eq[c][x]]$$
If we applied the standard first order logic \Intro{\lexists} to this
we would derive the claim that something is nothing:
$$\lexists [y][\lnot\lexists[x][\eq[x][y]]]$$
Which is a contradiction (in both standard and free logics).

Therefore, in free logic, to show that $\lexists[x][\Obj{F}x]$ it is
enough to derive $\Obj{F}c$ \emph{and $\lfrexists c$} from 

This explains our \Intro{\lexists} \Log{FL} rule.
\begin{prooftree}
    \AxiomC{}\DeduceC{$\lfrexists c$}
    \AxiomC{}\DeduceC{$!A[c/v]$}
    \RightLabel{\Intro{\lexists} \Log{FL}}
    \BinaryInfC{$\lexists[v][!A]$}
\end{prooftree}

\subsubsection{Free Universal Generalization}

In standard first order logic, to show that $\lforall[x][\Obj{F}x]$ 
you need to !!{derive} $\Obj{F}x$ for \emph{any} $c$. To ensure
that $c$ is arbitrary we require that no assumption is made about it---hence,
it should not appear free in undischarged assumptions of your derivation. 

In free logic, to show that $\lforall[x][\Obj{F}x]$ 
you need to !!{derive} $\Obj{F}x$ for \emph{any} $c$ \emph{that exists}.
To ensure that $c$ is arbitrary we require that no assumption is made
about it \emph{except that it exists}. Hence, $c$ might appear in an 
undiscarged assumption $\lfrexists c$, but no other. 

That explains our \Intro{\lforall} \Log{FL} rule.

\smallskip\noindent
\begin{tabular}{cp{10em}}
    \AxiomC{$\Discharge{\lfrexists c}{n}$}\noLine
    \UnaryInfC{$\mathcal{D}$}\noLine
    \UnaryInfC{$!A[c/v]$}
    \DischargeRule{\Intro{\lforall} \Log{FL}}{n}
    \UnaryInfC{$\lforall[v][!A]$}
    \DisplayProof

&   
    \emph{Restrictions:}

    - $c$ not in undischarged assumptions of $\mathcal{D}$
        other than $\lfrexists c$.

    - $c$ not in $!A$
\end{tabular}

\subsubsection{Free Existential Instantiation}

In standard first order logic, the !!{formula} $\lexists[v][\Obj{F}v]$ 
states that something is $F$. To derive some conclusion $!B$ from this
claim, we need to show that $!B$ can be derived from the mere
assumption that some \emph{arbitrary} $c$ is $F$, $!A[c/v]$.

In free logic, it is enough to derive our conclusion from the
assumption that some arbitrary \emph{existing} $c$ is $F$: the
assumptions $\lfrexists c$ \emph{and} $!A[c/v]$. In standard first
order logic, the supposition that $!A[c/v]$ holds for some $c$ already
implicitly assumes that $c$ exists. In free logic, that a certain 
!!{formula} $!A[c/v]$ holds doesn't tell us that $c$ exists. Hence
in free logic the mere assumption that some
$c$ is $F$ doesn't fully capture the force of the premise
$\lexists[v][\Obj{F}v]$. That premise not only tells us that 
that $!A[c/v]$ holds for some $c$ or other, but also that it holds 
for some $c$ or other \emph{that exists}. Hence our rule says 
that we can derive $!B$ from $\lexists[v][\Obj{F}v]$ provided 
that we can derive it from the assumption that $!A[c/v]$ holds for 
some arbitrary \emph{existing} $c$.

Hence our \Elim{\lexists} \Log{FL} rule:

\smallskip\noindent
{\setlength\extrarowheight{3em} 
\begin{tabular}{cp{10em}}
   \AxiomC{$\lexists[v][!A]$}
        \AxiomC{$\Discharge{\lfrexists c}{n},\Discharge{!A[c/v]}{n}$}
        \noLine
        \UnaryInfC{$\mathcal{D}$}
        \UnaryInfC{$!B$}
    \DischargeRule{\Elim{\lexists} \Log{FL}}{n}
    \BinaryInfC{$!B$}
    \DisplayProof

& 
    \emph{Restrictions:}

    - $c$ no in undischarged assumptions of $\mathcal{D}$
      other than $!A[c/v]$ and $\lfrexists c$.

    - $c$ not in $!A$

    - $c$ not in $!B$

\\
    
\end{tabular}
}   

\subsection{Deriving the $\lfrexists$ rules}

The rules showing that  $\lfrexists c$ is equivalent to
$\lexists[v][\eq[c][v]]$ are derived from the quantifier and identity
rules as follows.

\smallskip\noindent
\begin{tabular}{ll}

\AxiomC{$\lfrexists c$}
    \AxiomC{$\lfrexists c$}
    \RightLabel{\Intro{\eq} \Log{FL}}
    \UnaryInfC{$\eq[c][c]$}
\RightLabel{\Intro{\lexists} \Log{FL}}
\BinaryInfC{$\lexists[v][\eq[c][v]]$}
\DisplayProof
&
\AxiomC{$\lexists[v][\eq[c][v]]$}
    \AxiomC{$\Discharge{\lfrexists d}{1}$}
        \AxiomC{$\Discharge{\eq[c][d]}{1}$}
    \RightLabel{\Elim{\eq}}
    \BinaryInfC{$\lfrexists c$}
\DischargeRule{\Elim{\lexists} \Log{FL}}{1}
\BinaryInfC{$\lfrexists c$}
\DisplayProof    
\end{tabular}
\smallskip

The \Elim{\lexists} \Log{FL} is restricted. We check that the 
restriction are obeyed. We chose to instantiate $\lexists[v][\eq[c][v]]$
with $\eq[d][v]$ (picking the new variable $d$) and:
\begin{itemize}
    \item $d$ doesn't appear in undischarged assumptions of the 
    sub-!!{derivation} on the right other than $\lfrexists d$ and
    $\eq[c][d]$.
    \item $d$ doesn't appear in the instantiated !!{formula} $\eq[c][v]$.
    \item $d$ doesn't appear in our conclusion $\lfrexists c$.
\end{itemize}
Therefore the step is legitimate.

\end{document}