% Part: cheat-sheets
% Chapter: modal-propositional-logics
% Section: natural-deduction

\documentclass[../../../../include/open-logic-section]{subfiles}

\begin{document}

\olfileid{chs}{pml}{nd}

\olsection{Natural Deduction}

\subsection{Rules for \Log{K}}

\begin{figure}
\begin{defish}
\begin{center}
    {\setlength\extrarowheight{2em}
    \begin{tabular}{cc}
        \multicolumn{2}{c}{
            \AxiomC{}\DeduceC{$\Box!A$}
            \AxiomC{}\DeduceC{$\Box!B$}
            \AxiomC{$\Discharge{!A}{n},\Discharge{!B}{n}$ \emph{at
            most}}
            \DeduceC{$!C$} 
            \DischargeRule{$\Box$\Ax{K}}{n}
            \TrinaryInfC{$\Box!C$} 
            \DisplayProof
        }
    \\
        \AxiomC{}\DeduceC{$\lnot\Box\lnot!A$}
        \RightLabel{\Intro{\Diamond}K}
        \UnaryInfC{$\Diamond!A$}
        \DisplayProof
    &
        \AxiomC{}\DeduceC{$\Diamond!A$}
        \RightLabel{\Elim{\Diamond}K}
        \UnaryInfC{$\lnot\Box\lnot!A$}
        \DisplayProof
    \end{tabular}
    \\[1em] \noindent
    \emph{You may also use the derived rules:}
    \\[1em] \noindent
        \begin{tabular}{cccc}
        \AxiomC{}\DeduceC{$\lnot\Diamond!A$}
        \RightLabel{Dual}
        \UnaryInfC{$\Box\lnot!A$}
        \DisplayProof
    &
        \AxiomC{}\DeduceC{$\Box\lnot!A$}
        \RightLabel{Dual}
        \UnaryInfC{$\lnot\Diamond!A$}
        \DisplayProof
    &
        \AxiomC{}\DeduceC{$\Diamond\lnot!A$}
        \RightLabel{Dual}
        \UnaryInfC{$\lnot\Box!A$}
        \DisplayProof
    &
        \AxiomC{}\DeduceC{$\Box\lnot!A$}
        \RightLabel{Dual}
        \UnaryInfC{$\lnot\Diamond!A$}
        \DisplayProof
    \\[3em]
        \AxiomC{}\DeduceC{$\lnot\Diamond\lnot!A$}
        \RightLabel{Dual}
        \UnaryInfC{$\Box!A$}
        \DisplayProof
    &
        \AxiomC{}\DeduceC{$\Box!A$}
        \RightLabel{Dual}
        \UnaryInfC{$\lnot\Diamond\lnot!A$}
        \DisplayProof

    \end{tabular}

    \bigskip
    \AxiomC{}\DeduceC{$\Box!A_1$}
        \AxiomC{}\DeduceC{$\ldots$}
        \AxiomC{}\DeduceC{$\Box!A_k$}
        \AxiomC{$\Discharge{!A_1}{n},\ldots,\Discharge{!A_k}{n}$ \emph{at most}}\DeduceC{$!B$}
        \DischargeRule{$\Box$\Ax{K}}{n}
        \QuaternaryInfC{$\Box!B$}
    \DisplayProof

    \bigskip
    \AxiomC{}\DeduceC{$\Diamond!A$}
    \AxiomC{}\DeduceC{$\Box!B_1$}
    \AxiomC{}\DeduceC{\ldots}
    \AxiomC{}\DeduceC{$\Box!B_k$}
    \AxiomC{$\Discharge{!A}{n},\Discharge{!B_1}{n},\ldots,\Discharge{!B_k}{n}$ \emph{at most}}\DeduceC{$!C$}
    \DischargeRule{$\Diamond$\Ax{K}}{n}
    \QuinaryInfC{$\Diamond!C$}
    \DisplayProof

    \bigskip
    \begin{tabular}{cc}
    \AxiomC{}
    \DeduceC{$\Diamond\lfalse$}
    \RightLabel{$\Diamond\lfalse$}
    \UnaryInfC{$\lfalse$}
    \DisplayProof
    &
    \AxiomC{}
    \DeduceC{$\Diamond\lfalse$}
    \RightLabel{$\Diamond\lfalse$}
    \UnaryInfC{$!A$}
    \DisplayProof
    \end{tabular}
    }
\end{center}
\end{defish}
\caption{Natural Deduction rules for Propositional Modal Logic \Log{K}.}
\ollabel{natdedpmlk}
\end{figure}

The \Log{K} rules are given in \olref{natdedpmlk}. 

In  $\Box$\Ax{K} on the right has \emph{at most} $!A,!B$. It can be 
applied without one or without any too. Thus, these are correct
applications of $\Box$\Ax{K}:

\bigskip
\begin{tabular}{cc}
    \AxiomC{}\DeduceC{$\Box!A$}
    \AxiomC{$\Discharge{!A}{n}$ \emph{at
    most}}
    \DeduceC{$!C$} 
    \DischargeRule{$\Box$\Ax{K} {\color{blue}Correct}}{n}
    \BinaryInfC{$\Box!C$} 
    \DisplayProof
&
    \AxiomC{\emph{no undischarged assumption}}
    \DeduceC{$!C$}
    \RightLabel{$\Box$\Ax{K} {\color{blue}Correct}}
    \UnaryInfC{$\Box!C$} 
    \DisplayProof
\end{tabular}
\bigskip

\olref{natdedpmlk} also provides derived rules. Even though you do not 
strictly need them, some !!{derivation}s are very long without them, 
so I recommend learning them too. 

In the $\Diamond$\Ax{K} rule, the $\Box$ premise is optional but 
\emph{the $\Diamond$ premise is mandatory}. Therefore the first 
application of $\Diamond$\Ax{K} below is correct but the second isn't:

\bigskip\noindent
\begin{tabular}{cc}
     \AxiomC{$\Diamond!A$}
        \AxiomC{$\Discharge{!A}{n}$ \emph{at most}}
        \DeduceC{$!B$}
    \DischargeRule{$\Diamond$\Ax{K} {\color{blue}Correct}}{n}
    \BinaryInfC{$\Diamond!B$}
    \DisplayProof
&
    \AxiomC{\emph{no undisch. as.}}
    \DeduceC{$!B$}
    \RightLabel{{\color{red}Incorrect}}
    \UnaryInfC{$\Diamond!B$}
    \DisplayProof
\end{tabular}
\bigskip

\noindent(The derivation on the right would be 
correct in system \Log{D} and stronger, but using the \Ax{D} rule; see
below.)

The Dual (Duality) rules are particularly useful. They all apply the same
procedure: move a negation `through' a modal operator and switch the
main modal operator of a !!{formula}, and eliminate a double
negation if it arises. Thus $\Box\lnot$ can be converted
to $\lnot\Diamond$ and conversely, and $\lnot\Diamond\lnot$ can be
converted to $\Box$. 

Beware that the `Dual' rules only allow you to do these substitutions on the 
\emph{main} modal operator of a formula, not in \emph{sub}-!!{formula}s.
Thus the following are incorrect:

\bigskip \noindent
\begin{tabular}{cc}
\AxiomC{$!A\lor\Box\lnot!B$}
\RightLabel{Dual? {\color{red}Incorrect}}
\UnaryInfC{$!A\lor\lnot\Diamond!B$}
\DisplayProof
&
\AxiomC{$\Box\Diamond\lnot!B$}
\RightLabel{Dual? {\color{red}Incorrect}}
\UnaryInfC{$\Box\lnot\Box!B$}
\DisplayProof
\end{tabular}
\bigskip

In both cases the conclusion does follow from the premise, but 
not directly by a `Dual' rule. Instead one must `unpack' the premises:

\begin{prooftree}
\AxiomC{$!A\lor\Box\lnot!B$}
    \AxiomC{$\Discharge{!A}{1}$}
    \AxiomC{$\Discharge{\Box\lnot!B}{1}$}
    \RightLabel{Dual {\color{blue}Correct}}
    \UnaryInfC{$\lnot\Diamond!B$}
\DischargeRule{\Elim{\lor}}{1}
\TrinaryInfC{$!A\lor\lnot\Diamond!B$}
\end{prooftree}
\begin{prooftree}
\AxiomC{$\Box\Diamond\lnot!B$}
    \AxiomC{$\Discharge{\Diamond\lnot!B}{1}$}
        \RightLabel{Dual {\color{blue}Correct}}
        \UnaryInfC{$\lnot\Box!B$}
    \DischargeRule{$\Box$\Ax{K}}{1}
\BinaryInfC{$\Box\lnot\Box!B$}
\end{prooftree}

\subsection{Rules for Stronger Systems}

All others systems include rules for \Log{K}, plus one or more of the
rules in \olref{natdedpmlx}.

\begin{figure}
    \begin{defish}
    \begin{center} 
    {\setlength\extrarowheight{3em} 
    \begin{tabular}{ll}
        \AxiomC{\emph{no undischarged assumption}}\DeduceC{$!C$}
        \RightLabel{\Intro{\Diamond}\Ax{D}}
        \UnaryInfC{$\Diamond!C$}
        \DisplayProof
    &
        \AxiomC{}\DeduceC{$\Box!A$}
        \RightLabel{\Ax{D}}
        \UnaryInfC{$\Diamond!A$}
        \DisplayProof
    \\
        \AxiomC{}\DeduceC{$\Box!A$}
        \RightLabel{$\Box$\Ax{T}}
        \UnaryInfC{$!A$}
        \DisplayProof
    &   \AxiomC{}\DeduceC{$!A$}
        \RightLabel{$\Diamond$\Ax{T}}
        \UnaryInfC{$\Diamond!A$}
        \DisplayProof
    \\
        \AxiomC{}\DeduceC{$\Box!A$}
        \RightLabel{$\Box$\Ax{4}}
        \UnaryInfC{$\Box\Box!A$}
        \DisplayProof
    &   \AxiomC{}\DeduceC{$\Diamond\Diamond!A$}
        \RightLabel{$\Diamond$\Ax{4}}
        \UnaryInfC{$\Diamond!A$}
        \DisplayProof
    \\
        \AxiomC{}\DeduceC{$!A$}
        \RightLabel{$\Box$\Ax{B}}
        \UnaryInfC{$\Box\Diamond!A$}
        \DisplayProof
    &   \AxiomC{}\DeduceC{$\Diamond\Box!A$}
        \RightLabel{$\Diamond$\Ax{B}}
        \UnaryInfC{$!A$}
        \DisplayProof
    \\
        \AxiomC{}\DeduceC{$\Diamond!A$}
        \RightLabel{$\Box$\Ax{5}}
        \UnaryInfC{$\Box\Diamond!A$}
        \DisplayProof
    &   \AxiomC{}\DeduceC{$\Diamond\Box!A$}
        \RightLabel{$\Diamond$\Ax{5}}
        \UnaryInfC{$\Box!A$}
        \DisplayProof

    \end{tabular}
    }
    \end{center}
    \end{defish}

    \caption{Natural Deduction rules for stronger modal logics.}
\ollabel{natdedspmlx}
\end{figure}

Some systems are often
discussed and have a special name, namely \Log{D}, \Log{T}, \Log{B},
\Log{S4}, \Log{S5}. You should know these names and know the rules
available in each, given in \olref{natdedsrules}.

\begin{figure}
    \begin{tabular}{c|p{20em}|c}
        \hline
        System  & Accessibility is ... & Add rule(s) ... \\
        \hline
        \Log{D} & serial & \Ax{D} \\
        \Log{T} & serial, reflexive & \Ax{D},\Ax{T} \\
        \Log{B} & serial, reflexive and symmetric & \Ax{D},\Ax{T},\Ax{B} \\
        \Log{S4}  & serial, reflexive and transitive & \Ax{D},\Ax{T},\Ax{4} \\
        \Log{S5} & serial, reflexive, transitive, symmetric, euclidean & \Ax{D},\Ax{T},\Ax{B},\Ax{4},\Ax{5} \\
        \hline
    \end{tabular}
\caption{Standard systems \Log{D}, \Log{T}, \Log{B}, \Log{S4},
\Log{S5}.}
\ollabel{natdedsrules}
\end{figure}

As \olref{natdedsrules} makes clear, each property of the
accessibility relation corresponds to one set of rules, e.g. the
\Ax{D} set for seriallity, the \Ax{T} set for reflexivity and so on.
That being said, some properties entail others: if a relation is
reflexive then it is serial. Thus, we usually define \Log{B} as having
a reflexive and symmetric accessibility relation; no need to mention
that it's serial. System \Log{S5} has many equivalent definitions
because several combinations entail others, e.g. reflexive and
euclidean entails symmetric and transitive. 

More generally, systems can be named by stating their rules. \Log{KD45}
is the system that extends \Log{K} with rules \Ax{D},\Ax{4},\Ax{5}. It 
is suitable for models where the accessibility relation is serial, 
transitive, and euclidean (but not necessarily reflexive). This system
turns out not to be equivalent to the ones we know already. It doesn't 
have a special name other than \Log{KD45}, but it is often used to model belief. 

Our standard systems can thus be given more explicit names:

\bigskip
\begin{tabular}{ccc}
\hline 
System & Fully explicit name & Equivalent names \\
\hline
\Log{D} & \Log{KD} &  \\
\Log{T} & \Log{KDT} & \Log{KT} \\
\Log{B} & \Log{KDTB} & \Log{KTB}, \Log{KDB} \\
\Log{S4}  & \Log{KDT4} & \Log{KT4} \\
\Log{S5} & \Log{KDTB45} & \Log{KT5}, \Log{KTB4}, \Log{KDB4}, \dots \\
\hline    
\end{tabular}
\bigskip

Why the equivalent names on the last column? Corresponding to the fact 
that some properties of accessibility, or
combinations thereof, entail others, some rules and combination of
rules entail others, as detailed in the table below:

\bigskip\noindent
\begin{tabular}{cccc}
\hline
If accessibility is\dots & it's also\dots & so from rule(s)\dots & we derive rule(s)\dots \\
\hline 
reflexive & serial & \Ax{T} & \Ax{D} \\
serial + symmetric & reflexive &  \Ax{D}+\Ax{B} & \Ax{T} \\
symmetric + transitive & euclidean & \Ax{B}+\Ax{4} & \Ax{5} \\
reflexive + euclidean & sym. + trans. & \Ax{T}+\Ax{5} & \Ax{B},\Ax{4} \\
\hline 
\end{tabular}
\bigskip

This explains why \Log{B}, i.e. \Log{KDTB}, can also be called
\Log{KTB} (since \Ax{T} gives you \Ax{D}) and \Log{KDB} (since \Ax{D}
and \Ax{B} together give you \Ax{T}). 

\emph{You do not need to remember all these facts}, expect the
simplest one: reflexive entails serial. And \emph{you do not need to
remember all equivalent names}. For the exam, whay you need to be able
to do is:
\begin{itemize}
    \item Know which rules are applicable (or not) in each of the 
    standard systems \Log{K}, \Log{D}, \Log{T}, \Log{B},
    \Log{S4}, \Log{S5}. For instance, you should know
    that \Log{S4} is \Log{KDT4}, i.e. that it has rules \Ax{T} (hence
    \Ax{D}) and \Ax{4}, not rules \Ax{B} nor \Ax{5}.
    \item Understand explicit names, e.g. understand that \Log{KD45}
    is the system that extends \Log{K} with rules \Ax{D},\Ax{4},\Ax{5}.
    \item Be aware that different explicit names can turn out to be 
    the same system. For instance, \Log{KDT} isn't a different system 
    than \Log{KT}: they're both just the system \Log{T}.
    \item Be able to \emph{explain} the facts in the table above. Namely:
    \begin{itemize}
    \item Explain why a given (combination of) propertie(s) of accessibility
    entails another. For instance, you could be asked to explain why 
    an accessibility relation that is symmetric is serial is also reflexive.
    \item Derive some rules from others. For instance, give a !!{derivation}
    of the \Ax{B} rule from the \Ax{5} and \Ax{T} rules.
    \end{itemize}
    Examples of such exercises are given in the modal logic chapters
    of the KCL Philosophical Logic textbook.   
\end{itemize}

\end{document}