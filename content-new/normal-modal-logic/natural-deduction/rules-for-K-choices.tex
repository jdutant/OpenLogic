% Part: normal-modal-logic
% Chapter: natural-deduction 
% Section: rules-for-K-choices.tex

\documentclass[../../../include/open-logic-section]{subfiles}

\begin{document}

\olfileid{nml}{nd}{rkc}

\olsection{Note on the choice of rules for \Log{K}}

Our \Elim{\Diamond} and \Intro{\Diamond} rules are 
Duality rules are sound and sufficient for completeness. But they're
not entirely satisfactory: ideally, we'd like to have rules that
define $\Diamond$ on its own, not in terms of another operator
($\Box$). Can we do better?

Consider the $\Diamond$\Ax{K} and $\lfalse\Diamond$ rules:

\bigskip \noindent
{\def\defaultHypSeparation{\hskip .03in}
\begin{tabular}{cc}
\AxiomC{}\DeduceC{$\Diamond!A$}
\AxiomC{}\DeduceC{$\Box!B_1$}
\AxiomC{}\DeduceC{\ldots}
\AxiomC{}\DeduceC{$\Box!B_k$}
\AxiomC{$\Discharge{!A}{n},\Discharge{!B_1}{n},\ldots,\Discharge{!B_k}{n}$ \emph{at most}}\DeduceC{$!C$}
\DischargeRule{$\Diamond$\Ax{K}}{n}
\QuinaryInfC{$\Diamond!C$}
\DisplayProof
&
\AxiomC{}
\DeduceC{$\Diamond\lfalse$}
\RightLabel{$\Diamond\lfalse$}
\UnaryInfC{$\lfalse$}
\DisplayProof
\end{tabular}
}

\noindent Using these two rules as primitives instead, we could derive the
\Elim{\Diamond} rule like so:

\begin{prooftree}
\AxiomC{}\DeduceC{$\Diamond!A$}
\AxiomC{}\DeduceC{$\Discharge{\Box\lnot!A}{2}$}
    \AxiomC{$\Discharge{!A}{1}$}
    \AxiomC{$\Discharge{\lnot!A}{1}$}
    \RightLabel{\Intro{\lfalse}}
    \BinaryInfC{$\lfalse$}
\DischargeRule{$\Diamond$\Ax{K}}{1}
\TrinaryInfC{$\Diamond\lfalse$}
\RightLabel{$\lfalse\Diamond$}
\UnaryInfC{$\lfalse$}
\DischargeRule{$\lfalse_C$}{2}
\UnaryInfC{$\lnot\Box\lnot!A$}
\end{prooftree}

However, we would not be able to derive the \Intro{\Diamond} rule.
For the same reason, we would only be able to derive half of
the Duality rules.

\begin{prob}
Suppose we keep the semantic clause for $\Box$ as is, but change the 
semantic clause for $\Diamond$ to:
\begin{itemize}
    \item $\mSat{M}{\Diamond!A}[w]$ iff there are \emph{two distinct} $w',w''\in W$
    such that $Rww'$, $Rww''$ and $\mSat{M}{!A}[w']$ and $\mSat{M}{!A}[w'']$.
\end{itemize}
Answer the following:
\begin{enumerate}
    \item Show that with that semantics,
    $\Diamond!A\liff\lnot\Box\lnot!A$ (Duality) is not valid. Say
    whether both directions fail ($\Diamond!A\lif\lnot\Box\lnot!A$ and
    $\lnot\Box\lnot!A\lif\Diamond!A$)
    \item Show that with that semantics, the $\Diamond$\Ax{K} and
    $\lfalse\Diamond$ rules are sound.

    \bigskip \noindent
    {\def\defaultHypSeparation{\hskip .03in}
    \begin{tabular}{cc}
    \AxiomC{}\DeduceC{$\Diamond!A$}
    \AxiomC{}\DeduceC{$\Box!B_1$}
    \AxiomC{}\DeduceC{\ldots}
    \AxiomC{}\DeduceC{$\Box!B_k$}
    \AxiomC{$\Discharge{!A}{n},\Discharge{!B_1}{n},\ldots,\Discharge{!B_k}{n}$ \emph{at most}}\DeduceC{$!C$}
    \DischargeRule{$\Diamond$\Ax{K}}{n}
    \QuinaryInfC{$\Diamond!C$}
    \DisplayProof
    &
    \AxiomC{}
    \DeduceC{$\Diamond\lfalse$}
    \RightLabel{$\Diamond\lfalse$}
    \UnaryInfC{$\lfalse$}
    \DisplayProof
    \end{tabular}
    }

    \item Using the previous result, explain why the \Intro{\Diamond}
    rule below cannot be derived from the the $\Box$\Ax{K}, 
    $\Diamond$\Ax{K} and $\lfalse\Diamond$ rules.
    \begin{prooftree}
        \AxiomC{}\DeduceC{$\lnot\Box\lnot!A$}
        \RightLabel{\Intro{\Diamond}K}
        \UnaryInfC{$\Diamond!A$}
    \end{prooftree}
    
\end{enumerate}
\end{prob}

\end{document}
