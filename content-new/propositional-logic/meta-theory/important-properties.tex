% Part: propositional-logic
% Chapter: propositional-logic
% Section: important-properties

\documentclass[../../../include/open-logic-section]{subfiles}

\begin{document}

\olfileid{pl}{mth}{imp}

\olsection{Important Properties}

This section recaps important properties of propositional logic.

\subsection{The substitution rule}

This property captures a sense in which validity is `formal'.
$\pvar{A}\lor\lnot\pvar{A}$ is valid; but so is any formula of `the
same form', $!A\lor\lnot!A$. (What's the difference? $\pvar{A}$ is a
specific atomic sentence; $!A$ is any formula, including complex
ones.) What does `the same form' mean here? Well, any formula that is
like the original $\pvar{A}\lor\lnot\pvar{A}$, where the sentence
letter $\pvar{A}$ is uniformly replaced by some formula (possibly
complex). 

\begin{thm}[Substitution rule]
    \ollabel{thm:subst-rule}
    Let $!A[!B_1/\Obj{\pvar{A}},\ldots,!B_n/\Obj{\pvar{A}_j}]$ is the
    result of uniformly replacing !!{propositional variable}s
    $\Obj{\pvar{A}}_i,\ldots,\Obj{\pvar{A}}_j$ with formulas,
    respectively, and let
    $\Gamma[!B_1/\Obj{\pvar{A}},\ldots,!B_n/\Obj{\pvar{A}_j}]$ be the
    set of formulas resulting of doing so for each formula in
    $\Gamma$. We have:
    \begin{itemize}
        \item If $\Entails !A$, $\Entails
        !A[!B_1/\Obj{\pvar{A}},\ldots,!B_n/\Obj{\pvar{A}_j}]$.
        \item If $\Gamma \Entails !A$,
        $\Gamma[!B_1/\Obj{\pvar{A}},\ldots,!B_n/\Obj{\pvar{A}_j}]
        \Entails
        !A[!B_1/\Obj{\pvar{A}},\ldots,!B_n/\Obj{\pvar{A}_j}]$.
    \end{itemize}
\end{thm}

\subsection{Semantic deduction theorem}

This property connects validity of an argument and logical truth of
a corresponding conditional. 

\begin{thm}
    \begin{thm}[Semantic Deduction Theorem]
        \ollabel{thm:sem-deduction} $\Gamma \Entails !A \lif !B$ if and only
        if $\Gamma \cup \{!A\} \Entails !B$.
      \end{thm}
\end{thm}

This property is closely related to rules of proof for the conditional.
\emph{Modus ponens} is sound only if the left-to-right version of
the deduction theorem holds. \emph{Conditional proof} is sound only
if the right-to-left version is sound. 

This property is important because some weaker alternatives to classical
logic don't have it; in those logics, at least one of modus ponens
or conditional proof ceases to be usable for reasoning with the conditional.

\subsection{Soundness and Completeness}

\begin{thm}
    Propositional logic is sound and complete. The argument from $\Gamma$
    to $!A$ is valid if and only if provable.
    \begin{itemize}
        \item Soundness: if an argument is provable, it is valid. If
        $\Gamma \Proves !A$, then $\Gamma \Entails !A$. 
        \item Completeness: if an argument is valid, it is provable.
        If $\Gamma \Entails !A$, then $\Gamma \Proves !A$.
    \end{itemize} 
\end{thm}

Given soundness and completeness, the properties below hold for both
notions of consequence.


\subsection{Compactness}

\begin{prop}[Compactness Theorem]
    Propositional logic is \emph{compact}. Two equivalent formulations:
    \begin{enumerate}
        \item $\Gamma$ is satisfiable if and only if every \emph{finite} subset
    $\Gamma_0$ of $\Gamma$ is satisfiable. 
        \item If $\Gamma$ entails $!A$ then some finite subset of
        $\Gamma$ entails $!A$. If $\Gamma \Entails !A$ then there are
        $!B_1,\ldots,!B_n$ in $\Gamma$ such that $!B_1,\ldots,!B_n
        \Entails !A$.
    \end{enumerate}
\end{prop}

If a logic \emph{lacks} compactness, then some of its valid arguments
are \emph{essentially infinitary}. That is, in that logic, some
arguments with infinitely many premises are valid, but no argument
using any finite subset of those premises to reach the same conclusion
is valid. The best known example of infinitary consequence is:
\begin{quote}
There is at least $1$ thing.

There are at least $2$ things.

There are at least $3$ things.

\dots

There are at least $n$ things.

\dots

Therefore, there are infinitely many things.

\end{quote}
The conclusion follows from \emph{all} the premises of the form `there
are at least $n$ things', for any natural number $n$. But it doesn't
follow from any finite subset of those. Any \emph{finite} subset of
those has a maximal number $n$ which is finite and such that if you
have just $n$ things, all the premises in that finite subset are true.
Therefore, any finite subset of those premises could be true without
the conclusion being true. The set of \emph{all} the premises, however,
can't be true without there being infinitely many premises. 
Therefore, if a logic can properly express the claims `there are at
least $n$ things' for any $n$ and the claim `there are finitely many
things', it has infinitary consequences. \emph{Second-order logic} is
of that kind.

Compactness follows from soudness and completeness, because proofs are
finite. Suppose $\Gamma \Entails !A$ in a complete logic. Since the
logic is complete, there's a proof of $!A$ from $\Gamma$ ($\Gamma
\Proves !A$). But a proof uses at most finitely many premises from
$\Gamma$. Therefore there are $!B_1,\ldots,!B_n$ in $\Gamma$ such that
$B_1,\ldots,!B_n\Proves !A$. Since the logic is sound,
$B_1,\ldots,!B_n\Entails !A$. So, in general, if $\Gamma \Entails !A$,
there's a finite subset $!B_1,\ldots,!B_n$ of $\Gamma$ that entails
$!A$.

Since soundness and completeness entail compactness, any (sound) logic
that is not compact is incomplete. Second-order logic is a case in point.
And any (sound) logic that is complete is compact, and therefore
cannot express the claims in the infinitary argument above. First-order 
logic and propositional logic are cases in point.