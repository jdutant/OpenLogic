% Part: cheat-sheets
% Chapter: first-order-logic
% Section: semantics

\documentclass[../../../../include/open-logic-section]{subfiles}

\begin{document}

\olfileid{chs}{fol}{nd}

% page for all rules at once
\newpage
\olsection{Natural Deduction}

\begin{figure}
    \begin{defish}
    \emph{All propositional logic rules are allowed.}

    \emph{When $!B$ can be derived from $!A_1,\ldots,!A_n$ using
    propositional logic rules alone:}
        \begin{prooftree}
            \AxiomC{}\DeduceC{$!A_1$}
            \AxiomC{}\DeduceC{$\ldots$}
            \AxiomC{}\DeduceC{$!A_n$}
            \RightLabel{\Log{PL}}
            \TrinaryInfC{$!B$}
        \end{prooftree}

    \emph{Quantifier rules}

    \smallskip\noindent
    {\setlength\extrarowheight{3em} 
    \begin{tabular}{cp{8em}cp{4em}}
        \multicolumn{2}{c}{
        \AxiomC{}\DeduceC{$\lforall[v][!A]$}
        \RightLabel{\Elim{\lforall}}
        \UnaryInfC{$!A[c/v]$}
        \DisplayProof
        }
    &
        \multicolumn{2}{c}{
        \AxiomC{}\DeduceC{$!A[c/v]$}
        \RightLabel{\Intro{\lexists}}
        \UnaryInfC{$\lexists[v][!A]$}
        \DisplayProof
        }
    \\
        \AxiomC{$\mathcal{D}$}\noLine
        \UnaryInfC{$!A[c/v]$}
        \RightLabel{\Intro{\lforall}}
        \UnaryInfC{$\lforall[v][!A]$}
        \DisplayProof

    &   
        \emph{Restrictions:}

        - $c$ not in undischarged assumptions of $\mathcal{D}$.

        - $c$ not in $!A$

    \\

        \AxiomC{$\lexists[v][!A]$}
            \AxiomC{$\Discharge{!A[c/v]}{n}$}
            \noLine
            \UnaryInfC{$\mathcal{D}$}
            \UnaryInfC{$!B$}
        \DischargeRule{\Elim{\lexists}}{n}
        \BinaryInfC{$!B$}
        \DisplayProof

    & 
        \emph{Restrictions:}

        - $c$ no in undischarged assumptions of $\mathcal{D}$ other
          than $!A[c/v]$.

        - $c$ not in $!A$

        - $c$ not in $!B$

    \\
    \end{tabular}
    }

    \smallskip\noindent
    \emph{Identity rules}

    \smallskip\noindent
    \begin{tabular}{ll}
        \AxiomC{}
        \RightLabel{\Intro{\eq}}
        \UnaryInfC{$\eq[c][c]$}
        \DisplayProof
        &
        \AxiomC{$\eq[c][d]$}
            \AxiomC{$!A[c/v]$}
        \RightLabel{\Elim{\eq}}
        \BinaryInfC{$!A[d/v]$}
        \DisplayProof
    \end{tabular}

    \end{defish}
    \caption{Natural Deduction rules for First Order Logic with Identity.}
    \ollabel{natdedfol}
\end{figure}
    
The rules are summarized in \olref{natdedfol}

\subsection{Propositional Rules}

You can use all the rules of propositional logic. In addition, you 
can omit the detail of propositional steps in your proofs by 
using the \Log{PL} rule:

\begin{defish}
    When $!B$ can be derived from $!A_1,\ldots,!A_n$ ($n\geq 0$) using
    propositional logic rules alone, we may simply write: 
    \begin{prooftree}
        \AxiomC{}\DeduceC{$!A_1$}
        \AxiomC{}\DeduceC{$\ldots$}
        \AxiomC{}\DeduceC{$!A_n$}
        \RightLabel{\Log{PL}}
        \TrinaryInfC{$!B$}
    \end{prooftree}
\end{defish}
    

\begin{ex}
    The following !!{derivation} of $!A\lor!B,\lnot!A\Proves{\Log{K}}!B$:
    \begin{prooftree}
    \AxiomC{$!A\lor!B$}
        \AxiomC{$\Discharge{!A}{1}$}
        \AxiomC{$\lnot!A$}
        \RightLabel{\Elim{\lnot}}
        \BinaryInfC{$!B$}
                \AxiomC{$!B$}
        \DischargeRule{\Elim{\lor}}{1}
        \TrinaryInfC{$!B$}
    \end{prooftree}
may be abbreviated:
\begin{prooftree}
    \AxiomC{$!A\lor!B$}
        \AxiomC{$\lnot!A$}
        \RightLabel{\Log{PL}}
    \BinaryInfC{$!B$}
\end{prooftree}
\end{ex}

    
Note that $\Log{PL}$ cannot be used to apply propositional steps 
to subformulas \emph{embedded} within quantifiers. For instance:
    
    \smallskip
    \begin{tabular}{cc}
        \AxiomC{$\lnot\lnot\lforall[x][\Atom{P}{x}]$}
        \RightLabel{\Log{PL} {\color{blue}Correct}}
        \UnaryInfC{$\lforall[x][\Atom{P}{x}]$}
        \DisplayProof
        &
        \AxiomC{$\lforall[x][\lnot\lnot\Atom{P}{x}]$}
        \RightLabel{\Log{PL} {\color{red}Incorrect}}
        \UnaryInfC{$\lforall[x][\Atom{P}{x}]$}
        \DisplayProof
    \end{tabular}
    \smallskip
    
    The first is correct because from !!a{formula} $\lnot\lnot!A$ you can
    derive $!A$ using propositional rules alone. It does not matter
    whether $!A$ contains a quantifier: you need not `unpack' it. 
    The second is incorrect because from !!a{formula} 
    $\lforall[x][\lnot\lnot\Atom{P}{x}]$ you cannot apply double negation
    elimination unless you `unpack' the quantifier $\lforall x$ first. 
    See the full !!{derivation}s below.
    
    \smallskip
    \begin{tabular}{cc}
        \AxiomC{$\lnot\lnot\lforall[x][\Atom{P}{x}]$}
        \AxiomC{$\Discharge{\lnot\lforall[x][\Atom{P}{x}]}{1}$}
        \DischargeRule{\Elim{\lnot}}{1}
        \BinaryInfC{$\lforall[x][\Atom{P}{x}]$}
        \DisplayProof
        &
        \AxiomC{$\lforall[x][\lnot\lnot\Atom{P}{x}]$}
        \RightLabel{\Elim{\lforall}}
        \UnaryInfC{$\lnot\lnot\Atom{P}{a}$}
            \AxiomC{$\Discharge{\lnot\Atom{P}{a}}{1}$}
        \DischargeRule{\Elim{\lnot}}{1}
        \BinaryInfC{$\Atom{P}{a}$}
        \RightLabel{\Intro{\lforall}}
        \UnaryInfC{$\lforall[x][\Atom{P}{x}]$}
        \DisplayProof
    \end{tabular}
    \smallskip

\subsection{Quantifier rules}

\subsubsection{Unrestricted rules}

Universal instantiation \Elim{\lforall} and existential generalization
\Intro{\lexists} are easy, because unrestricted.

\subsubsection{Restricted rules}

Universal generalization \Intro{\lforall} and existential instantiation
\Intro{\lexists} must be used with care, because they are restricted.
In both cases the restrictions are here to ensure $c$ 
represent an \emph{arbitrary} object. They do so by ensuring that 
whatever you derive could have been derived for any constant you 
would chose other than $c$. 

A safe way to apply to rule is to make sure you pick a $c$ that 
does not already appear anywhere else in the proof. For \Elim{\lforall} 
also make sure that the conclusion $!B$ you derive with the rule does 
not contain $c$ either.

\subsection{Identity rules}

The identity rules are Reflexivity and Leibniz's law. They are 
straightforward.

Note that Leibniz's law allow you to replace \emph{one or more} 
occurrences of $c$ in $!A$ with $d$.

\end{document}