% Part: cheat-sheets
% Chapter: modal-propositional-logics
% Section: syntax

\documentclass[../../../../include/open-logic-section]{subfiles}

\begin{document}

\olfileid{chs}{pml}{syn}

\olsection{Syntax}

\begin{defn}

\emph{!!^{formula}s} of the basic modal language are defined as in
propositional logic with the addition of:

    \begin{enumerate}
        \tagitem{prvBox}{If $!A$ is !!a{formula}, then $\Box !A$ is
        !!a{formula}.}{}
      
      \tagitem{prvDiamond}{If $!A$ is !!a{formula}, then $\Diamond !A$ is
        !!a{formula}.}{}
      \end{enumerate}

\end{defn}
    
\begin{tagblock}{defBox,defDiamond}
\begin{defn}
    Formulas constructed using the defined operators are to be understood
    as follows:
    
    \begin{tagenumerate}{defBox,defDiamond}
    
    \tagitem{defBox}{$\Box !A$ abbreviates $\lnot\Diamond\lnot !A$. }{}
    
    \tagitem{defDiamond}{$\Diamond !A$ abbreviates $\lnot\Box\lnot !A$.}{}
    \end{tagenumerate}
\end{defn}
\end{tagblock}

!!^a{formula} is \emph{modal-free} iff it does not contain $\Box$ or
$\Diamond$.
    
\end{document}