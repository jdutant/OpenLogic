% Part: normal-modal-logic
% Chapter: natural-deduction 
% Section: more-rules

\documentclass[../../../include/open-logic-section]{subfiles}

\begin{document}

\olfileid{nml}{nd}{mrl}

\olsection{Rules for stronger systems}

We provide rules for stronger systems.

\iftag{prvDiamond}{Except for \Ax{D}, rules come in two forms, e.g.
$\Box$\Ax{T} and $\Diamond$\Ax{T}. In each case only one is needed: 
the other can be derived (using Duality). However, we allow both 
forms to simplify our proofs.}{}

\subsection{Rules for \Log{D}, \Log{T}}

In \Log{D}, we relax the constraint that \Intro{\Diamond}\Ax{K} must
have a $\Diamond!A$ premise: this results in \Intro{\Diamond}\Ax{D}
below.

\begin{defish}

    \AxiomC{\emph{no undischarged assumption}}\DeduceC{$!C$}
    \RightLabel{\Intro{\Diamond}\Ax{D}}
    \UnaryInfC{$\Diamond!C$}
    \DisplayProof
\hfill
    \AxiomC{}\DeduceC{$\Box!A$}
    \RightLabel{\Ax{D}}
    \UnaryInfC{$\Diamond!A$}
    \DisplayProof
    \bigskip


    \AxiomC{}\DeduceC{$\Box!A$}
    \RightLabel{$\Box$\Ax{T}}
    \UnaryInfC{$!A$}
    \DisplayProof
\hfill
   \AxiomC{}\DeduceC{$!A$}
    \RightLabel{$\Diamond$\Ax{T}}
    \UnaryInfC{$\Diamond!A$}
    \DisplayProof

\end{defish}

The \Ax{D} rule is sound for \emph{serial} frames and \Ax{T} is sound
for \emph{reflexive} frames.

The \Ax{D} rule is !!{derivable} from the \Ax{T} rules:

\begin{prooftree}
\AxiomC{}\DeduceC{$\Box!A$}
\RightLabel{$\Box$\Ax{T}}
\UnaryInfC{$!A$}
\RightLabel{$\Diamond$\Ax{T}}
\UnaryInfC{$\Diamond!A$}
\end{prooftree}

\Log{D} is \Log{K} plus the \Ax{D} rule.

\Log{T} is \Log{K} plus the \Ax{T} rule. We also allow the \Ax{D} 
rule. 

\subsection{Rules for \Log{S4}, \Log{B}, \Log{S5}}

\begin{defish}
    \AxiomC{}\DeduceC{$\Box!A$}
    \RightLabel{$\Box$\Ax{4}}
    \UnaryInfC{$\Box\Box!A$}
    \DisplayProof
\hfill
    \AxiomC{}\DeduceC{$\Diamond\Diamond!A$}
    \RightLabel{$\Diamond$\Ax{4}}
    \UnaryInfC{$\Diamond!A$}
    \DisplayProof

\bigskip
    \AxiomC{}\DeduceC{$!A$}
    \RightLabel{$\Box$\Ax{B}}
    \UnaryInfC{$\Box\Diamond!A$}
    \DisplayProof
\hfill
    \AxiomC{}\DeduceC{$\Diamond\Box!A$}
    \RightLabel{$\Diamond$\Ax{B}}
    \UnaryInfC{$!A$}
    \DisplayProof

\bigskip
    \AxiomC{}\DeduceC{$\Diamond!A$}
    \RightLabel{$\Box$\Ax{5}}
    \UnaryInfC{$\Box\Diamond!A$}
    \DisplayProof
\hfill
    \AxiomC{}\DeduceC{$\Box!A$}
    \RightLabel{$\Diamond$\Ax{5}}
    \UnaryInfC{$\Diamond\Box!A$}
    \DisplayProof

\end{defish}

The \Ax{4} rules are sound for \emph{transitive} frames; the \Ax{B}
rules are sound for \emph{symmetric} frames and the \Ax{5} rules are
sound for \emph{euclidean} frames. 

Individually, rules \Ax{T}, \Ax{4}, \Ax{B} and \Ax{5} are independent
of each other. However, the following relations hold:
\begin{itemize}
    \item Given any two of \Ax{4}, \Ax{B} and \Ax{5} we can derive the third.
    \item Given \Ax{5} \emph{and \Ax{T}}, we can derive \Ax{4} and \Ax{B} rules.
\end{itemize}
From \Ax{5} alone we cannot derive \Ax{4} or \Ax{B}. From \Ax{T} and \Ax{4}
alone we cannot derive \Ax{B} nor \Ax{5}. From \Ax{T} and \Ax{B} we cannot
derive \Ax{4} nor \Ax{5}. 

\begin{ex}
Deriving $\Box$\Ax{B} from \Ax{4} and \Ax{5} rules:
\begin{prooftree}
\AxiomC{}\DeduceC{$\Diamond!A$}
        \AxiomC{$!A$}
        \RightLabel{$\Box$\Ax{B}}
        \UnaryInfC{$\Box\Diamond!A$}
        \RightLabel{$\Box$\Ax{5}}
        \UnaryInfC{$\Box\Box\Diamond!A$}
    \RightLabel{\Intro{\Box}}
    \BinaryInfC{$\Diamond\Box\Box\Diamond$}
    \RightLabel{$\Diamond$\Ax{B}}
    \UnaryInfC{$\Box\Diamond !A$}
\end{prooftree}
\end{ex}

\begin{prob}
Show that $\Box$\Ax{5} can be derived using the \Ax{4} and \Ax{B}
rules.
\end{prob}

\begin{prob}
Show that $\Box$\Ax{B} can be derived using the \Ax{T} and \Ax{5}
rules.
\end{prob}

\begin{prob}
Show that $\Box$\Ax{4} can be derived using the \Ax{T} and \Ax{5}
rules.
\end{prob}

\Log{S4} is \Log{T} plus the \Ax{4} rules. Therefore it has the \Ax{K},
\Ax{D}, \Ax{T} and \Ax{4} rules.

\Log{B} is \Log{T} plus the \Ax{B} rules. Therefore it has the \Ax{K},
\Ax{D}, \Ax{T} and \Ax{4} rules.

\Log{S5} is \Log{T} plus the \Ax{5} rules. Therefore it has 
all the rules we have seen: \Ax{K}, \Ax{D}, \Ax{T}, \Ax{4}, \Ax{B}
and \Ax{5} rules.

More generally, we can describe systems by listing their rules. 
\Log{KD45} is the system given by the \Ax{K}, \Ax{D}, \Ax{4} and \Ax{5}
rules. \Log{S4} can be called \Log{KT4} or \Log{KDT4}; \Log{S5} can 
be called \Log{KT5}, \Log{KT45}, \Log{KDTB45}, and many other names.

\end{document}
