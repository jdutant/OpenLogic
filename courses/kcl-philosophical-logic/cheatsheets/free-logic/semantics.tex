% Part: cheat-sheets
% Chapter: free-logic
% Section: introduction

\documentclass[../../../../include/open-logic-section]{subfiles}

\begin{document}

\olfileid{chs}{fre}{sem}

\olsection{Semantics}

Free logic allows \emph{empty constants}, i.e. constant that do not refer to 
any object in the domain of the quantifiers. This is required 
to ensure that:
$$\lexists[v][c=v]$$ isn't a logical truth for every constant $c$ (as
happens in first order logic).

\subsection{Semantics for Negative Free Logic}

Call this Negative Semantics. In Negative semantics empty constants
don't denote anything. $\Value{c}{M}$ is undefined when $c$ is empty.
The clauses for \emph{atomic} and \emph{identity} !!{formula}s are
changed accordingly to require that their terms are defined. The
clauses for quantifiers aren't changed, but we'll see that the
deduction rules for quantifiers must change. For
$\lforall[v][\Obj{F}v]\lif \Obj{F}a$ isn't valid as it is in first
order logic. 


\begin{defn}[!!^{structure}s]
    In the negative system of free logic, $\Log NFL$,
    \article{structure} \emph{!!{structure}}~$\Struct M$ for a language
    $\Lang L_{\Log{FL}}$ of free logic consists of:
    \begin{enumerate}
    \item \emph{Domain:} a set, $\Domain M$,
    \item \emph{Interpretation of !!{constant}s:} for any !!{constant}~$c$ of
      $\Lang L_{\Log{FL}}$, its interpretation $\Assign{c}{M}$ is !!a{element} of 
      $\Domain M$ or undefined,
    \item \emph{Interpretation of !!{predicate}s:} for each $n$-place
      !!{predicate}~$R$ of $\Lang L_{\Log{FL}}$ (other than $\eq$), an $n$-place
      relation $\Assign{R}{M} \subseteq \Domain{M}^n$
    \tagitem{fnTerms}{\item \emph{Interpretation of !!{function}s:} for each $n$-place
      !!{function}~$f$ of $\Lang L_{\Log{FL}}$, an $n$-place function $\Assign{f}{M}
      \colon \Domain{M}^n \to \Domain{M}$}{}
    \end{enumerate}
    The system of negative free logic is \emph{inclusive} if $\Domain M$ is
    allowed to be empty. There is only one such !!{structure}, for interpretations
    can only be defined one way when the domain is empty. We call it the 
    \emph{empty !!{structure}}.
    \end{defn}


    \begin{defn}[Variable Assignment]
        A \emph{variable assignment}~$s$ for a !!{structure}~$\Struct{M}$ is a
        (partial) function such that for any !!{variable}~$x$ of
        $\Lang L_{\Log{FL}}$, $s(x) \in \Domain M$ if defined. In addition, we stipulate
        an \emph{empty variable assignement} $s^{*}$ that doesn't assign any value to
        any variable.
        \end{defn}
        
        \begin{defn}[!!^{value} of Terms]
        If $t$ is a term of the language~$\Lang L_{\Log{FL}}$, $\Struct M$ is a
        !!{structure} for~$\Lang L_{\Log{FL}}$, and $s$ is a !!{variable} assignment
        for~$\Struct M$, the \emph{!!{value}}~$\Value{t}{M}[s]$ is defined as
        follows:
        \begin{enumerate}
        \item \indcase{t}{c}{$\Value{\indfrm}{M}[s] = \Assign{\indcomplex}{M}$, if defined.}
        \item \indcase{t}{x}{$\Value{\indfrm}{M}[s] = s(\indcomplex)$, if defined.}
        \tagitem{fnTerms}{\indcase{t}{\Atom{f}{t_1, \ldots, t_n}}{
        \[
        \Value{\indfrm}{M}[s] = \Assign{f}{M}(\Value{t_1}{M}[s], \ldots,
        \Value{t_n}{M}[s])\textrm{, if defined}.
        \]}
        }{}
        \end{enumerate}
\end{defn}


\begin{defn}[$x$-Variant]
    If $s$ is a !!{variable} assignment for a !!{structure}~$\Struct M$, we
    call !!a{variable} assignment $s'$ an \emph{$x$-variant} of $s$ iff:
    \begin{enumerate}
    \item it differs from $s$ at most in what it assigns for $s$, that is 
    either $\Value{y}{M}[s'] = \Value{y}{M}[s]$, or both are undefined, for
    any variable $y$ other than $x$.
    \item it is defined for $x$, that is, $\Value{y}{M}[s'] \in \Domain M$.
    We write $\varAssign{s'}{s}{x}$ when $s'$ is an $x$-variant of $s$.
    \end{enumerate}
\end{defn}
        
\begin{defn}[Satisfaction]
    \ollabel{defn:satisfaction}
    Satisfaction of a !!{formula}~$!A$ in a !!{structure}~$\Struct M$
    relative to a !!{variable} assignment~$s$, in symbols:
    $\Sat{M}{!A}[s]$, is defined recursively as follows. (We write
    $\Sat/{M}{!A}[s]$ to mean ``not $\Sat{M}{!A}[s]$.'')
    \begin{enumerate}
    \tagitem{prvFalse}{%
      \indcase{!A}{\lfalse}{$\Sat/{M}{\indfrm}[s]$.}}{}
    
    \tagitem{prvTrue}{%
      \indcase{!A}{\ltrue}{$\Sat{M}{\indfrm}[s]$.}}{}
    
    \item \indcase{!A}{\Atom{R}{t_1, \dots, t_n}}{$\Sat{M}{\indfrm}[s]$
      iff $\Value{t_i}{M}[s]$ is defined for any $1\leq i \leq n$ and 
      $\langle \Value{t_1}{M}[s], \dots, \Value{t_n}{M}[s] \rangle \in
      \Assign{R}{M}$.}
    
    \item \indcase{!A}{\Atom{\lfrexists}{t}}{$\Sat{M}{\indfrm}[s]$ iff 
      $\Value{t}{M}[s]$ is defined.}
    
    \item \indcase{!A}{\eq[t_1][t_2]}{$\Sat{M}{\indfrm}[s]$ iff 
      $\Value{t_1}{M}[s], \Value{t_2}{M}[s]$ are both defined and
      $\Value{t_1}{M}[s] = \Value{t_2}{M}[s]$.}
    
    \tagitem{prvNot}{%
      \indcase{!A}{\lnot !B}{$\Sat{M}{\indfrm}[s]$ iff
        $\Sat/{M}{!B}[s]$.}}{}
    
    \tagitem{prvAnd}{%
      \indcase{!A}{(!B \land !C)}{$\Sat{M}{\indfrm}[s]$ iff $\Sat{M}{!B}[s]$
        and $\Sat{M}{!C}[s]$.}}{}
    
    \tagitem{prvOr}{%
      \indcase{!A}{(!B \lor !C)}{$\Sat{M}{\indfrm}[s]$ iff
        $\Sat{M}{!A}[s]$ or $\Sat{M}{!B}[s]$ (or both).}}{}
    
    \tagitem{prvIf}{%
      \indcase{!A}{(!B \lif !C)}{$\Sat{M}{\indfrm}[s]$ iff $\Sat/{M}{!B}[s]$
        or $\Sat{M}{!C}[s]$ (or both).}}{}
    
    \tagitem{prvIff}{%
      \indcase{!A}{(!B \liff !C)}{$\Sat{M}{\indfrm}[s]$ iff either both
        $\Sat{M}{!B}[s]$ and $\Sat{M}{!C}[s]$, or neither $\Sat{M}{!B}[s]$
        nor $\Sat{M}{!C}[s]$.}}{}
    
    \tagitem{prvAll}{%
      \indcase{!A}{\lforall[x][!B]}{$\Sat{M}{\indfrm}[s]$ iff for every
        $x$-variant~$s'$ of $s$ \emph{for which $s'(x)$ is defined}, $\Sat{M}{!B}[s']$.}}{}
    
    \tagitem{prvEx}{%
      \indcase{!A}{\lexists[x][!B]}{$\Sat{M}{\indfrm}[s]$ iff there is an
        $x$-variant~$s'$ of $s$ \emph{for which $s'(x)$ is defined} such that $\Sat{M}{!B}[s']$.}}{}
    \end{enumerate}
\end{defn}
    
Truth and validity are defined as in first-order logic.
    
\begin{defn}[Truth]
Where $!A$ is a closed !!{formula}, we say that $A$ is \emph{true in}, or
\emph{satisfied by} a !!{structure} $\Struct M$, written $\Sat{M}{!A}$ iff 
$A$ is satisfied relative to any assignement in $\Struct M$:
$$\Sat{M}{!A}\textrm{ iff }\Sat{M}{!A}[s]\textrm{ for any }s\textrm{ in }\Struct M$$
\end{defn}

\begin{defn}[Validity]
    A !!{formula} $!A$ is \emph{valid}, $\Entails !A$, iff $\Sat{M}{!A}$ for every
    !!{structure}~$\Struct M$.
\end{defn}

\subsection{Semantics for Positive Free Logic}

Call these Positive Semantics. In Positive Semantics empty terms are 
assigned a referent but this referent is deemed 'non-existing'. 
We have two domains: the \emph{inner} domain over which quantifiers
range, and the \emph{outer} domain that includes, but is not restricted
to, the inner domain. Empty names are assigned to objects outside the 
inner domain. 

Positive semantics raise questions of interpretation: what are these
objects outside the inner domain? Do we take them seriously? If we 
don't, what is the real meaning of empty names? But you must first 
understand the semantics mechanics, leaving the philosophical questions 
aside. 

Predicates are allowed to take values from the outer domain. Thus
$\Obj{P}a$ or $\Obj{G}ab$ may be true in a !!{structure} even if $a$,
$a,b$ are empty names. Identity claims $\eq[c][d]$ are substantial
even if $c$, $d$ are empty, for they may or may not be true. 

Quantifiers only range over the inner domain, however. This is ensured
by restricting variable assignments to objects in the \emph{inner}
domain.


\begin{defn}[!!^{structure}s]
    In the positive system of free logic, $\Log NFL$,
    \article{structure} \emph{!!{structure}}~$\Struct M$ for a language
    $\Lang L_{\Log{FL}}$ of free logic consists of:
    \begin{enumerate}
    \item \emph{(Inner) Domain:} a set, $\Domain M$,
    \item \emph{Outer Domain:} a non-empty set, $\OuterDomain M$ that
        includes $\Domain M$ ($\Domain M \subseteq \OuterDomain M $),
    \item \emph{Interpretation of !!{constant}s:} for any !!{constant}~$c$ of
      $\Lang L_{\Log{FL}}$, its interpretation $\Assign{c}{M}$ is !!a{element} of 
      $\OuterDomain M$,
    \item \emph{Interpretation of !!{predicate}s:} for each $n$-place
      !!{predicate}~$R$ of $\Lang L_{\Log{FL}}$ (other than $\eq$), an $n$-place
      relation $\Assign{R}{M} \subseteq \OuterDomain M^n$
    \tagitem{fnTerms}{\item \emph{Interpretation of !!{function}s:} for each $n$-place
      !!{function}~$f$ of $\Lang L_{\Log{FL}}$, an $n$-place function $\Assign{f}{M}
      \colon \OuterDomain M^n \to \OuterDomain M$.}{}
    \end{enumerate}
    The system of positive free logic is \emph{inclusive} if $\Domain M$ is
    allowed to be empty. There are several such !!{structure}s, as they may
    still differ on the !!{element}s of $\OuterDomain M$ outside of $M$ and the 
    interpretations of constants and predicates. We call a !!{structure} with
    empty $\Domain M$ an \emph{empty structure}. 
\end{defn}
    
\begin{defn}[Variable Assignment]
    A \emph{variable assignment}~$s$ for a !!{structure}~$\Struct{M}$ is a
      function such that for any !!{variable}~$x$ of
      $\Lang L_{\Log{FL}}$, $s(x) \in \OuterDomain M$. 
\end{defn}
    
\begin{defn}[!!^{value} of Terms]
    If $t$ is a term of the language~$\Lang L_{\Log{FL}}$, $\Struct M$ is a
    !!{structure} for~$\Lang L_{\Log{FL}}$, and $s$ is a !!{variable} assignment
    for~$\Struct M$, the \emph{!!{value}}~$\Value{t}{M}[s]$ is defined as
    follows:
    \begin{enumerate}
    \item \indcase{t}{c}{$\Value{\indfrm}{M}[s] = \Assign{\indcomplex}{M}$.}
    \item \indcase{t}{x}{$\Value{\indfrm}{M}[s] = s(\indcomplex)$.}
    \tagitem{fnTerms}{\indcase{t}{\Atom{f}{t_1, \ldots, t_n}}{
    \[
    \Value{\indfrm}{M}[s] = \Assign{f}{M}(\Value{t_1}{M}[s], \ldots,
    \Value{t_n}{M}[s])\textrm{, if defined}.
    \]}
    }{}
    \end{enumerate}
\end{defn}
    
\begin{defn}[$x$-Variant]
    If $s$ is a !!{variable} assignment for a !!{structure}~$\Struct M$, we
    call !!a{variable} assignment $s'$ an \emph{$x$-variant} of $s$ iff:
    \begin{enumerate}
    \item it differs from $s$ at most in what it assigns for $s$, that is 
     $\Value{y}{M}[s'] = \Value{y}{M}[s]$, for any variable $y$ other 
     than $x$.
    \item it is defined for $x$, that is, $\Value{y}{M}[s'] \in \Domain M$.
    We write $\varAssign{s'}{s}{x}$ when $s'$ is an $x$-variant of $s$.
    \end{enumerate}
\end{defn}
    
\begin{defn}[Satisfaction]
    \ollabel{defn:satisfaction}
    Satisfaction of a !!{formula}~$!A$ in a !!{structure}~$\Struct M$
    relative to a !!{variable} assignment~$s$, in symbols:
    $\Sat{M}{!A}[s]$, is defined recursively as follows. (We write
    $\Sat/{M}{!A}[s]$ to mean ``not $\Sat{M}{!A}[s]$.'')
    \begin{enumerate}
    \tagitem{prvFalse}{%
      \indcase{!A}{\lfalse}{$\Sat/{M}{\indfrm}[s]$.}}{}
    
    \tagitem{prvTrue}{%
      \indcase{!A}{\ltrue}{$\Sat{M}{\indfrm}[s]$.}}{}
    
    \item \indcase{!A}{\Atom{R}{t_1, \dots, t_n}}{$\Sat{M}{\indfrm}[s]$
      iff $\langle \Value{t_1}{M}[s], \dots, \Value{t_n}{M}[s] \rangle \in
      \Assign{R}{M}$.}
    
    \item \indcase{!A}{\Atom{\lfrexists}{t}}{$\Sat{M}{\indfrm}[s]$ iff 
      $\Value{t}{M}[s] \in \Domain M$.}
    
    \item \indcase{!A}{\eq[t_1][t_2]}{$\Sat{M}{\indfrm}[s]$ iff 
      $\Value{t_1}{M}[s] = \Value{t_2}{M}[s]$.}
    
    \tagitem{prvNot}{%
      \indcase{!A}{\lnot !B}{$\Sat{M}{\indfrm}[s]$ iff
        $\Sat/{M}{!B}[s]$.}}{}
    
    \tagitem{prvAnd}{%
      \indcase{!A}{(!B \land !C)}{$\Sat{M}{\indfrm}[s]$ iff $\Sat{M}{!B}[s]$
        and $\Sat{M}{!C}[s]$.}}{}
    
    \tagitem{prvOr}{%
      \indcase{!A}{(!B \lor !C)}{$\Sat{M}{\indfrm}[s]$ iff
        $\Sat{M}{!A}[s]$ or $\Sat{M}{!B}[s]$ (or both).}}{}
    
    \tagitem{prvIf}{%
      \indcase{!A}{(!B \lif !C)}{$\Sat{M}{\indfrm}[s]$ iff $\Sat/{M}{!B}[s]$
        or $\Sat{M}{!C}[s]$ (or both).}}{}
    
    \tagitem{prvIff}{%
      \indcase{!A}{(!B \liff !C)}{$\Sat{M}{\indfrm}[s]$ iff either both
        $\Sat{M}{!B}[s]$ and $\Sat{M}{!C}[s]$, or neither $\Sat{M}{!B}[s]$
        nor $\Sat{M}{!C}[s]$.}}{}
    
    \tagitem{prvAll}{%
      \indcase{!A}{\lforall[x][!B]}{$\Sat{M}{\indfrm}[s]$ iff for every
        $x$-variant~$s'$ of $s$, $\Sat{M}{!B}[s']$.}}{}
    
    \tagitem{prvEx}{%
      \indcase{!A}{\lexists[x][!B]}{$\Sat{M}{\indfrm}[s]$ iff there is an
        $x$-variant~$s'$ of $s$ so that $\Sat{M}{!B}[s']$.}}{}
    \end{enumerate}
\end{defn}
    
Truth and validity are defined as usual.
   


\end{document}