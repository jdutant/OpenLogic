% Part: cheat-sheets
% Chapter: free-logic
% Section: introduction

\documentclass[../../../../include/open-logic-section]{subfiles}

\begin{document}

\olfileid{chs}{fre}{int}

\olsection{Introduction}

Free Logic is an alternative to first order logic that is `free of
existence assumptions'. 

\begin{itemize}
\item Its \emph{syntax} is the same as first order logic. We don't
repeat it here.
\item Its !!{structure}s are more general that those of first order logic. 
First order logic !!{structure}s are a special case of !!{structure}s
of free logic. 
\item Correspondingly, free logic is \emph{weaker} than first order 
logic: some arguments valid in first order logic are not valid in 
free logic, and some logical truths in first order logic are not valid
in free logic.
\item The converse is not true. Anything that is valid in free logic 
is also valid in first order logic. 
\end{itemize}

There are two different ways of developing free logic: \emph{negative}
and \emph{positive} free logic. Both extend classical propositional 
logic, but are weaker than first order logic. 

There is a third way, called \emph{neutral} free logic. But this also
gives up some elements of propositional logic (bivalene). We do not 
study it in this module. 

\end{document}