% Part: formatives
% Chapter: two
% Section: first-order-logic

\documentclass[../../../include/open-logic-section]{subfiles}

\begin{document}

\olfileid{for}{two}{fol}

\olsection{First Order Logic}

\begin{prob}

    Consider the model $\mathcal{M}$ with $\mathcal{D}=\{1,2,3\}$ and
    $\mathcal{I}(F)=\{1,3\}$, $\mathcal{I}(G)=\{\langle1,1\rangle,\langle2,1\rangle,\langle2,2\rangle,\langle3,2\rangle\}$.
    Say whether the following formulas are true in $\mathcal{M}$. No
    need to justify your answers.
    \begin{enumerate}
    \item $\forall x(Fx\lif\exists yGxy)$
    \item $\forall x\exists y(Gxx\lif Gyx)$
    \end{enumerate}

\end{prob}

\begin{prob}

    Formalize the following argument in first-order logic. Is the argument
    valid? If it is, give a natural deduction proof; otherwise a counterexample.
    \begin{quote}
    Every player followed some player whom Ali didn't see. 
    
    Ali saw every player who followed someone.
    
    Therefore: no player followed any player who followed someone.
    \end{quote}
\end{prob}

\begin{prob}
    Give natural deduction !!{derivation}s of the following.
    \begin{enumerate}
    \item $\forall x(Fx\lif\exists yGxy),\forall x\forall y(Gxy\lif Gyx)\vdash\forall x(Fx\lif\exists yGyx)$
    \item $\exists xGax\vdash\forall x(x=a\lif\exists yGxy)$
    \end{enumerate}
\end{prob}

\end{document}