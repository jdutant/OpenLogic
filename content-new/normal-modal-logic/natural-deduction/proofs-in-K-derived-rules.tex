% Part: normal-modal-logic
% Chapter: natural-deduction 
% Section: proofs-in-K-derived-rules

\documentclass[../../../include/open-logic-section]{subfiles}

\begin{document}

\olfileid{nml}{nd}{pdk}

\olsection{Examples of \usetoken{P}{derivation} in \Log{K} using
derived rules}

\begin{ex}
!!^{derivation} of $\Diamond (!A\lor!B)\Proves[\Log{K}]\Diamond!A\lor\Diamond!B$.

{\small
\begin{prooftree}
\AxiomC{$\Diamond (!A\lor!B)$}
    \AxiomC{$\Discharge{\lnot(\Diamond!A\lor\Diamond!B)}{2}$}
    \RightLabel{\Log{PL}}
    \UnaryInfC{$\lnot\Diamond!A$}
    \RightLabel{\Elim{\Diamond}\Ax{K}$^*$}
    \UnaryInfC{$\Box\lnot!A$}
        \AxiomC{$\Discharge{\lnot(\Diamond!A\lor\Diamond!B)}{2}$}
        \RightLabel{\Log{PL}}
        \UnaryInfC{$\lnot\Diamond!B$}
        \RightLabel{\Elim{\Diamond}\Ax{K}$^*$}
        \UnaryInfC{$\Box\lnot!B$}
            \AxiomC{$\Discharge{\lnot!A}{1}$}
            \AxiomC{$\Discharge{\lnot!B}{1}$}
            \RightLabel{\Log{PL}}
            \BinaryInfC{$\lnot(!A\lor!B)$}
    \DischargeRule{$\Box$\Ax{K}}{1}
    \TrinaryInfC{$\Box\lnot(!A\lor!B)$}
    \RightLabel{\Intro{\Diamond}\Ax{K}$^*$}
    \UnaryInfC{$\lnot\Diamond(!A\lor!B)$}
\RightLabel{\Elim{\lnot}}
\BinaryInfC{$\lfalse$}
\DischargeRule{\FalseCl}{2}
\UnaryInfC{$\Diamond!A\lor\Diamond!B$}
\end{prooftree}
}

\end{ex}

\begin{explain}
The overall strategy is a reductio. Our conclusion is a disjunction,
and as often, there is no hope of deriving either disjunct from the
premise alone: $\Diamond (!A\lor!B)\Entails/[\Log{K}]\Diamond!A$, and
similarly for $\Diamond !B$. We must use \emph{reductio} instead: we
assume $\lnot(\Diamond!A\lor\Diamond!B)$ and derive a contradiction. 

The first \Log{PL} step uses the fact that $\lnot(!A\lor!B)\Entails\lnot!A$
is valid in propositional logic alone, hence !!{derivable} using 
propositional rules alone. From this it of course follows that $\lnot\Diamond!A$
can be derived from $\lnot(\Diamond!A\lor\Diamond!B)$ using the 
propositional rules alone (just put $\Diamond!A$ for $!A$, $\Diamond!B$
for $!B$); therefore the \Log{PL} step is allowed. 

Similarly, the \Log{PL} step from $\lnot!A,\lnot!B$ to $\lnot(!A\lor!B)$
is allowed because $\lnot!A,\lnot!B\Entails\lnot(!A\lor!B)$ holds in 
propositional logic alone.

The Duality rules \Elim{\Diamond}\Ax{K}$^*$ and \Intro{\Diamond}\Ax{K}$^*$
simplify our !!{derivation} further. Without them we would need to use
the same steps as in their derivation, substantially lengthening our
proof.

\end{explain}


\end{document}